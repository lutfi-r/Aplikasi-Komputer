\documentclass[a4paper,10pt]{article}
\usepackage{eumat}

\begin{document}
\begin{eulernotebook}
\begin{eulercomment}
Nama : Muhammad Lutfi Ramadhan\\
NIM : 23030630021\\
Kelas : Matematika B 2023

\begin{eulercomment}
\eulerheading{EMT untuk Statistika}
\begin{eulercomment}
Di buku catatan ini, kami mendemonstrasikan plot statistik utama,
pengujian, dan distribusi di Euler.

Mari kita mulai dengan beberapa statistik deskriptif. Ini bukan
pengantar statistik. Jadi, Anda mungkin memerlukan latar belakang
untuk memahami detailnya.

Asumsikan pengukuran berikut. Kami ingin menghitung nilai rata-rata
dan deviasi standar yang diukur.
\end{eulercomment}
\begin{eulerprompt}
>M=[1000,1004,998,997,1002,1001,998,1004,998,997]; ...
>median(M), mean(M), dev(M),
\end{eulerprompt}
\begin{euleroutput}
  999
  999.9
  2.72641400622
\end{euleroutput}
\begin{eulercomment}
Kita dapat memplot plot kotak-dan-kumis untuk datanya. Dalam kasus
kami, tidak ada outlier.
\end{eulercomment}
\begin{eulerprompt}
>aspect(1.75); boxplot(M):
\end{eulerprompt}
\eulerimg{15}{images/Tugas Individu Pekan13-14_Muhammad Lutfi Ramadhan_23030630021-001.png}
\begin{eulercomment}
aspect (1.75) digunakan untuk mengatur rasio aspek dari plot
(perbandingan antara lebar dan tinggi.\\
boxplot(M) digunakan untuk mmebuat boxplot atau diagram kotak dari
data di dalam variabel M. Boxplot adalah visualisasi statistik yang
menunjukkan persebaran data, termasuk nilai minimum, median, dan nilai
maksimum.

Contoh, kita asumsikan jumlah pria berikut dalam rentang ukuran
tertentu.
\end{eulercomment}
\begin{eulerprompt}
>r=155.5:4:187.5; v=[22,71,136,169,139,71,32,8];
\end{eulerprompt}
\begin{eulercomment}
Berikut adalah alur pendistribusiannya.
\end{eulercomment}
\begin{eulerprompt}
>plot2d(r,v,a=150,b=200,c=0,d=190,bar=1,style="\(\backslash\)/"):
\end{eulerprompt}
\eulerimg{15}{images/Tugas Individu Pekan13-14_Muhammad Lutfi Ramadhan_23030630021-002.png}
\begin{eulercomment}
Kita bisa memasukkan data mentah tersebut ke dalam tabel.

Tabel adalah metode untuk menyimpan data statistik. Tabel kita harus
berisi tiga kolom: Awal jangkauan, akhir jangkauan, jumlah pria dalam
jangkauan.

Tabel dapat dicetak dengan header. Kami menggunakan vektor string
untuk mengatur header.
\end{eulercomment}
\begin{eulerprompt}
>T:=r[1:8]' | r[2:9]' | v'; writetable(T,labc=["BB","BA","Frek"])
\end{eulerprompt}
\begin{euleroutput}
          BB        BA      Frek
       155.5     159.5        22
       159.5     163.5        71
       163.5     167.5       136
       167.5     171.5       169
       171.5     175.5       139
       175.5     179.5        71
       179.5     183.5        32
       183.5     187.5         8
\end{euleroutput}
\begin{eulercomment}
Jika kita memerlukan nilai rata-rata dan statistik ukuran lainnya,
kita perlu menghitung titik tengah rentang tersebut. Kita bisa
menggunakan dua kolom pertama tabel kita untuk ini.

Sumbol "\textbar{}" digunakan untuk memisahkan kolom, fungsi "writetable"
digunakan untuk menulis tabel, dengan opsi "labc" untuk menentukan
header kolom.
\end{eulercomment}
\begin{eulerprompt}
>(T[,1]+T[,2])/2 
\end{eulerprompt}
\begin{euleroutput}
          157.5 
          161.5 
          165.5 
          169.5 
          173.5 
          177.5 
          181.5 
          185.5 
\end{euleroutput}
\begin{eulerprompt}
>M=fold(r,[0.5,0.5])
\end{eulerprompt}
\begin{euleroutput}
  [157.5,  161.5,  165.5,  169.5,  173.5,  177.5,  181.5,  185.5]
\end{euleroutput}
\begin{eulercomment}
Sekarang kita dapat menghitung mean dan deviasi sampel dengan
frekuensi tertentu.
\end{eulercomment}
\begin{eulerprompt}
>\{m,d\}=meandev(M,v); m, d,
\end{eulerprompt}
\begin{euleroutput}
  169.901234568
  5.98912964449
\end{euleroutput}
\begin{eulercomment}
Mari kita tambahkan distribusi nilai normal ke diagram batang di atas.
Rumus distribusi normal dengan mean m dan simpangan baku d adalah:

\end{eulercomment}
\begin{eulerformula}
\[
y=\frac{1}{d\sqrt{2\pi}}e^{\frac{-(x-m)^2}{2d^2}}.
\]
\end{eulerformula}
\begin{eulercomment}
Karena nilainya antara 0 dan 1, maka untuk memplotnya pada bar plot
harus dikalikan dengan 4 kali jumlah data.
\end{eulercomment}
\begin{eulerprompt}
>plot2d("qnormal(x,m,d)*sum(v)*4", ...
>xmin=min(r),xmax=max(r),thickness=3,add=1):
\end{eulerprompt}
\eulerimg{15}{images/Tugas Individu Pekan13-14_Muhammad Lutfi Ramadhan_23030630021-003.png}
\eulerheading{Tabel}
\begin{eulercomment}
Di direktori buku catatan ini Anda menemukan file dengan tabel. Data
tersebut merupakan hasil survei. Berikut adalah empat baris pertama
file tersebut. Datanya berasal dari buku online Jerman "Einführung in
die Statistik mit R" oleh A. Handl.
\end{eulercomment}
\begin{eulerprompt}
>printfile("table.dat",4);
\end{eulerprompt}
\begin{euleroutput}
  Person Sex Age Titanic Evaluation Tip Problem
  1 m 30 n . 1.80 n
  2 f 23 y g 1.80 n
  3 f 26 y g 1.80 y
\end{euleroutput}
\begin{eulercomment}
Tabel berisi 7 kolom angka atau token (string). Kami ingin membaca
tabel dari file. Pertama, kami menggunakan terjemahan kami sendiri
untuk tokennya.

Untuk ini, kami mendefinisikan kumpulan token. Fungsi strtokens()
mendapatkan vektor string token dari string tertentu.
\end{eulercomment}
\begin{eulerprompt}
>mf:=["m","f"]; yn:=["y","n"]; ev:=strtokens("g vg m b vb");
\end{eulerprompt}
\begin{eulercomment}
Sekarang kita membaca tabel dengan terjemahan ini.

Argumen tok2, tok4 dll. adalah terjemahan dari kolom tabel. Argumen
ini tidak ada dalam daftar parameter readtable(), jadi Anda perlu
menyediakannya dengan ":=".
\end{eulercomment}
\begin{eulerprompt}
>\{MT,hd\}=readtable("table.dat",tok2:=mf,tok4:=yn,tok5:=ev,tok7:=yn);
>load over statistics;
>writetable(MT[1:10],labc=hd,wc=5,tok2:=mf,tok4:=yn,tok5:=ev,tok7:=yn);
\end{eulerprompt}
\begin{euleroutput}
   Person  Sex  Age Titanic Evaluation  Tip Problem
        1    m   30       n          .  1.8       n
        2    f   23       y          g  1.8       n
        3    f   26       y          g  1.8       y
        4    m   33       n          .  2.8       n
        5    m   37       n          .  1.8       n
        6    m   28       y          g  2.8       y
        7    f   31       y         vg  2.8       n
        8    m   23       n          .  0.8       n
        9    f   24       y         vg  1.8       y
       10    m   26       n          .  1.8       n
\end{euleroutput}
\begin{eulercomment}
Titik "." mewakili nilai-nilai, yang tidak tersedia.

Jika kita tidak ingin menentukan token yang akan diterjemahkan
terlebih dahulu, kita hanya perlu menentukan, kolom mana yang berisi
token dan bukan angka.
\end{eulercomment}
\begin{eulerprompt}
>ctok=[2,4,5,7]; \{MT,hd,tok\}=readtable("table.dat",ctok=ctok);
\end{eulerprompt}
\begin{euleroutput}
  Could not open the file
  table.dat
  for reading!
  Try "trace errors" to inspect local variables after errors.
  readtable:
      if filename!=none then open(filename,"r"); endif;
\end{euleroutput}
\begin{eulercomment}
ctok=[2,4,5,7]: Ini adalah untuk menentukan kolom yang akan diambil
yaitu kolom ke-2, ke-4, ke-5, dan ke-7.
\end{eulercomment}
\begin{eulerprompt}
>tok
\end{eulerprompt}
\begin{euleroutput}
  m
  n
  f
  y
  g
  vg
\end{euleroutput}
\begin{eulercomment}
Tabel berisi entri dari file dengan token yang diterjemahkan ke dalam
angka.

String khusus NA = "." diartikan sebagai "Tidak Tersedia", dan
mendapatkan NAN (bukan angka) di tabel. Terjemahan ini dapat diubah
dengan parameter NA, dan NAval.
\end{eulercomment}
\begin{eulerprompt}
>MT[1]
\end{eulerprompt}
\begin{euleroutput}
  [1,  1,  30,  2,  NAN,  1.8,  2]
\end{euleroutput}
\begin{eulercomment}
Berikut isi tabel dengan nomor yang belum diterjemahkan.
\end{eulercomment}
\begin{eulerprompt}
>writetable(MT,wc=5)
\end{eulerprompt}
\begin{euleroutput}
      1    1   30    2    .  1.8    2
      2    3   23    4    5  1.8    2
      3    3   26    4    5  1.8    4
      4    1   33    2    .  2.8    2
      5    1   37    2    .  1.8    2
      6    1   28    4    5  2.8    4
      7    3   31    4    6  2.8    2
      8    1   23    2    .  0.8    2
      9    3   24    4    6  1.8    4
     10    1   26    2    .  1.8    2
     11    3   23    4    6  1.8    4
     12    1   32    4    5  1.8    2
     13    1   29    4    6  1.8    4
     14    3   25    4    5  1.8    4
     15    3   31    4    5  0.8    2
     16    1   26    4    5  2.8    2
     17    1   37    2    .  3.8    2
     18    1   38    4    5    .    2
     19    3   29    2    .  3.8    2
     20    3   28    4    6  1.8    2
     21    3   28    4    1  2.8    4
     22    3   28    4    6  1.8    4
     23    3   38    4    5  2.8    2
     24    3   27    4    1  1.8    4
     25    1   27    2    .  2.8    4
\end{euleroutput}
\begin{eulercomment}
Untuk kenyamanan, Anda dapat memasukkan keluaran readtable() ke dalam
daftar.
\end{eulercomment}
\begin{eulerprompt}
>Table=\{\{readtable("table.dat",ctok=ctok)\}\};
\end{eulerprompt}
\begin{euleroutput}
  Could not open the file
  table.dat
  for reading!
  Try "trace errors" to inspect local variables after errors.
  readtable:
      if filename!=none then open(filename,"r"); endif;
\end{euleroutput}
\begin{eulercomment}
Dengan menggunakan kolom token yang sama dan token yang dibaca dari
file, kita dapat mencetak tabel. Kita dapat menentukan ctok, tok, dll.
atau menggunakan tabel daftar.
\end{eulercomment}
\begin{eulerprompt}
>writetable(Table,ctok=ctok,wc=5);
\end{eulerprompt}
\begin{euleroutput}
   Person  Sex  Age Titanic Evaluation  Tip Problem
        1    m   30       n          .  1.8       n
        2    f   23       y          g  1.8       n
        3    f   26       y          g  1.8       y
        4    m   33       n          .  2.8       n
        5    m   37       n          .  1.8       n
        6    m   28       y          g  2.8       y
        7    f   31       y         vg  2.8       n
        8    m   23       n          .  0.8       n
        9    f   24       y         vg  1.8       y
       10    m   26       n          .  1.8       n
       11    f   23       y         vg  1.8       y
       12    m   32       y          g  1.8       n
       13    m   29       y         vg  1.8       y
       14    f   25       y          g  1.8       y
       15    f   31       y          g  0.8       n
       16    m   26       y          g  2.8       n
       17    m   37       n          .  3.8       n
       18    m   38       y          g    .       n
       19    f   29       n          .  3.8       n
       20    f   28       y         vg  1.8       n
       21    f   28       y          m  2.8       y
       22    f   28       y         vg  1.8       y
       23    f   38       y          g  2.8       n
       24    f   27       y          m  1.8       y
       25    m   27       n          .  2.8       y
\end{euleroutput}
\begin{eulercomment}
Fungsi tablecol() mengembalikan nilai kolom tabel, melewatkan baris
apa pun dengan nilai NAN ("." dalam file), dan indeks kolom, yang
berisi nilai-nilai ini.
\end{eulercomment}
\begin{eulerprompt}
>\{c,i\}=tablecol(MT,[5,6]);
\end{eulerprompt}
\begin{eulercomment}
Kita bisa menggunakan ini untuk mengekstrak kolom dari tabel untuk
tabel baru.
\end{eulercomment}
\begin{eulerprompt}
>j=[1,5,6]; writetable(MT[i,j],labc=hd[j],ctok=[2],tok=tok)
\end{eulerprompt}
\begin{euleroutput}
      Person Evaluation       Tip
           2          g       1.8
           3          g       1.8
           6          g       2.8
           7         vg       2.8
           9         vg       1.8
          11         vg       1.8
          12          g       1.8
          13         vg       1.8
          14          g       1.8
          15          g       0.8
          16          g       2.8
          20         vg       1.8
          21          m       2.8
          22         vg       1.8
          23          g       2.8
          24          m       1.8
\end{euleroutput}
\begin{eulercomment}
Tentu saja, kita perlu mengekstrak tabel itu sendiri dari daftar Tabel
dalam kasus ini.
\end{eulercomment}
\begin{eulerprompt}
>MT=Table[1];
\end{eulerprompt}
\begin{eulercomment}
Tentu saja, kita juga dapat menggunakannya untuk menentukan nilai
rata-rata suatu kolom atau nilai statistik lainnya.
\end{eulercomment}
\begin{eulerprompt}
>mean(tablecol(MT,6))
\end{eulerprompt}
\begin{euleroutput}
  2.175
\end{euleroutput}
\begin{eulercomment}
Fungsi getstatistics() mengembalikan elemen dalam vektor, dan
jumlahnya. Kami menerapkannya pada nilai "m" dan "f" di kolom kedua
tabel kami.
\end{eulercomment}
\begin{eulerprompt}
>\{xu,count\}=getstatistics(tablecol(MT,2)); xu, count,
\end{eulerprompt}
\begin{euleroutput}
  [1,  3]
  [12,  13]
\end{euleroutput}
\begin{eulercomment}
Kita bisa mencetak hasilnya di tabel baru.
\end{eulercomment}
\begin{eulerprompt}
>writetable(count',labr=tok[xu])
\end{eulerprompt}
\begin{euleroutput}
           m        12
           f        13
\end{euleroutput}
\begin{eulercomment}
Fungsi selecttable() mengembalikan tabel baru dengan nilai dalam satu
kolom yang dipilih dari vektor indeks. Pertama kita mencari indeks
dari dua nilai kita di tabel token.
\end{eulercomment}
\begin{eulerprompt}
>v:=indexof(tok,["g","vg"])
\end{eulerprompt}
\begin{euleroutput}
  [5,  6]
\end{euleroutput}
\begin{eulercomment}
Sekarang kita dapat memilih baris tabel, yang memiliki salah satu
nilai v pada baris ke-5.
\end{eulercomment}
\begin{eulerprompt}
>MT1:=MT[selectrows(MT,5,v)]; i:=sortedrows(MT1,5);
\end{eulerprompt}
\begin{eulercomment}
Sekarang kita dapat mencetak tabel, dengan nilai yang diekstraksi dan
diurutkan di kolom ke-5.
\end{eulercomment}
\begin{eulerprompt}
>writetable(MT1[i],labc=hd,ctok=ctok,tok=tok,wc=7);
\end{eulerprompt}
\begin{euleroutput}
   Person    Sex    Age Titanic Evaluation    Tip Problem
        2      f     23       y          g    1.8       n
        3      f     26       y          g    1.8       y
        6      m     28       y          g    2.8       y
       18      m     38       y          g      .       n
       16      m     26       y          g    2.8       n
       15      f     31       y          g    0.8       n
       12      m     32       y          g    1.8       n
       23      f     38       y          g    2.8       n
       14      f     25       y          g    1.8       y
        9      f     24       y         vg    1.8       y
        7      f     31       y         vg    2.8       n
       20      f     28       y         vg    1.8       n
       22      f     28       y         vg    1.8       y
       13      m     29       y         vg    1.8       y
       11      f     23       y         vg    1.8       y
\end{euleroutput}
\begin{eulercomment}
Untuk statistik selanjutnya, kami ingin menghubungkan dua kolom tabel.
Jadi kita ekstrak kolom 2 dan 4 dan urutkan tabelnya.
\end{eulercomment}
\begin{eulerprompt}
>i=sortedrows(MT,[2,4]);  ...
>  writetable(tablecol(MT[i],[2,4])',ctok=[1,2],tok=tok)
\end{eulerprompt}
\begin{euleroutput}
           m         n
           m         n
           m         n
           m         n
           m         n
           m         n
           m         n
           m         y
           m         y
           m         y
           m         y
           m         y
           f         n
           f         y
           f         y
           f         y
           f         y
           f         y
           f         y
           f         y
           f         y
           f         y
           f         y
           f         y
           f         y
\end{euleroutput}
\begin{eulercomment}
Dengan getstatistics(), kita juga bisa menghubungkan jumlah dalam dua
kolom tabel satu sama lain.
\end{eulercomment}
\begin{eulerprompt}
>MT24=tablecol(MT,[2,4]); ...
>\{xu1,xu2,count\}=getstatistics(MT24[1],MT24[2]); ...
>writetable(count,labr=tok[xu1],labc=tok[xu2])
\end{eulerprompt}
\begin{euleroutput}
                     n         y
           m         7         5
           f         1        12
\end{euleroutput}
\begin{eulercomment}
Sebuah tabel dapat ditulis ke file.
\end{eulercomment}
\begin{eulerprompt}
>filename="test.dat"; ...
>writetable(count,labr=tok[xu1],labc=tok[xu2],file=filename);
\end{eulerprompt}
\begin{eulercomment}
Kemudian kita bisa membaca tabel dari file tersebut.
\end{eulercomment}
\begin{eulerprompt}
>\{MT2,hd,tok2,hdr\}=readtable(filename,>clabs,>rlabs); ...
>writetable(MT2,labr=hdr,labc=hd)
\end{eulerprompt}
\begin{euleroutput}
                     n         y
           m         7         5
           f         1        12
\end{euleroutput}
\begin{eulercomment}
Dan hapus filenya.
\end{eulercomment}
\begin{eulerprompt}
>fileremove(filename);
\end{eulerprompt}
\eulerheading{Distribusi}
\begin{eulercomment}
Dengan plot2d, ada metode yang sangat mudah untuk memplot sebaran data
eksperimen.
\end{eulercomment}
\begin{eulerprompt}
>p=normal(1,1000); 
>plot2d(p,distribution=20,style="\(\backslash\)/");
>plot2d("qnormal(x,0,1)",add=1): 
\end{eulerprompt}
\eulerimg{15}{images/Tugas Individu Pekan13-14_Muhammad Lutfi Ramadhan_23030630021-004.png}
\begin{eulercomment}
p=normal(1,1000); digunakan untuk menciptakan 1000 sampel acak yang
terdistribusi normal dengan mean (rata-rata) 1 dan standar deviasi
1000.\\
plot2d("qnormal(x,0,1)",add=1);\\
digunakan untuk menambahkan plot dari distribusi normal standar
(dengan mean 0 dan standar deviasi 1) ke grafik yang sama. Fungsi
qnormal(x,0,1) mengacu pada distribusi kumulatif dari variabel acak
normal standar. add=1 menunjukkan bahwa grafik ini harus ditambahkan
ke grafik yang sudah ada, bukan dibuat baru.

Perlu diperhatikan perbedaan antara bar plot (sampel) dan kurva normal
(distribusi sebenarnya). Masukkan kembali ketiga perintah untuk
melihat hasil pengambilan sampel lainnya.

Berikut adalah perbandingan 10 simulasi dari 1000 nilai terdistribusi
normal menggunakan apa yang disebut plot kotak. Plot ini menunjukkan
median, kuartil 25\% dan 75\%, nilai minimal dan maksimal, serta
outlier.
\end{eulercomment}
\begin{eulerprompt}
>p=normal(10,1000); boxplot(p):
\end{eulerprompt}
\eulerimg{15}{images/Tugas Individu Pekan13-14_Muhammad Lutfi Ramadhan_23030630021-005.png}
\begin{eulercomment}
Untuk menghasilkan bilangan bulat acak, Euler memiliki intrandom. Mari
kita simulasikan lemparan dadu dan plot distribusinya.

Kita menggunakan fungsi getmultiplicities(v,x), yang menghitung
seberapa sering elemen v muncul di x. Kemudian kita plot hasilnya
menggunakan kolomplot().
\end{eulercomment}
\begin{eulerprompt}
>k=intrandom(1,6000,6);  ...
>columnsplot(getmultiplicities(1:6,k));  ...
>ygrid(1000,color=red):
\end{eulerprompt}
\eulerimg{15}{images/Tugas Individu Pekan13-14_Muhammad Lutfi Ramadhan_23030630021-006.png}
\begin{eulercomment}
Meskipun inrandom(n,m,k) mengembalikan bilangan bulat yang
terdistribusi secara seragam dari 1 hingga k, distribusi bilangan
bulat lainnya dapat digunakan dengan randpint().

Dalam contoh berikut, probabilitas untuk 1,2,3 masing-masing adalah
0,4,0.1,0.5.
\end{eulercomment}
\begin{eulerprompt}
>randpint(1,1000,[0.4,0.1,0.5]); getmultiplicities(1:3,%)
\end{eulerprompt}
\begin{euleroutput}
  [378,  102,  520]
\end{euleroutput}
\begin{eulercomment}
Euler dapat menghasilkan nilai acak dari lebih banyak distribusi.
Lihat referensinya.

Misalnya, kita mencoba distribusi eksponensial. Variabel acak kontinu
X dikatakan berdistribusi eksponensial, jika PDF-nya diberikan oleh

\end{eulercomment}
\begin{eulerformula}
\[
f_X(x)=\lambda e^{-\lambda x},\quad x>0,\quad \lambda>0,
\]
\end{eulerformula}
\begin{eulercomment}
with parameter\\
\end{eulercomment}
\begin{eulerformula}
\[
\lambda=\frac{1}{\mu},\quad \mu \text{ is the mean, and denoted by } X \sim \text{Exponential}(\lambda).
\]
\end{eulerformula}
\begin{eulerprompt}
>plot2d(randexponential(1,1000,2),>distribution):
\end{eulerprompt}
\eulerimg{15}{images/Tugas Individu Pekan13-14_Muhammad Lutfi Ramadhan_23030630021-009.png}
\begin{eulercomment}
Parameter pertama (1) adalah lambda, yang merupakan parameter
distribusi eksponensial.\\
Parameter kedua (1000) menunjukkan jumlah angka acak yang dihasilkan.\\
Parameter ketiga (2) bisa menunjukkan dimensi atau bentuk output.

Untuk banyak distribusi, Euler dapat menghitung fungsi distribusi dan
inversnya.
\end{eulercomment}
\begin{eulerprompt}
>plot2d("normaldis",-4,4): 
\end{eulerprompt}
\eulerimg{15}{images/Tugas Individu Pekan13-14_Muhammad Lutfi Ramadhan_23030630021-010.png}
\begin{eulercomment}
Berikut ini adalah salah satu cara untuk memplot kuantil.
\end{eulercomment}
\begin{eulerprompt}
>plot2d("qnormal(x,1,1.5)",-4,6);  ...
>plot2d("qnormal(x,1,1.5)",a=2,b=5,>add,>filled):
\end{eulerprompt}
\eulerimg{15}{images/Tugas Individu Pekan13-14_Muhammad Lutfi Ramadhan_23030630021-011.png}
\begin{eulerformula}
\[
\text{normaldis(x,m,d)}=\int_{-\infty}^x \frac{1}{d\sqrt{2\pi}}e^{-\frac{1}{2}(\frac{t-m}{d})^2}\ dt.
\]
\end{eulerformula}
\begin{eulercomment}
Peluang berada di kawasan hijau adalah sebagai berikut.
\end{eulercomment}
\begin{eulerprompt}
>normaldis(5,1,1.5)-normaldis(2,1,1.5)
\end{eulerprompt}
\begin{euleroutput}
  0.248662156979
\end{euleroutput}
\begin{eulercomment}
Ini dapat dihitung secara numerik dengan integral berikut.\\
\end{eulercomment}
\begin{eulerformula}
\[
\int_2^5 \frac{1}{1.5\sqrt{2\pi}}e^{-\frac{1}{2}(\frac{x-1}{1.5})^2}\ dx.
\]
\end{eulerformula}
\begin{eulerprompt}
>gauss("qnormal(x,1,1.5)",2,5)
\end{eulerprompt}
\begin{euleroutput}
  0.248662156979
\end{euleroutput}
\begin{eulercomment}
Mari kita bandingkan distribusi binomial dengan distribusi normal yang
mean dan deviasinya sama. Fungsi invbindis() menyelesaikan interpolasi
linier antara nilai integer.
\end{eulercomment}
\begin{eulerprompt}
>invbindis(0.95,1000,0.5), invnormaldis(0.95,500,0.5*sqrt(1000))
\end{eulerprompt}
\begin{euleroutput}
  525.516721219
  526.007419394
\end{euleroutput}
\begin{eulercomment}
Fungsi qdis() adalah kepadatan distribusi chi-kuadrat. Seperti biasa,
Euler memetakan vektor ke fungsi ini. Dengan demikian kita mendapatkan
plot semua distribusi chi-kuadrat dengan derajat 5 sampai 30 dengan
mudah dengan cara berikut.
\end{eulercomment}
\begin{eulerprompt}
>plot2d("qchidis(x,(5:5:50)')",0,50):
\end{eulerprompt}
\eulerimg{15}{images/Tugas Individu Pekan13-14_Muhammad Lutfi Ramadhan_23030630021-012.png}
\begin{eulercomment}
Euler memiliki fungsi akurat untuk mengevaluasi distribusi. Mari kita
periksa chidis() dengan integral.

Penamaannya mencoba untuk konsisten. Misalnya.,

- distribusi chi-kuadratnya adalah chidis(),\\
- fungsi kebalikannya adalah invchidis(),\\
- kepadatannya adalah qchidis().

Pelengkap distribusi (ekor atas) adalah chicdis().
\end{eulercomment}
\begin{eulerprompt}
>chidis(1.5,2), integrate("qchidis(x,2)",0,1.5)
\end{eulerprompt}
\begin{euleroutput}
  0.527633447259
  0.527633447259
\end{euleroutput}
\eulerheading{Distribusi Diskrit}
\begin{eulercomment}
Distribusi diskret adalah jenis distribusi probabilitas yang digunakan
untuk variabel acak diskret, yaitu variabel yang hanya dapat memiliki
nilai tertentu, biasanya dalam bentuk bilangan bulat.

Untuk menentukan distribusi diskrit Anda sendiri, Anda dapat
menggunakan metode berikut.

Pertama kita atur fungsi distribusinya.
\end{eulercomment}
\begin{eulerprompt}
>wd = 0|((1:6)+[-0.01,0.01,0,0,0,0])/6
\end{eulerprompt}
\begin{euleroutput}
  [0,  0.165,  0.335,  0.5,  0.666667,  0.833333,  1]
\end{euleroutput}
\begin{eulercomment}
Perintah ini menggunakan operator \textbar{} dan + untuk membuat nilai dalam
variabel wd.

1:6 Ini menghasilkan vektor [1, 2, 3, 4, 5, 6].\\
(1:6) + [-0.01, 0.01, 0, 0, 0, 0]: Operasi ini menambahkan kedua
vektor elemen per elemen.\\
Hasilnya: \\
\end{eulercomment}
\begin{eulerformula}
\[
[1-0.01, 2+0.01, 3, 4, 5, 6] = [0.99, 2.01, 3, 4, 5, 6]
\]
\end{eulerformula}
\begin{eulercomment}
[1-0.01,2+0.01,3,4,5,6]=[0.99,2.01,3,4,5,6]/6 Membagi setiap elemen
hasil penjumlahan tadi dengan 6.\\
Hasilnya:\\
\end{eulercomment}
\begin{eulerformula}
\[
[\frac {0.99}{6}, \frac {2.01}{6}, \frac {3}{6}, \frac {4}{6}, \frac {5}{6}, \frac {6}{6}] = [0.165, 0.335, 0.5,0.6667, 0.8333, 1]
\]
\end{eulerformula}
\begin{eulercomment}
Artinya dengan probabilitas wd[i+1]-wd[i] kita menghasilkan nilai acak
i.

Ini hampir merupakan distribusi yang seragam. Mari kita tentukan
generator nomor acak untuk ini. Fungsi find(v,x) mencari nilai x pada
vektor v. Fungsi ini juga berfungsi untuk vektor x.
\end{eulercomment}
\begin{eulerprompt}
>function wrongdice (n,m) := find(wd,random(n,m))
\end{eulerprompt}
\begin{eulercomment}
Kesalahannya sangat halus sehingga kita hanya melihatnya dengan banyak
iterasi.

Fungsi wrongdice mengembalikan sebuah matriks berukuran n x m, di mana
setiap elemen dari matriks ini adalah indeks posisi dari elemen wd
yang paling sesuai (atau mendekati) nilai acak dari random(n, m).
\end{eulercomment}
\begin{eulerprompt}
>columnsplot(getmultiplicities(1:6,wrongdice(1,1000000))):
\end{eulerprompt}
\begin{euleroutput}
  Variable or function wd not found.
  Try "trace errors" to inspect local variables after errors.
  wrongdice:
      useglobal; return find(wd,random(n,m)) 
  Error in:
  ... nsplot(getmultiplicities(1:6,wrongdice(1,1000000))): ...
                                                       ^
\end{euleroutput}
\begin{eulercomment}
Hasil columnsplot akan menunjukkan frekuensi relatif dari setiap angka
(1 hingga 6), yang memungkinkan Anda untuk melihat apakah distribusi
itu merata atau tidak.

Berikut adalah fungsi sederhana untuk memeriksa keseragaman distribusi
nilai 1...K dalam v. Kita menerima hasilnya, jika untuk semua
frekuensi

\end{eulercomment}
\begin{eulerformula}
\[
\left|f_i-\frac{1}{K}\right| < \frac{\delta}{\sqrt{n}}.
\]
\end{eulerformula}
\begin{eulercomment}
Metode tersebut merupakan metode statistik untuk menguji keseragaman
distribusi. Distribusi dianggap seragam jika frekuensi setiap nilai
dalam v mendekati frekuensi ideal 1/K, dengan deviasi yang tidak
melebihi batas toleransi.
\end{eulercomment}
\begin{eulerprompt}
>function checkrandom (v, delta=1) ...
\end{eulerprompt}
\begin{eulerudf}
    K=max(v); n=cols(v);
    fr=getfrequencies(v,1:K);
    return max(fr/n-1/K)<delta/sqrt(n);
    endfunction
\end{eulerudf}
\begin{eulercomment}
Memang fungsinya menolak distribusi seragam.
\end{eulercomment}
\begin{eulerprompt}
>checkrandom(wrongdice(1,1000000))
\end{eulerprompt}
\begin{euleroutput}
  0
\end{euleroutput}
\begin{eulercomment}
Dan ia menerima generator acak bawaan.

Manual:\\
- Asumsi dadu, maka peluang setiap sisi = 1/6\\
Dalam 1 juta lemparan maka\\
\end{eulercomment}
\begin{eulerformula}
\[
1000000 \times \frac{1}{6} \approx 166667
\]
\end{eulerformula}
\begin{eulercomment}
- Frekuensi setiap sisi fr. Proporsi tiap sisi = fr/n\\
Misalkan frekuensi munculnya angka adalah\\
\end{eulercomment}
\begin{eulerformula}
\[
[160000, 170000, 180000, 150000, 170000, 170000]
\]
\end{eulerformula}
\begin{eulercomment}
Maka proporsi setiap angka:\\
\end{eulercomment}
\begin{eulerformula}
\[
\frac{[160000, 170000, 180000, 150000, 170000, 170000]}{1000000}
\]
\end{eulerformula}
\begin{eulerformula}
\[
[0.16, 0.17, 0.18, 0.15, 0.17, 0.17]
\]
\end{eulerformula}
\begin{eulercomment}
- Deviasi maksimum fn/n - 1/K\\
\end{eulercomment}
\begin{eulerformula}
\[
\frac{1}{K} = \frac{1}{6} = 0.1667
\]
\end{eulerformula}
\begin{eulerformula}
\[
([0.16, 0.17, 0.18, 0.15, 0.17, 0.17]-0.1667)
\]
\end{eulerformula}
\begin{eulerformula}
\[
max(-0.0067, 0.0033, 0.0133, -0.0167, 0.0033, 0.0033)= 0.0133
\]
\end{eulerformula}
\begin{eulercomment}
- Bandingkan dengan batas toleransi.\\
\end{eulercomment}
\begin{eulerformula}
\[
Batas= \frac{delta}{\sqrt{n}} = \frac{1}{\sqrt{1000000}} = \frac{1}{1000} = 0.001
\]
\end{eulerformula}
\begin{eulerformula}
\[
0.0133>0.001
\]
\end{eulerformula}
\begin{eulercomment}
Hasil 0 di sini mengindikasikan bahwa fungsi checkrandom telah
menentukan bahwa distribusi tidak seragam.

\end{eulercomment}
\begin{eulerprompt}
>checkrandom(intrandom(1,1000000,6))
\end{eulerprompt}
\begin{euleroutput}
  1
\end{euleroutput}
\begin{eulercomment}
checkrandom mengembalikan 1 atau true yang berarti bahwa distribusi
dari 1 juta bilangan acak rentang 1 sampai 6 dianggap cukup seragam
dalam batas toleransi yang ditetapkan.

Kita dapat menghitung distribusi binomial. Pertama ada binomialsum(),
yang mengembalikan probabilitas i atau kurang hit dari n percobaan.

Misal kita akan menghitung probabilitas dari distribusi binomial di
mana terdapat 1000 percobaan (misalnya, 1000 kali pelemparan koin),
dengan probabilitas sukses pada setiap percobaan sebesar 0.4, dan kita
ingin mengetahui probabilitas mendapatkan tepat 410 sukses.\\
Secara matematis, ini dihitung dengan rumus:\\
\end{eulercomment}
\begin{eulerformula}
\[
P(X \leq 410)= \binom{1000}{410} \cdot (0.4)^{410} \cdot (0.6)^{1000-410}
\]
\end{eulerformula}
\begin{eulerprompt}
>bindis(410,1000,0.4)
\end{eulerprompt}
\begin{euleroutput}
  0.751401349654
\end{euleroutput}
\begin{eulerprompt}
>bindis(4,10,0.6)
\end{eulerprompt}
\begin{euleroutput}
  0.1662386176
\end{euleroutput}
\begin{eulercomment}
Manual:\\
Secara matematis, ini dihitung dengan rumus:\\
\end{eulercomment}
\begin{eulerformula}
\[
P(X \leq 4)= \binom{10}{4} \cdot (0.6)^{4} \cdot (0.4)^{10-4}
\]
\end{eulerformula}
\begin{eulercomment}
- Untuk k = 0\\
\end{eulercomment}
\begin{eulerformula}
\[
P(X=0)= \binom{10}{0} \cdot (0.6)^{0} \cdot (0.4)^{10} \approx 0.00010
\]
\end{eulerformula}
\begin{eulercomment}
- Untuk k = 1\\
\end{eulercomment}
\begin{eulerformula}
\[
P(X = 1)= \binom{10}{1} \cdot (0.6)^{1} \cdot (0.4)^{9} \approx 0.00157
\]
\end{eulerformula}
\begin{eulercomment}
- Untuk k = 2\\
\end{eulercomment}
\begin{eulerformula}
\[
P(X = 2)= \binom{10}{2} \cdot (0.6)^{2} \cdot (0.4)^{8} \approx 0.01061
\]
\end{eulerformula}
\begin{eulercomment}
- Untuk k = 3\\
\end{eulercomment}
\begin{eulerformula}
\[
P(X = 3)= \binom{10}{3} \cdot (0.6)^{3} \cdot (0.4)^{7} \approx 0.04246
\]
\end{eulerformula}
\begin{eulercomment}
- Untuk k = 4\\
\end{eulercomment}
\begin{eulerformula}
\[
P(X = 4)= \binom{10}{4} \cdot (0.6)^{4} \cdot (0.4)^{6} \approx 0.11147
\]
\end{eulerformula}
\begin{eulercomment}
Maka,\\
\end{eulercomment}
\begin{eulerformula}
\[
P(X \leq 4)= P(X=0)+P(X=1)+P(X=2)+P(X=3)+P(X=4)
\]
\end{eulerformula}
\begin{eulerformula}
\[
P(X \leq 4)= 0.00010 + 0.00157 + 0.01061+ 0.04246+ 0.11147
\]
\end{eulerformula}
\begin{eulerformula}
\[
P(X \leq 4)\approx 0.1662
\]
\end{eulerformula}
\begin{eulercomment}
Fungsi Beta terbalik digunakan untuk menghitung interval kepercayaan
Clopper-Pearson untuk parameter p. Tingkat defaultnya adalah alfa.

Arti dari interval ini adalah jika p berada di luar interval, hasil
pengamatan 410 dalam 1000 jarang terjadi.
\end{eulercomment}
\begin{eulerprompt}
>clopperpearson(410,1000)
\end{eulerprompt}
\begin{euleroutput}
  [0.37932,  0.441212]
\end{euleroutput}
\begin{eulercomment}
Perintah berikut adalah cara langsung untuk mendapatkan hasil di atas.
Namun untuk n yang besar, penjumlahan langsungnya tidak akurat dan
lambat.
\end{eulercomment}
\begin{eulerprompt}
>p=0.4; i=0:410; n=1000; sum(bin(n,i)*p^i*(1-p)^(n-i))
\end{eulerprompt}
\begin{euleroutput}
  0.751401349655
\end{euleroutput}
\begin{eulercomment}
Omong-omong, invbinsum() menghitung kebalikan dari binomialsum().
\end{eulercomment}
\begin{eulerprompt}
>invbindis(0.75,1000,0.4)
\end{eulerprompt}
\begin{euleroutput}
  409.932733047
\end{euleroutput}
\begin{eulercomment}
Di Bridge, kami mengasumsikan 5 kartu beredar (dari 52) di dua tangan
(26 kartu). Mari kita hitung probabilitas distribusi yang lebih buruk
dari 3:2 (misalnya 0:5, 1:4, 4:1, atau 5:0).
\end{eulercomment}
\begin{eulerprompt}
>2*hypergeomsum(1,5,13,26)
\end{eulerprompt}
\begin{euleroutput}
  0.321739130435
\end{euleroutput}
\begin{eulercomment}
Ada juga simulasi distribusi multinomial.
\end{eulercomment}
\begin{eulerprompt}
>randmultinomial(10,1000,[0.4,0.1,0.5])
\end{eulerprompt}
\begin{euleroutput}
            407           105           488 
            397            95           508 
            397           108           495 
            378            96           526 
            403            97           500 
            410            90           500 
            389           115           496 
            385           109           506 
            373            90           537 
            396           103           501 
\end{euleroutput}
\eulerheading{Merencanakan Data/ Plot Data}
\begin{eulercomment}
Untuk memetakan data, kami mencoba hasil pemilu Jerman sejak tahun
1990, diukur dalam jumlah kursi.
\end{eulercomment}
\begin{eulerprompt}
>BW := [ ...
>1990,662,319,239,79,8,17; ...
>1994,672,294,252,47,49,30; ...
>1998,669,245,298,43,47,36; ...
>2002,603,248,251,47,55,2; ...
>2005,614,226,222,61,51,54; ...
>2009,622,239,146,93,68,76; ...
>2013,631,311,193,0,63,64];
\end{eulerprompt}
\begin{eulercomment}
Untuk beberapa bagian, kami menggunakan rangkaian nama.
\end{eulercomment}
\begin{eulerprompt}
>P:=["CDU/CSU","SPD","FDP","Gr","Li"];
\end{eulerprompt}
\begin{eulercomment}
Mari kita cetak persentasenya dengan baik.

Pertama kita mengekstrak kolom yang diperlukan. Kolom 3 sampai 7
adalah kursi masing-masing partai, dan kolom 2 adalah jumlah kursi
seluruhnya. Kolom 1 adalah tahun pemilihan.
\end{eulercomment}
\begin{eulerprompt}
>BT:=BW[,3:7]; BT:=BT/sum(BT); YT:=BW[,1]';
\end{eulerprompt}
\begin{eulercomment}
Kemudian statistiknya kita cetak dalam bentuk tabel. Kami menggunakan
nama sebagai header kolom, dan tahun sebagai header untuk baris. Lebar
default untuk kolom adalah wc=10, tetapi kami lebih memilih keluaran
yang lebih padat. Kolom akan diperluas untuk label kolom, jika perlu.
\end{eulercomment}
\begin{eulerprompt}
>writetable(BT*100,wc=6,dc=0,>fixed,labc=P,labr=YT)
\end{eulerprompt}
\begin{euleroutput}
         CDU/CSU   SPD   FDP    Gr    Li
    1990      48    36    12     1     3
    1994      44    38     7     7     4
    1998      37    45     6     7     5
    2002      41    42     8     9     0
    2005      37    36    10     8     9
    2009      38    23    15    11    12
    2013      49    31     0    10    10
\end{euleroutput}
\begin{eulercomment}
Perkalian matriks berikut ini menjumlahkan persentase dua partai besar
yang menunjukkan bahwa partai-partai kecil berhasil memperoleh suara
di parlemen hingga tahun 2009.
\end{eulercomment}
\begin{eulerprompt}
>BT1:=(BT.[1;1;0;0;0])'*100
\end{eulerprompt}
\begin{euleroutput}
  [84.29,  81.25,  81.1659,  82.7529,  72.9642,  61.8971,  79.8732]
\end{euleroutput}
\begin{eulercomment}
Ada juga plot statistik sederhana. Kami menggunakannya untuk
menampilkan garis dan titik secara bersamaan. Alternatifnya adalah
memanggil plot2d dua kali dengan \textgreater{}add.
\end{eulercomment}
\begin{eulerprompt}
>statplot(YT,BT1,"b"):
\end{eulerprompt}
\eulerimg{15}{images/Tugas Individu Pekan13-14_Muhammad Lutfi Ramadhan_23030630021-035.png}
\begin{eulercomment}
Tentukan beberapa warna untuk setiap pesta.
\end{eulercomment}
\begin{eulerprompt}
>CP:=[rgb(0.5,0.5,0.5),red,yellow,green,rgb(0.8,0,0)];
\end{eulerprompt}
\begin{eulercomment}
Sekarang kita bisa memplot hasil pemilu 2009 dan perubahannya menjadi
satu plot dengan menggunakan gambar. Kita dapat menambahkan vektor
kolom ke setiap plot.
\end{eulercomment}
\begin{eulerprompt}
>figure(2,1);  ...
>figure(1); columnsplot(BW[6,3:7],P,color=CP); ...
>figure(2); columnsplot(BW[6,3:7]-BW[5,3:7],P,color=CP);  ...
>figure(0):
\end{eulerprompt}
\eulerimg{15}{images/Tugas Individu Pekan13-14_Muhammad Lutfi Ramadhan_23030630021-036.png}
\begin{eulercomment}
Plot data menggabungkan deretan data statistik dalam satu plot.
\end{eulercomment}
\begin{eulerprompt}
>J:=BW[,1]'; DP:=BW[,3:7]'; ...
>dataplot(YT,BT',color=CP);  ...
>labelbox(P,colors=CP,styles="[]",>points,w=0.2,x=0.3,y=0.4):
\end{eulerprompt}
\eulerimg{15}{images/Tugas Individu Pekan13-14_Muhammad Lutfi Ramadhan_23030630021-037.png}
\begin{eulercomment}
Plot kolom 3D memperlihatkan baris data statistik dalam bentuk kolom.
Kami memberikan label untuk baris dan kolom. sudut adalah sudut
pandang.
\end{eulercomment}
\begin{eulerprompt}
>columnsplot3d(BT,scols=P,srows=YT, ...
>  angle=30°,ccols=CP):
\end{eulerprompt}
\eulerimg{15}{images/Tugas Individu Pekan13-14_Muhammad Lutfi Ramadhan_23030630021-038.png}
\begin{eulercomment}
Representasi lainnya adalah plot mosaik. Perhatikan bahwa kolom plot
mewakili kolom matriks di sini. Karena panjang label CDU/CSU, kami
mengambil jendela yang lebih kecil dari biasanya.
\end{eulercomment}
\begin{eulerprompt}
>shrinkwindow(>smaller);  ...
>mosaicplot(BT',srows=YT,scols=P,color=CP,style="#"); ...
>shrinkwindow():
\end{eulerprompt}
\eulerimg{15}{images/Tugas Individu Pekan13-14_Muhammad Lutfi Ramadhan_23030630021-039.png}
\begin{eulercomment}
Kita juga bisa membuat diagram lingkaran. Karena hitam dan kuning
membentuk koalisi, kami menyusun ulang elemen-elemennya.
\end{eulercomment}
\begin{eulerprompt}
>i=[1,3,5,4,2]; piechart(BW[6,3:7][i],color=CP[i],lab=P[i]):
\end{eulerprompt}
\eulerimg{15}{images/Tugas Individu Pekan13-14_Muhammad Lutfi Ramadhan_23030630021-040.png}
\begin{eulercomment}
Ini adalah jenis plot lainnya.
\end{eulercomment}
\begin{eulerprompt}
>starplot(normal(1,10)+4,lab=1:10,>rays):
\end{eulerprompt}
\eulerimg{15}{images/Tugas Individu Pekan13-14_Muhammad Lutfi Ramadhan_23030630021-041.png}
\begin{eulercomment}
Beberapa plot di plot2d bagus untuk statika. Berikut adalah plot
impuls dari data acak, terdistribusi secara seragam di [0,1].
\end{eulercomment}
\begin{eulerprompt}
>plot2d(makeimpulse(1:10,random(1,10)),>bar):
\end{eulerprompt}
\eulerimg{15}{images/Tugas Individu Pekan13-14_Muhammad Lutfi Ramadhan_23030630021-042.png}
\begin{eulercomment}
Namun untuk data yang terdistribusi secara eksponensial, kita mungkin
memerlukan plot logaritmik.
\end{eulercomment}
\begin{eulerprompt}
>logimpulseplot(1:10,-log(random(1,10))*10):
\end{eulerprompt}
\eulerimg{15}{images/Tugas Individu Pekan13-14_Muhammad Lutfi Ramadhan_23030630021-043.png}
\begin{eulercomment}
Fungsi Columnplot() lebih mudah digunakan, karena hanya memerlukan
vektor nilai. Selain itu, ia dapat mengatur labelnya ke apa pun yang
kita inginkan, kami telah mendemonstrasikannya di tutorial ini.

Ini adalah aplikasi lain, di mana kita menghitung karakter dalam
sebuah kalimat dan membuat statistik.
\end{eulercomment}
\begin{eulerprompt}
>v=strtochar("the quick brown fox jumps over the lazy dog"); ...
>w=ascii("a"):ascii("z"); x=getmultiplicities(w,v); ...
>cw=[]; for k=w; cw=cw|char(k); end; ...
>columnsplot(x,lab=cw,width=0.05):
\end{eulerprompt}
\eulerimg{15}{images/Tugas Individu Pekan13-14_Muhammad Lutfi Ramadhan_23030630021-044.png}
\begin{eulercomment}
Dimungkinkan juga untuk mengatur sumbu secara manual.
\end{eulercomment}
\begin{eulerprompt}
>n=10; p=0.4; i=0:n; x=bin(n,i)*p^i*(1-p)^(n-i); ...
>columnsplot(x,lab=i,width=0.05,<frame,<grid); ...
>yaxis(0,0:0.1:1,style="->",>left); xaxis(0,style="."); ...
>label("p",0,0.25), label("i",11,0); ...
>textbox(["Binomial distribution","with p=0.4"]):
\end{eulerprompt}
\eulerimg{15}{images/Tugas Individu Pekan13-14_Muhammad Lutfi Ramadhan_23030630021-045.png}
\begin{eulercomment}
Berikut ini cara memplot frekuensi bilangan dalam suatu vektor.

Kami membuat vektor bilangan acak bilangan bulat 1 hingga 6.
\end{eulercomment}
\begin{eulerprompt}
>v:=intrandom(1,10,10)
\end{eulerprompt}
\begin{euleroutput}
  [3,  2,  6,  10,  4,  1,  5,  3,  6,  7]
\end{euleroutput}
\begin{eulercomment}
Kemudian ekstrak nomor unik di v.
\end{eulercomment}
\begin{eulerprompt}
>vu:=unique(v)
\end{eulerprompt}
\begin{euleroutput}
  [1,  2,  3,  4,  5,  6,  7,  10]
\end{euleroutput}
\begin{eulercomment}
Dan plot frekuensi dalam plot kolom.
\end{eulercomment}
\begin{eulerprompt}
>columnsplot(getmultiplicities(vu,v),lab=vu,style="/"):
\end{eulerprompt}
\begin{eulercomment}
Kami ingin mendemonstrasikan fungsi distribusi nilai empiris.
\end{eulercomment}
\begin{eulerprompt}
>x=normal(1,20);
\end{eulerprompt}
\begin{eulercomment}
Fungsi empdist(x,vs) memerlukan array nilai yang diurutkan. Jadi kita
harus mengurutkan x sebelum kita dapat menggunakannya.
\end{eulercomment}
\begin{eulerprompt}
>xs=sort(x);
\end{eulerprompt}
\begin{eulercomment}
Kemudian kita plot distribusi empiris dan beberapa batang kepadatan ke
dalam satu plot. Alih-alih plot batang untuk distribusi kali ini kami
menggunakan plot gigi gergaji.
\end{eulercomment}
\begin{eulerprompt}
>figure(2,1); ...
>figure(1); plot2d("empdist",-4,4;xs); ...
>figure(2); plot2d(histo(x,v=-4:0.2:4,<bar));  ...
>figure(0):
\end{eulerprompt}
\eulerimg{15}{images/Tugas Individu Pekan13-14_Muhammad Lutfi Ramadhan_23030630021-046.png}
\begin{eulercomment}
Plot sebar mudah dilakukan di Euler dengan plot titik biasa. Grafik
berikut menunjukkan bahwa X dan X+Y jelas berkorelasi positif.
\end{eulercomment}
\begin{eulerprompt}
>x=normal(1,100); plot2d(x,x+rotright(x),>points,style=".."):
\end{eulerprompt}
\eulerimg{15}{images/Tugas Individu Pekan13-14_Muhammad Lutfi Ramadhan_23030630021-047.png}
\begin{eulercomment}
Seringkali kita ingin membandingkan dua sampel dengan distribusi yang
berbeda. Hal ini dapat dilakukan dengan plot kuantil-kuantil.

Untuk pengujiannya, kami mencoba distribusi student-t dan distribusi
eksponensial.
\end{eulercomment}
\begin{eulerprompt}
>x=randt(1,1000,5); y=randnormal(1,1000,mean(x),dev(x)); ...
>plot2d("x",r=6,style="--",yl="normal",xl="student-t",>vertical); ...
>plot2d(sort(x),sort(y),>points,color=red,style="x",>add):
\end{eulerprompt}
\eulerimg{15}{images/Tugas Individu Pekan13-14_Muhammad Lutfi Ramadhan_23030630021-048.png}
\begin{eulercomment}
Plot tersebut dengan jelas menunjukkan bahwa nilai terdistribusi
normal cenderung lebih kecil di ujung ekstrim.

Jika kita mempunyai dua distribusi yang ukurannya berbeda, kita dapat
memperluas distribusi yang lebih kecil atau mengecilkan distribusi
yang lebih besar. Fungsi berikut ini baik untuk keduanya. Dibutuhkan
nilai median dengan persentase antara 0 dan 1.
\end{eulercomment}
\begin{eulerprompt}
>function medianexpand (x,n) := median(x,p=linspace(0,1,n-1));
\end{eulerprompt}
\begin{eulercomment}
Mari kita bandingkan dua distribusi yang sama.
\end{eulercomment}
\begin{eulerprompt}
>x=random(1000); y=random(400); ...
>plot2d("x",0,1,style="--"); ...
>plot2d(sort(medianexpand(x,400)),sort(y),>points,color=red,style="x",>add):
\end{eulerprompt}
\eulerimg{15}{images/Tugas Individu Pekan13-14_Muhammad Lutfi Ramadhan_23030630021-049.png}
\eulerheading{Regresi dan Korelasi}
\begin{eulercomment}
Regresi linier dapat dilakukan dengan fungsi polyfit() atau berbagai
fungsi fit.

Sebagai permulaan kita menemukan garis regresi untuk data univariat
dengan polyfit(x,y,1).
\end{eulercomment}
\begin{eulerprompt}
>x=1:10; y=[2,3,1,5,6,3,7,8,9,8]; writetable(x'|y',labc=["x","y"])
\end{eulerprompt}
\begin{euleroutput}
           x         y
           1         2
           2         3
           3         1
           4         5
           5         6
           6         3
           7         7
           8         8
           9         9
          10         8
\end{euleroutput}
\begin{eulercomment}
Kami ingin membandingkan kecocokan yang tidak berbobot dan berbobot.
Pertama koefisien kecocokan linier.
\end{eulercomment}
\begin{eulerprompt}
>p=polyfit(x,y,1)
\end{eulerprompt}
\begin{euleroutput}
  Need two real matrices with same number of rows for givensqr
  fit:
      \{x,y,c\}=givensqr(A,b);
  Try "trace errors" to inspect local variables after errors.
  polyfit:
      return fit(A,y')';
\end{euleroutput}
\begin{eulercomment}
Regresi linear dapat ditulis dalam bentuk:\\
\end{eulercomment}
\begin{eulerformula}
\[
y=mx + b
\]
\end{eulerformula}
\begin{eulercomment}
dengan\\
\end{eulercomment}
\begin{eulerformula}
\[
m = \frac{n \sum xy - (\sum x)(\sum y)}{n \sum x^2 - (\sum x)^2}
\]
\end{eulerformula}
\begin{eulerformula}
\[
b = \frac{\sum y - m (\sum x)}{n}
\]
\end{eulerformula}
\begin{eulercomment}
Kita hitung:\\
\end{eulercomment}
\begin{eulerformula}
\[
n = 10, x=[1,2,3,4,5,6,7,8,9,10], y=[2,3,1,5,6,3,7,8,9,8]
\]
\end{eulerformula}
\begin{eulerformula}
\[
\sum x = 55
\]
\end{eulerformula}
\begin{eulerformula}
\[
\sum y = 52
\]
\end{eulerformula}
\begin{eulerformula}
\[
\sum xy = 353
\]
\end{eulerformula}
\begin{eulerformula}
\[
\sum x^2 = 385
\]
\end{eulerformula}
\begin{eulercomment}
Maka:\\
\end{eulercomment}
\begin{eulerformula}
\[
m = \frac{10 (353) - (55)(52)}{10 (385) - (55)^2} = \frac {3530-2860}{3850-3025} = 0.812121
\]
\end{eulerformula}
\begin{eulerformula}
\[
b = \frac{52 - 0.812121 (55)}{10}=0.733333
\]
\end{eulerformula}
\begin{eulercomment}
Jadi, b, m = 0.733333, 0.812121


Sekarang koefisien dengan bobot yang menekankan nilai terakhir.
\end{eulercomment}
\begin{eulerprompt}
>w &= "exp(-(x-10)^2/10)"; pw=polyfit(x,y,1,w=w(x))
\end{eulerprompt}
\begin{euleroutput}
  [4.71566,  0.38319]
\end{euleroutput}
\begin{eulercomment}
Kami memasukkan semuanya ke dalam satu plot untuk titik dan garis
regresi, dan untuk bobot yang digunakan.
\end{eulercomment}
\begin{eulerprompt}
>figure(2,1);  ...
>figure(1); statplot(x,y,"b",xl="Regression"); ...
>  plot2d("evalpoly(x,p)",>add,color=blue,style="--"); ...
>  plot2d("evalpoly(x,pw)",5,10,>add,color=red,style="--"); ...
>figure(2); plot2d(w,1,10,>filled,style="/",fillcolor=red,xl=w); ...
>figure(0):
\end{eulerprompt}
\begin{euleroutput}
  Matrices must fit for plotarea!
  plot2d:
      if auto then plotarea(xx,yy); endif;
  Try "trace errors" to inspect local variables after errors.
  statplot:
      plot2d(x,y,style=lstyle,xl=xl,yl=yl,color=color,vertical=vert ...
\end{euleroutput}
\begin{eulercomment}
Contoh lain kita membaca survei siswa, usia mereka, usia orang tua
mereka dan jumlah saudara kandung dari sebuah file.

Tabel ini berisi "m" dan "f" di kolom kedua. Kami menggunakan variabel
tok2 untuk mengatur terjemahan yang tepat alih-alih membiarkan
readtable() mengumpulkan terjemahannya.
\end{eulercomment}
\begin{eulerprompt}
>\{MS,hd\}:=readtable("table1.dat",tok2:=["m","f"]);  ...
>writetable(MS,labc=hd,tok2:=["m","f"]);
\end{eulerprompt}
\begin{euleroutput}
      Person       Sex       Age    Mother    Father  Siblings
           1         m        29        58        61         1
           2         f        26        53        54         2
           3         m        24        49        55         1
           4         f        25        56        63         3
           5         f        25        49        53         0
           6         f        23        55        55         2
           7         m        23        48        54         2
           8         m        27        56        58         1
           9         m        25        57        59         1
          10         m        24        50        54         1
          11         f        26        61        65         1
          12         m        24        50        52         1
          13         m        29        54        56         1
          14         m        28        48        51         2
          15         f        23        52        52         1
          16         m        24        45        57         1
          17         f        24        59        63         0
          18         f        23        52        55         1
          19         m        24        54        61         2
          20         f        23        54        55         1
\end{euleroutput}
\begin{eulercomment}
Bagaimana usia bergantung satu sama lain? Kesan pertama muncul dari
plot sebar berpasangan.
\end{eulercomment}
\begin{eulerprompt}
>scatterplots(tablecol(MS,3:5),hd[3:5]):
\end{eulerprompt}
\begin{eulercomment}
Jelas terlihat bahwa usia ayah dan ibu saling bergantung satu sama
lain. Mari kita tentukan dan plot garis regresinya.
\end{eulercomment}
\begin{eulerprompt}
>cs:=MS[,4:5]'; ps:=polyfit(cs[1],cs[2],1)
\end{eulerprompt}
\begin{euleroutput}
  [17.3789,  0.740964]
\end{euleroutput}
\begin{eulercomment}
Ini jelas merupakan model yang salah. Garis regresinya adalah
s=17+0,74t, dengan t adalah umur ibu dan s adalah umur ayah. Perbedaan
usia mungkin sedikit bergantung pada usia, tapi tidak terlalu banyak.

Sebaliknya, kami mencurigai fungsi seperti s=a+t. Maka a adalah mean
dari s-t. Ini adalah perbedaan usia rata-rata antara ayah dan ibu.
\end{eulercomment}
\begin{eulerprompt}
>da:=mean(cs[2]-cs[1])
\end{eulerprompt}
\begin{euleroutput}
  3.65
\end{euleroutput}
\begin{eulercomment}
Mari kita plot ini menjadi satu plot sebar.
\end{eulercomment}
\begin{eulerprompt}
>plot2d(cs[1],cs[2],>points);  ...
>plot2d("evalpoly(x,ps)",color=red,style=".",>add);  ...
>plot2d("x+da",color=blue,>add):
\end{eulerprompt}
\begin{eulercomment}
Berikut adalah plot kotak dari dua zaman tersebut. Ini hanya
menunjukkan, bahwa usianya berbeda-beda.
\end{eulercomment}
\begin{eulerprompt}
>boxplot(cs,["mothers","fathers"]):
\end{eulerprompt}
\begin{eulercomment}
Menariknya, perbedaan median tidak sebesar perbedaan mean.
\end{eulercomment}
\begin{eulerprompt}
>median(cs[2])-median(cs[1])
\end{eulerprompt}
\begin{euleroutput}
  1.5
\end{euleroutput}
\begin{eulercomment}
Koefisien korelasi menunjukkan korelasi positif.
\end{eulercomment}
\begin{eulerprompt}
>Koefisien korelasi menunjukkan korelasi positif.correl(cs[1],cs[2])
\end{eulerprompt}
\begin{euleroutput}
  Variable Koefisien not found!
  Error in:
  Koefisien korelasi menunjukkan korelasi positif.correl(cs[1],c ...
            ^
\end{euleroutput}
\begin{eulercomment}
Korelasi pangkat merupakan ukuran keteraturan yang sama pada kedua
vektor. Hal ini juga cukup positif.
\end{eulercomment}
\begin{eulerprompt}
>rankcorrel(cs[1],cs[2])
\end{eulerprompt}
\begin{euleroutput}
  0.758925292358
\end{euleroutput}
\eulerheading{Membuat Fungsi baru}
\begin{eulercomment}
Tentu saja, bahasa EMT dapat digunakan untuk memprogram fungsi-fungsi
baru. Misalnya, kita mendefinisikan fungsi skewness.

\end{eulercomment}
\begin{eulerformula}
\[
\text{sk}(x) = \dfrac{\sqrt{n} \sum_i (x_i-m)^3}{\left(\sum_i (x_i-m)^2\right)^{3/2}}
\]
\end{eulerformula}
\begin{eulercomment}
m adalah rata-rata dari x.
\end{eulercomment}
\begin{eulerprompt}
>function skew (x:vector) ...
\end{eulerprompt}
\begin{eulerudf}
  m=mean(x);
  return sqrt(cols(x))*sum((x-m)^3)/(sum((x-m)^2))^(3/2);
  endfunction
\end{eulerudf}
\begin{eulercomment}
Seperti yang Anda lihat, kita dapat dengan mudah menggunakan bahasa
matriks untuk mendapatkan implementasi yang sangat singkat dan
efisien. Mari kita coba fungsi ini.
\end{eulercomment}
\begin{eulerprompt}
>data=normal(20); skew(normal(10))
\end{eulerprompt}
\begin{euleroutput}
  0.00180922922014
\end{euleroutput}
\begin{eulercomment}
Berikut adalah fungsi lainnya, yang disebut koefisien skewness
Pearson.
\end{eulercomment}
\begin{eulerprompt}
>function skew1 (x) := 3*(mean(x)-median(x))/dev(x)
>skew1(data)
\end{eulerprompt}
\begin{euleroutput}
  -0.0723383096094
\end{euleroutput}
\eulerheading{Simulasi Monte Carlo}
\begin{eulercomment}
Kita simulasikan variabel acak berdistribusi normal 1000-5 sebanyak
sejuta kali. Untuk ini, kita gunakan fungsi normal(m,n), yang
menghasilkan matriks nilai berdistribusi 0-1, atau normal(n) yang
secara default bernilai m=1.
\end{eulercomment}
\begin{eulerprompt}
>n=1000000; x=normal(n)
\end{eulerprompt}
\begin{euleroutput}
  [0.594669,  0.404839,  1.57377,  -0.594406,  0.957243,  0.989854,
  -0.750387,  -1.59055,  1.11801,  -0.945978,  1.01442,  -0.725136,
  1.16019,  -0.637816,  -1.00468,  0.451281,  -2.60576,  -0.620727,
  0.247091,  0.633153,  -0.120661,  -0.732626,  -1.02127,  -0.0902834,
  0.243185,  -1.61108,  0.0818376,  -0.632437,  -0.425978,  0.0703536,
  -0.209187,  -0.225337,  0.445292,  -0.754709,  1.53436,  -0.979981,
  -1.03102,  0.852066,  0.379605,  -1.09705,  1.20582,  -1.86178,
  0.884487,  -0.192057,  -0.07492,  0.764815,  -0.802051,  0.18518,
  1.29316,  -0.241986,  0.90877,  0.562809,  -0.0182247,  0.213672,
  0.523917,  -0.271766,  0.62491,  -1.1295,  -0.726493,  0.638357,
  -0.783011,  -1.07234,  0.884189,  -0.4191,  -0.41695,  -1.15576,
  0.274547,  -0.0633383,  -0.164788,  0.883511,  -1.97886,  -2.41251,
  1.44818,  -0.828575,  1.90299,  0.644959,  -0.362327,  -0.111006,
  -1.16964,  0.665621,  0.486508,  -0.50645,  0.51728,  -2.13836,
  0.706669,  -0.918796,  1.42452,  0.435429,  -0.137239,  -0.0160858,
  -1.04948,  -1.26106,  -0.988632,  0.0747217,  0.124635,  -2.45891,
  0.00636367,  1.36691,  -0.477445,  -0.338019,  -1.12295,  -1.57755,
  -0.289792,  -0.313483,  -0.703814,  -0.190004,  0.0156531,  -0.743473,
  -0.399538,  -0.939934,  -0.765945,  -0.473433,  -1.30978,  -1.6371,
  0.879558,  0.722848,  2.26233,  0.6636,  0.454741,  0.620934,
   ... ]
\end{euleroutput}
\begin{eulerprompt}
>n=1000000; x=normal(n)*5+1000
\end{eulerprompt}
\begin{euleroutput}
  [997.217,  1002.59,  1009.99,  998.583,  1005.33,  1000.05,  1004.29,
  1008.29,  992.032,  1013.82,  996.485,  1002.4,  999.836,  1005.51,
  999.52,  997.989,  1008.56,  1011.49,  999.683,  999.686,  988.919,
  997.435,  1008.75,  1000.05,  1005.1,  993.698,  999.77,  998.385,
  1003.1,  997.314,  999.377,  1000.46,  998.008,  995.515,  997.618,
  996.455,  1004.83,  1002.92,  1002.93,  1000.02,  1001.99,  986.759,
  1003.92,  993.189,  996.029,  989.77,  1006.03,  994.944,  998.395,
  1004.7,  997.238,  1011.16,  1001.51,  992.181,  1000.37,  994.389,
  999.321,  998.169,  1005.56,  1000.49,  993.994,  1000.62,  997.579,
  1005.94,  1008.35,  999.007,  1001.37,  998.159,  1000.34,  995.868,
  996.674,  997.832,  996.366,  1004.65,  1000.83,  1004.8,  1006.08,
  993.986,  1006.84,  999.963,  996.663,  1000.55,  1002.6,  995.723,
  996.858,  1006.43,  1000.97,  1000.79,  1002.96,  1009.19,  998.141,
  1000.08,  996.891,  991.271,  998.378,  997.582,  999.068,  999.808,
  1002.07,  990.138,  1000.07,  991.562,  991.012,  996.126,  994.941,
  999.757,  999.237,  1007.14,  1003.09,  998.811,  1002.92,  1002.77,
  997.681,  994.765,  997.174,  1002.42,  995.981,  1005.29,  1002.61,
  1000.33,  1002.12,  998.899,  994.029,  991.05,  1002.69,  1004.1,
  1003.5,  999.189,  998.607,  987.859,  1004.61,  993.364,  990.851,
  993.328,  1008.34,  1004.39,  1007.6,  996.818,  999.149,  1007.52,
   ... ]
\end{euleroutput}
\begin{eulercomment}
terdapat juga fungsi randnormal(n,m,mean,dev), yang dapat kita
gunakan. Fungsi ini mematuhi skema penamaan "rand..." untuk generator
acak.
\end{eulercomment}
\begin{eulerprompt}
>n=1000000; x=randnormal(1,n,1000,5)
\end{eulerprompt}
\begin{euleroutput}
  [993.909,  995.449,  1002.06,  1003.15,  1007.19,  999.615,  998.355,
  999.025,  1006.23,  1002,  1000.7,  1000.2,  1005.38,  991.512,
  991.345,  1001.71,  991.031,  990.978,  998.105,  993.577,  999.242,
  989.155,  1007.17,  1001.49,  1001.54,  1001.32,  1014.04,  1005.8,
  1008.42,  1003.44,  994.948,  1010.62,  1006.41,  1005.93,  998.34,
  997.584,  997.494,  1003.14,  997.31,  997.461,  989.823,  999.147,
  1005.93,  996.811,  991.033,  1009.29,  996.466,  998.813,  1002.88,
  1005.13,  1008.62,  1005.91,  992.439,  1000.37,  997.342,  1005.15,
  1002.47,  991.082,  992.811,  995.208,  992.557,  1003.33,  1004.19,
  1002.22,  994.346,  993.58,  993.792,  998.402,  1004.58,  1008.68,
  993.079,  995.845,  1002.32,  1003.82,  1002,  999.093,  1011.66,
  1003.97,  1001.16,  1000.43,  1013.04,  1002.65,  1000.9,  997.038,
  999.367,  1003.75,  996.096,  996.257,  999.541,  1003.5,  997.854,
  995.397,  994.204,  985.007,  1005.95,  1000.34,  1006.44,  994.612,
  1003.64,  1005.92,  1000.94,  1001.69,  995.274,  992.588,  998.776,
  990.343,  1004.71,  998.007,  1000.23,  998.707,  1002.59,  1003.65,
  1000.12,  1009.27,  992.548,  998.5,  1000.64,  1000.44,  995.67,
  1003.11,  997.103,  1005.87,  1005.09,  996.395,  1002.52,  1000.84,
  1006.47,  1001.31,  1007.94,  998.893,  1003.86,  1005.1,  993.376,
  1005.83,  1000.41,  1000.2,  992.949,  998.768,  1008.35,  1005.16,
   ... ]
\end{euleroutput}
\begin{eulercomment}
10 nilai pertama x adalah
\end{eulercomment}
\begin{eulerprompt}
>x[1:10]
\end{eulerprompt}
\begin{euleroutput}
  [993.909,  995.449,  1002.06,  1003.15,  1007.19,  999.615,  998.355,
  999.025,  1006.23,  1002]
\end{euleroutput}
\begin{eulercomment}
Distribusi dapat kita plot dengan flag \textgreater{}distribution dari plot2d.
\end{eulercomment}
\begin{eulerprompt}
>plot2d(x,>distribution);  ...
> plot2d("qnormal(x,1000,5)",color=red,thickness=2,>add):
\end{eulerprompt}
\eulerimg{15}{images/Tugas Individu Pekan13-14_Muhammad Lutfi Ramadhan_23030630021-061.png}
\begin{eulercomment}
kita juga dapat mengatur jumlah interval untuk distribusi menjadi 100.
Kemudian kita akan melihat seberapa dekat kecocokan distribusi yang
diamati dan distribusi yang sebenarnya. Bagaimanapun, kita telah
menghasilkan satu juta kejadian.
\end{eulercomment}
\begin{eulerprompt}
>plot2d(x,distribution=100); ...
>plot2d("qnormal(x,1000,5)",color=red,thickness=2,>add):
\end{eulerprompt}
\eulerimg{15}{images/Tugas Individu Pekan13-14_Muhammad Lutfi Ramadhan_23030630021-062.png}
\begin{eulercomment}
kita dapat menghitung nilai rata-rata simulasi dan deviasinya harus
sangat dekat dengan nilai yang diharapkan.
\end{eulercomment}
\begin{eulerprompt}
>mean(x), dev(x)
\end{eulerprompt}
\begin{euleroutput}
  1000.00174908
  4.99977952257
\end{euleroutput}
\begin{eulercomment}
rumus nilai rata rata

\end{eulercomment}
\begin{eulerformula}
\[
mean = \frac{\sum x_i}{n}
\]
\end{eulerformula}
\begin{eulerprompt}
>xm=sum(x)/n
\end{eulerprompt}
\begin{euleroutput}
  1000.00174908
\end{euleroutput}
\begin{eulercomment}
Rumus simpangan percobaannya (deviasi)\\
\end{eulercomment}
\begin{eulerformula}
\[
deviasi= \sqrt{\frac{\sum (x - xm)^2}{n-1}}
\]
\end{eulerformula}
\begin{eulerprompt}
>sqrt(sum((x-xm)^2/(n-1)))
\end{eulerprompt}
\begin{euleroutput}
  4.99977952257
\end{euleroutput}
\begin{eulercomment}
Perhatikan bahwa x-xm adalah vektor nilai yang dikoreksi, di mana xm
dikurangi dari semua elemen vektor x.

Berikut adalah 10 nilai pertama x-xm.
\end{eulercomment}
\begin{eulerprompt}
>short (x-xm)[1:10]
\end{eulerprompt}
\begin{euleroutput}
  [-6.0931,  -4.5526,  2.0616,  3.1487,  7.1846,  -0.38714,  -1.6464,
  -0.97655,  6.2287,  2.0015]
\end{euleroutput}
\begin{eulercomment}
Dengan menggunakan bahasa matriks, kita dapat dengan mudah menjawab
pertanyaan lainnya. Misalnya, kita ingin menghitung proporsi x yang
melebihi 1015.\\
Ekspresi x\textgreater{}=1015 menghasilkan vektor 1 dan 0. Menjumlahkan vektor ini
menghasilkan jumlah kali x[i]\textgreater{}=1015 terjadi.
\end{eulercomment}
\begin{eulerprompt}
>sum(x>=1015)/n
\end{eulerprompt}
\begin{euleroutput}
  0.001318
\end{euleroutput}
\begin{eulercomment}
Probabilitas yang diharapkan dari hal ini dapat dihitung dengan fungsi
normaldis(x). sehingga,

\end{eulercomment}
\begin{eulerformula}
\[
normaldis(c,m,s)=P(X \leq c)
\]
\end{eulerformula}
\begin{eulercomment}
dimana X terdistribusi secara normal m-s.
\end{eulercomment}
\begin{eulerprompt}
>1-normaldis(1015,1000,5)
\end{eulerprompt}
\begin{euleroutput}
  0.00134989803163
\end{euleroutput}
\begin{eulercomment}
cara kerja \textgreater{}distribution flag dari plot2d adalah menggunakan fungsi
histo(x), yang menghasilkan histogram frekuensi nilai dalam x. Fungsi
ini mengembalikan batas interval dan jumlah dalam interval ini. Kami
menormalkan jumlah untuk mendapatkan frekuensi.
\end{eulercomment}
\begin{eulerprompt}
>\{t,s\}=histo(x,40); plot2d(t,s/n,>bar):
\end{eulerprompt}
\eulerimg{15}{images/Tugas Individu Pekan13-14_Muhammad Lutfi Ramadhan_23030630021-066.png}
\begin{eulercomment}
Fungsi histo() juga dapat menghitung frekuensi dalam interval yang
diberikan.
\end{eulercomment}
\begin{eulerprompt}
>\{t,s\}=histo(x,v=[950,980,990,1010,1020,1050]); t, s,
\end{eulerprompt}
\begin{euleroutput}
  [950,  980,  990,  1010,  1020,  1050]
  [35,  22694,  954435,  22806,  30]
\end{euleroutput}
\begin{eulercomment}
hasil tersebut merupakan semua nilai acak yang berada antara 950 dan
1050.

menghitung total jumlah nilai dalam s, yang sama dengan total jumlah
elemen dalam x
\end{eulercomment}
\begin{eulerprompt}
>sum(s)
\end{eulerprompt}
\begin{euleroutput}
  1000000
\end{euleroutput}
\begin{eulercomment}
kita akan mensimulasikan 1000 kali lemparan 3 dadu, dan menanyakan
pembagian jumlahnya.
\end{eulercomment}
\begin{eulerprompt}
>ds:=sum(intrandom(1000,3,6))';  fs=getmultiplicities(3:18,ds)
\end{eulerprompt}
\begin{euleroutput}
  [6,  17,  32,  38,  51,  92,  115,  151,  126,  113,  113,  58,  45,
  27,  12,  4]
\end{euleroutput}
\begin{eulercomment}
kita akan plot hasil tersebut
\end{eulercomment}
\begin{eulerprompt}
>columnsplot(fs,lab=3:18):
\end{eulerprompt}
\eulerimg{15}{images/Tugas Individu Pekan13-14_Muhammad Lutfi Ramadhan_23030630021-067.png}
\begin{eulercomment}
kita akan menggunakan rekursi tingkat lanjut.\\
Fungsi berikut menghitung banyaknya cara bilangan k dapat
direpresentasikan sebagai jumlah dari n bilangan dalam rentang 1
sampai m.
\end{eulercomment}
\begin{eulerprompt}
>function map countways (k; n, m) ...
\end{eulerprompt}
\begin{eulerudf}
    if n==1 then return k>=1 && k<=m
    else
      sum=0; 
      loop 1 to m; sum=sum+countways(k-#,n-1,m); end;
      return sum;
    end;
  endfunction
\end{eulerudf}
\begin{eulercomment}
Berikut hasil pelemparan dadu sebanyak lima kali.
\end{eulercomment}
\begin{eulerprompt}
>countways(5:25,5,5)
\end{eulerprompt}
\begin{euleroutput}
  [1,  5,  15,  35,  70,  121,  185,  255,  320,  365,  381,  365,  320,
  255,  185,  121,  70,  35,  15,  5,  1]
\end{euleroutput}
\begin{eulerprompt}
>cw=countways(3:18,3,6)
\end{eulerprompt}
\begin{euleroutput}
  [1,  3,  6,  10,  15,  21,  25,  27,  27,  25,  21,  15,  10,  6,  3,
  1]
\end{euleroutput}
\begin{eulercomment}
Kita akan menambahkan nilai yang diharapkan ke plot.
\end{eulercomment}
\begin{eulerprompt}
>plot2d(cw/6^3*1000,>add); plot2d(cw/6^3*1000,>points,>add):
\end{eulerprompt}
\eulerimg{15}{images/Tugas Individu Pekan13-14_Muhammad Lutfi Ramadhan_23030630021-068.png}
\begin{eulercomment}
Untuk simulasi lain, deviasi nilai rata-rata n 0-1-variabel acak
terdistribusi normal adalah 1/sqrt(n).
\end{eulercomment}
\begin{eulerprompt}
>longformat; 1/sqrt(10)
\end{eulerprompt}
\begin{euleroutput}
  0.316227766017
\end{euleroutput}
\begin{eulercomment}
Mari kita periksa ini dengan simulasi. Kami menghasilkan 10.000 kali
10 vektor acak.
\end{eulercomment}
\begin{eulerprompt}
>M=normal(10000,10); dev(mean(M)')
\end{eulerprompt}
\begin{euleroutput}
  0.319021256442
\end{euleroutput}
\begin{eulerprompt}
>plot2d(mean(M)',>distribution):
\end{eulerprompt}
\eulerimg{15}{images/Tugas Individu Pekan13-14_Muhammad Lutfi Ramadhan_23030630021-069.png}
\begin{eulercomment}
Median dari 10 bilangan acak berdistribusi normal 0-1 mempunyai
deviasi yang lebih besar.

Karena kita dapat dengan mudah menghasilkan jalan acak, kita dapat
mensimulasikan proses Wiener. Kami mengambil 1000 langkah dari 1000
proses. Kami kemudian memplot deviasi standar dan rata-rata langkah
ke-n dari proses ini bersama dengan nilai yang diharapkan berwarna
merah.
\end{eulercomment}
\begin{eulerprompt}
>n=1000; m=1000; M=cumsum(normal(n,m)/sqrt(m)); ...
>t=(1:n)/n; figure(2,1); ...
>figure(1); plot2d(t,mean(M')'); plot2d(t,0,color=red,>add); ...
>figure(2); plot2d(t,dev(M')'); plot2d(t,sqrt(t),color=red,>add); ...
>figure(0):
\end{eulerprompt}
\eulerimg{15}{images/Tugas Individu Pekan13-14_Muhammad Lutfi Ramadhan_23030630021-070.png}
\eulerheading{uji chi-kuadrat}
\begin{eulercomment}
uji chi-kuadrat adalah alat penting dalam statistik. Di Euler, banyak
tes yang diterapkan. Semua pengujian ini mengembalikan kesalahan yang
kita terima jika kita menolak hipotesis nol.

Misalnya, kami menguji lemparan dadu untuk distribusi yang seragam.
Pada 600 kali lemparan, kami mendapatkan nilai berikut, yang kami
masukkan ke dalam uji chi-kuadrat.
\end{eulercomment}
\begin{eulerprompt}
>chitest([90,103,114,101,103,89],dup(100,6)')
\end{eulerprompt}
\begin{euleroutput}
  0.498830517952
\end{euleroutput}
\begin{eulercomment}
Ini adalah nilai p-value dari uji chi-kuadrat

Uji chi-kuadrat juga memiliki mode yang menggunakan simulasi Monte
Carlo untuk menguji statistiknya,menggunakan Parameter \textgreater{}p menafsirkan
vektor y sebagai vektor probabilitas.
\end{eulercomment}
\begin{eulerprompt}
>chitest([90,103,114,101,103,89],dup(1/6,6)',>p,>montecarlo)
\end{eulerprompt}
\begin{euleroutput}
  0.488
\end{euleroutput}
\begin{eulercomment}
Ini adalah p-value dari uji chi-kuadrat menggunakan pendekatan Monte
Carlo.Dengan simulasi Monte Carlo, kita memperoleh p-value yang mirip
dengan uji chi-kuadrat standar (0,4988 di uji pertama)

Selanjutnya kita menghasilkan 1000 lemparan dadu menggunakan generator
angka acak, dan melakukan tes yang sama.
\end{eulercomment}
\begin{eulerprompt}
>n=1000; t=random([1,n*6]); chitest(count(t*6,6),dup(n,6)')
\end{eulerprompt}
\begin{euleroutput}
  0.45357780172
\end{euleroutput}
\begin{eulercomment}
Mari kita uji nilai rata-rata 100 dengan uji-t.
\end{eulercomment}
\begin{eulerprompt}
>s=200+normal([1,100])*10; ...
>ttest(mean(s),dev(s),100,200)
\end{eulerprompt}
\begin{euleroutput}
  0.2654993686
\end{euleroutput}
\begin{eulercomment}
Fungsi ttest() memerlukan nilai mean, deviasi, jumlah data, dan nilai
mean yang akan diuji.

Sekarang mari kita periksa dua pengukuran untuk mean yang sama. Kami
menolak hipotesis bahwa keduanya mempunyai mean yang sama, jika
hasilnya \textless{}0,05.
\end{eulercomment}
\begin{eulerprompt}
>tcomparedata(normal(1,10),normal(1,10))
\end{eulerprompt}
\begin{euleroutput}
  0.306435607333
\end{euleroutput}
\begin{eulercomment}
Jika kita menambahkan bias pada satu distribusi, kita akan mendapatkan
lebih banyak penolakan. Ulangi simulasi ini beberapa kali untuk
melihat efeknya.
\end{eulercomment}
\begin{eulerprompt}
>tcomparedata(normal(1,10),normal(1,10)+2)
\end{eulerprompt}
\begin{euleroutput}
  0.000742363759602
\end{euleroutput}
\begin{eulercomment}
Menambah nilai 2 ke salah satu distribusi menyebabkan p-value menjadi
sangat kecil.

Pada contoh berikutnya, kita membuat 20 lemparan dadu acak sebanyak
100 kali dan menghitung yang ada di dalamnya. Rata-rata harus ada
20/6=3,3.
\end{eulercomment}
\begin{eulerprompt}
>R=random(100,20); R=sum(R*6<=1)'; mean(R)
\end{eulerprompt}
\begin{euleroutput}
  3.2
\end{euleroutput}
\begin{eulercomment}
Sekarang kita bandingkan jumlah satuan dengan distribusi binomial.
Pertama kita plot distribusinya.
\end{eulercomment}
\begin{eulerprompt}
>plot2d(R,distribution=max(R)+1,even=1,style="\(\backslash\)/"):
\end{eulerprompt}
\eulerimg{15}{images/Tugas Individu Pekan13-14_Muhammad Lutfi Ramadhan_23030630021-071.png}
\begin{eulercomment}
kita akan Menghitung frekuensi kemunculan setiap jumlah angka "1"
dalam 20 lemparan dadu acak yang telah dilakukan 100 kali
\end{eulercomment}
\begin{eulerprompt}
>t=count(R,21);
\end{eulerprompt}
\begin{eulercomment}
Kemudian kami menghitung nilai yang diharapkan.
\end{eulercomment}
\begin{eulerprompt}
>n=0:20; b=bin(20,n)*(1/6)^n*(5/6)^(20-n)*100;
\end{eulerprompt}
\begin{eulercomment}
Kita harus mengumpulkan beberapa angka untuk mendapatkan kategori yang
cukup besar.
\end{eulercomment}
\begin{eulerprompt}
>t1=sum(t[1:2])|t[3:7]|sum(t[8:21]); ...
>b1=sum(b[1:2])|b[3:7]|sum(b[8:21]);
\end{eulerprompt}
\begin{eulercomment}
Uji chi-square menolak hipotesis bahwa distribusi kita merupakan
distribusi binomial, jika hasilnya \textless{}0,05.
\end{eulercomment}
\begin{eulerprompt}
>chitest(t1,b1)
\end{eulerprompt}
\begin{euleroutput}
  0.585701058715
\end{euleroutput}
\begin{eulercomment}
Contoh berikut berisi hasil dua kelompok orang (misalnya laki-laki dan
perempuan) yang memilih satu dari enam partai.
\end{eulercomment}
\begin{eulerprompt}
>A=[23,37,43,52,64,74;27,39,41,49,63,76];  ...
>writetable(A,wc=6,labr=["m","f"],labc=1:6)
\end{eulerprompt}
\begin{euleroutput}
             1     2     3     4     5     6
       m    23    37    43    52    64    74
       f    27    39    41    49    63    76
\end{euleroutput}
\begin{eulercomment}
Kita akan menguji independensi suara dari jenis kelamin.
\end{eulercomment}
\begin{eulerprompt}
>tabletest(A)
\end{eulerprompt}
\begin{euleroutput}
  0.990701632326
\end{euleroutput}
\begin{eulercomment}
Berikut ini adalah tabel yang diharapkan, jika kita mengasumsikan
frekuensi pemungutan suara yang diamati.
\end{eulercomment}
\begin{eulerprompt}
>writetable(expectedtable(A),wc=6,dc=1,labr=["m","f"],labc=1:6)
\end{eulerprompt}
\begin{euleroutput}
             1     2     3     4     5     6
       m  24.9  37.9  41.9  50.3  63.3  74.7
       f  25.1  38.1  42.1  50.7  63.7  75.3
\end{euleroutput}
\begin{eulercomment}
Kita dapat menghitung koefisien kontingensi yang dikoreksi. Karena
sangat mendekati 0, kami menyimpulkan bahwa pemungutan suara tidak
bergantung pada jenis kelamin.
\end{eulercomment}
\begin{eulerprompt}
>contingency(A)
\end{eulerprompt}
\begin{euleroutput}
  0.0427225484717
\end{euleroutput}
\begin{eulercomment}
\begin{eulercomment}
\eulerheading{uji F}
\begin{eulercomment}
Selanjutnya kita menggunakan analisis varians (uji F) untuk menguji
tiga sampel data yang berdistribusi normal untuk nilai mean yang sama.
Metode tersebut disebut ANOVA (analisis varians). Di Euler, fungsi
varanalisis() digunakan.
\end{eulercomment}
\begin{eulerprompt}
>x1=[109,111,98,119,91,118,109,99,115,109,94]; mean(x1),
\end{eulerprompt}
\begin{euleroutput}
  106.545454545
\end{euleroutput}
\begin{eulerprompt}
>x2=[120,124,115,139,114,110,113,120,117]; mean(x2),
\end{eulerprompt}
\begin{euleroutput}
  119.111111111
\end{euleroutput}
\begin{eulerprompt}
>x3=[120,112,115,110,105,134,105,130,121,111]; mean(x3)
\end{eulerprompt}
\begin{euleroutput}
  116.3
\end{euleroutput}
\begin{eulerprompt}
>varanalysis(x1,x2,x3)
\end{eulerprompt}
\begin{euleroutput}
  0.0138048221371
\end{euleroutput}
\begin{eulercomment}
Dengan p-value sebesar 0.0138 (1,38\%), kita bisa menolak hipotesis
bahwa ketiga sampel memiliki mean yang sama pada tingkat signifikansi
5\% (0.05) dan bahkan pada tingkat signifikansi 1\% (0.01). Artinya,
terdapat perbedaan yang signifikan antara mean dari setidaknya satu
sampel.

Ada juga uji median, yang menolak sampel data dengan distribusi
rata-rata yang berbeda, menguji median dari sampel yang disatukan.
\end{eulercomment}
\begin{eulerprompt}
>a=[56,66,68,49,61,53,45,58,54]
\end{eulerprompt}
\begin{euleroutput}
  [56,  66,  68,  49,  61,  53,  45,  58,  54]
\end{euleroutput}
\begin{eulerprompt}
>b=[72,81,51,73,69,78,59,67,65,71,68,71]
\end{eulerprompt}
\begin{euleroutput}
  [72,  81,  51,  73,  69,  78,  59,  67,  65,  71,  68,  71]
\end{euleroutput}
\begin{eulerprompt}
>mediantest(a,b)
\end{eulerprompt}
\begin{euleroutput}
  0.0241724220052
\end{euleroutput}
\begin{eulercomment}
Tes kesetaraan lainnya adalah tes peringkat. Ini jauh lebih tajam
daripada tes median.
\end{eulercomment}
\begin{eulerprompt}
>ranktest(a,b)
\end{eulerprompt}
\begin{euleroutput}
  0.00199969612469
\end{euleroutput}
\begin{eulercomment}
Pada contoh berikut, kedua distribusi mempunyai mean yang sama.
\end{eulercomment}
\begin{eulerprompt}
>ranktest(random(1,100),random(1,50)*3-1)
\end{eulerprompt}
\begin{euleroutput}
  0.0695597439892
\end{euleroutput}
\begin{eulercomment}
ini menunjukkan bahwa perbedaan tidak cukup signifikan pada tingkat
signifikansi 5\%, sehingga hipotesis bahwa kedua distribusi memiliki
median yang sama tidak dapat ditolak.

Sekarang mari kita coba mensimulasikan dua perlakuan a dan b yang
diterapkan pada orang yang berbeda.
\end{eulercomment}
\begin{eulerprompt}
>a=[8.0,7.4,5.9,9.4,8.6,8.2,7.6,8.1,6.2,8.9];
>b=[6.8,7.1,6.8,8.3,7.9,7.2,7.4,6.8,6.8,8.1];
\end{eulerprompt}
\begin{eulercomment}
Tes signum memutuskan, apakah a lebih baik dari b.
\end{eulercomment}
\begin{eulerprompt}
>signtest(a,b)
\end{eulerprompt}
\begin{euleroutput}
  0.0546875
\end{euleroutput}
\begin{eulercomment}
Ini kesalahan yang terlalu besar untuk menolak hipotesis. Kita tidak
dapat menolak bahwa a sama baiknya dengan b,Karena p \textgreater{} 0.05.

Uji Wilcoxon lebih tajam dibandingkan uji ini, namun mengandalkan
nilai kuantitatif perbedaannya.
\end{eulercomment}
\begin{eulerprompt}
>wilcoxon(a,b)
\end{eulerprompt}
\begin{euleroutput}
  0.0296680599405
\end{euleroutput}
\begin{eulercomment}
Mari kita coba dua tes lagi menggunakan rangkaian yang dihasilkan.
\end{eulercomment}
\begin{eulerprompt}
>wilcoxon(normal(1,20),normal(1,20)-1)
\end{eulerprompt}
\begin{euleroutput}
  0.033677278493
\end{euleroutput}
\begin{eulercomment}
ini menunjukkan bahwa ada perbedaan signifikan antara kedua sampel
pada tingkat signifikansi 5\%.
\end{eulercomment}
\begin{eulerprompt}
>wilcoxon(normal(1,20),normal(1,20))
\end{eulerprompt}
\begin{euleroutput}
  0.843268467533
\end{euleroutput}
\begin{eulercomment}
hasil ini  jauh di atas 0.05, sehingga kita tidak bisa menolak
hipotesis bahwa kedua sampel berasal dari distribusi yang sama.

\begin{eulercomment}
\eulerheading{Angka Acak}
\begin{eulercomment}
Berikut ini adalah pengujian pembangkit bilangan acak. Euler
menggunakan generator yang sangat bagus, jadi kita tidak perlu
mengharapkan adanya masalah.

Pertama kita menghasilkan sepuluh juta angka acak di [0,1].
\end{eulercomment}
\begin{eulerprompt}
>n:=10000000; r:=random(1,n);
\end{eulerprompt}
\begin{eulercomment}
Selanjutnya kita hitung jarak antara dua angka yang kurang dari 0,05.
\end{eulercomment}
\begin{eulerprompt}
>a:=0.05; d:=differences(nonzeros(r<a));
\end{eulerprompt}
\begin{eulercomment}
Terakhir, kami memplot berapa kali, setiap jarak terjadi, dan
membandingkannya dengan nilai yang diharapkan.
\end{eulercomment}
\begin{eulerprompt}
>m=getmultiplicities(1:100,d); plot2d(m); ...
>  plot2d("n*(1-a)^(x-1)*a^2",color=red,>add):
\end{eulerprompt}
\eulerimg{15}{images/Tugas Individu Pekan13-14_Muhammad Lutfi Ramadhan_23030630021-072.png}
\begin{eulercomment}
Hapus datanya.
\end{eulercomment}
\begin{eulerprompt}
>remvalue n;
\end{eulerprompt}
\begin{eulercomment}
Kami ingin menghitung nilai rata-rata dan simpangan baku yang diukur.
\end{eulercomment}
\begin{eulerprompt}
>M=[1000,1004,998,997,1002,1001,998,1004,998,997]; ...
>mean(M), dev(M),
\end{eulerprompt}
\begin{euleroutput}
  999.9
  2.72641400622
\end{euleroutput}
\begin{eulercomment}
Kita dapat membuat diagram kotak dan kumis untuk data tersebut. Dalam
kasus kita, tidak ada outlier.
\end{eulercomment}
\begin{eulerprompt}
>boxplot(M):
\end{eulerprompt}
\eulerimg{15}{images/Tugas Individu Pekan13-14_Muhammad Lutfi Ramadhan_23030630021-073.png}
\begin{eulercomment}
Kami menghitung probabilitas bahwa suatu nilai lebih besar dari 1005,
dengan asumsi nilai terukur dan distribusi normal.

Semua fungsi untuk distribusi dalam Euler diakhiri dengan ...dis dan
menghitung distribusi probabilitas kumulatif (CPF).

Kami mencetak hasilnya dalam \% dengan akurasi 2 digit menggunakan
fungsi cetak.
\end{eulercomment}
\begin{eulerprompt}
>print((1-normaldis(1005,mean(M),dev(M)))*100,2,unit=" %")
\end{eulerprompt}
\begin{euleroutput}
        3.07 %
\end{euleroutput}
\begin{eulercomment}
Untuk contoh berikutnya, kami mengasumsikan jumlah pria berikut dalam
rentang ukuran tertentu.
\end{eulercomment}
\begin{eulerprompt}
>r=155.5:4:187.5; v=[22,71,136,169,139,71,32,8];
\end{eulerprompt}
\begin{eulercomment}
Berikut adalah plot distribusinya.
\end{eulercomment}
\begin{eulerprompt}
>plot2d(r,v,a=150,b=200,c=0,d=190,bar=1,style="\(\backslash\)/"):
\end{eulerprompt}
\eulerimg{15}{images/Tugas Individu Pekan13-14_Muhammad Lutfi Ramadhan_23030630021-074.png}
\begin{eulercomment}
Kita dapat memasukkan data mentah tersebut ke dalam tabel.

Tabel adalah metode untuk menyimpan data statistik. Tabel kita harus
berisi tiga kolom: Awal rentang, akhir rentang, jumlah orang dalam
rentang.

Tabel dapat dicetak dengan tajuk. Kita menggunakan vektor string untuk
mengatur tajuk.
\end{eulercomment}
\begin{eulerprompt}
>T:=r[1:8]' | r[2:9]' | v'; writetable(T,labc=["from","to","count"])
\end{eulerprompt}
\begin{euleroutput}
        from        to     count
       155.5     159.5        22
       159.5     163.5        71
       163.5     167.5       136
       167.5     171.5       169
       171.5     175.5       139
       175.5     179.5        71
       179.5     183.5        32
       183.5     187.5         8
\end{euleroutput}
\begin{eulercomment}
Jika kita memerlukan nilai rata-rata dan statistik ukuran lainnya,
kita perlu menghitung titik tengah rentang. Kita dapat menggunakan dua
kolom pertama tabel kita untuk ini.
\end{eulercomment}
\begin{eulerprompt}
>(T[,1]+T[,2])/2
\end{eulerprompt}
\begin{euleroutput}
                157.5 
                161.5 
                165.5 
                169.5 
                173.5 
                177.5 
                181.5 
                185.5 
\end{euleroutput}
\begin{eulercomment}
Namun lebih mudah untuk melipat rentang dengan vektor [1/2,1/2].
\end{eulercomment}
\begin{eulerprompt}
>l=fold(r,[0.5,0.5])
\end{eulerprompt}
\begin{euleroutput}
  [157.5,  161.5,  165.5,  169.5,  173.5,  177.5,  181.5,  185.5]
\end{euleroutput}
\begin{eulercomment}
Sekarang kita dapat menghitung rata-rata dan deviasi sampel dengan
frekuensi yang diberikan.
\end{eulercomment}
\begin{eulerprompt}
>\{m,d\}=meandev(l,v); m, d,
\end{eulerprompt}
\begin{euleroutput}
  169.901234568
  5.98912964449
\end{euleroutput}
\begin{eulercomment}
Mari kita tambahkan distribusi normal nilai-nilai tersebut ke plot.
\end{eulercomment}
\begin{eulerprompt}
>plot2d("qnormal(x,m,d)*sum(v)*4", ...
>xmin=min(r),xmax=max(r),thickness=3,add=1):
\end{eulerprompt}
\eulerimg{15}{images/Tugas Individu Pekan13-14_Muhammad Lutfi Ramadhan_23030630021-075.png}
\begin{eulercomment}
\begin{eulercomment}
\eulerheading{Pengantar untuk Pengguna Proyek R}
\begin{eulercomment}
Jelasnya, EMT tidak bersaing dengan R sebagai paket statistik. Namun,
ada banyak prosedur dan fungsi statistik yang tersedia di EMT juga.
Jadi EMT dapat memenuhi kebutuhan dasar. Bagaimanapun, EMT hadir
dengan paket numerik dan sistem aljabar komputer.

Notebook ini cocok untuk Anda yang sudah familiar dengan R, namun
perlu mengetahui perbedaan sintaksis EMT dan R. Kami mencoba
memberikan gambaran umum tentang hal-hal yang sudah jelas dan kurang
jelas yang perlu Anda ketahui.

Selain itu, kami mencari cara untuk bertukar data antara kedua sistem.
\end{eulercomment}
\eulerheading{Sintaks Dasar}
\begin{eulercomment}
Hal pertama yang Anda pelajari di R adalah membuat vektor. Dalam EMT,
perbedaan utamanya adalah operator : dapat mengambil ukuran langkah.
Selain itu, ia mempunyai daya ikat yang rendah.
\end{eulercomment}
\begin{eulerprompt}
>n:=10; 0:n/20:n-1
\end{eulerprompt}
\begin{euleroutput}
  [0,  0.5,  1,  1.5,  2,  2.5,  3,  3.5,  4,  4.5,  5,  5.5,  6,  6.5,
  7,  7.5,  8,  8.5,  9]
\end{euleroutput}
\begin{eulerprompt}
>x:=[10.4, 5.6, 3.1, 6.4, 21.7]; [x,0,x]
\end{eulerprompt}
\begin{euleroutput}
  [10.4,  5.6,  3.1,  6.4,  21.7,  0,  10.4,  5.6,  3.1,  6.4,  21.7]
\end{euleroutput}
\begin{eulercomment}
Operator titik dua dengan ukuran langkah EMT digantikan oleh fungsi
seq() di R. Kita dapat menulis fungsi ini di EMT.
\end{eulercomment}
\begin{eulerprompt}
>function seq(a,b,c) := a:b:c; ...
>seq(0,-0.1,-1)
\end{eulerprompt}
\begin{euleroutput}
  [0,  -0.1,  -0.2,  -0.3,  -0.4,  -0.5,  -0.6,  -0.7,  -0.8,  -0.9,  -1]
\end{euleroutput}
\begin{eulerprompt}
>function seq(a,b,c) := a:b:c; ...
>seq(0,-0.5,-5)
\end{eulerprompt}
\begin{euleroutput}
  [0,  -0.5,  -1,  -1.5,  -2,  -2.5,  -3,  -3.5,  -4,  -4.5,  -5]
\end{euleroutput}
\begin{eulerprompt}
>function rep(x:vector,n:index) := flatten(dup(x,n)); ...
>rep(x,2)
\end{eulerprompt}
\begin{euleroutput}
  [10.4,  5.6,  3.1,  6.4,  21.7,  10.4,  5.6,  3.1,  6.4,  21.7]
\end{euleroutput}
\begin{eulercomment}
Fungsi rep() dari R tidak ada di EMT. Untuk masukan vektor dapat
dituliskan sebagai berikut.

Perhatikan bahwa "=" atau ":=" digunakan untuk tugas. Operator "-\textgreater{}"
digunakan untuk satuan dalam EMT.
\end{eulercomment}
\begin{eulerprompt}
>125km -> " miles"
\end{eulerprompt}
\begin{euleroutput}
  77.6713990297 miles
\end{euleroutput}
\begin{eulercomment}
Operator "\textless{}-" untuk penugasan memang bukan ide yang baik untuk R.

tetapi di EMT operator "\textless{}-" itu bukan penugasan melainkan perbandingan\\
Berikut ini akan membandingkan a dan -4 di EMT.
\end{eulercomment}
\begin{eulerprompt}
>a:=2; a<-4
\end{eulerprompt}
\begin{euleroutput}
  0
\end{euleroutput}
\begin{eulercomment}
EMT dan R memiliki vektor bertipe boolean. Namun dalam EMT, angka 0
dan 1 digunakan untuk mewakili salah dan benar. Di R, nilai benar dan
salah tetap bisa digunakan dalam aritmatika biasa seperti di EMT.
\end{eulercomment}
\begin{eulerprompt}
>x<5, %*x
\end{eulerprompt}
\begin{euleroutput}
  [0,  0,  1,  0,  0]
  [0,  0,  3.1,  0,  0]
\end{euleroutput}
\begin{eulercomment}
EMT memunculkan kesalahan atau menghasilkan NAN tergantung pada tanda
"kesalahan".
\end{eulercomment}
\begin{eulerprompt}
>errors off; 0/0, isNAN(sqrt(-1)), errors on;
\end{eulerprompt}
\begin{euleroutput}
  NAN
  1
\end{euleroutput}
\begin{eulercomment}
Stringnya sama di R dan EMT. Keduanya berada di lokal saat ini, bukan
di Unicode.

Di R ada paket untuk Unicode. Di EMT, string dapat berupa string
Unicode. String unicode dapat diterjemahkan ke pengkodean lokal dan
sebaliknya. Selain itu, u"..." dapat berisi entitas HTML.
\end{eulercomment}
\begin{eulerprompt}
>u"&#169; Ren&eacut; Grothmann"
\end{eulerprompt}
\begin{euleroutput}
  © René Grothmann
\end{euleroutput}
\begin{eulercomment}
karakter khusus (hak cipta © dan karakter aksen é),

\end{eulercomment}
\begin{eulerprompt}
>chartoutf([480])
\end{eulerprompt}
\begin{euleroutput}
  Ǡ
\end{euleroutput}
\begin{eulercomment}
Berikut ini mungkin tidak ditampilkan dengan benar pada sistem sebagai
A dengan titik dan garis di atasnya. Itu tergantung pada font yang
Anda gunakan.

Penggabungan string dilakukan dengan "+" atau "\textbar{}". Penggabungan ini
akan menghasilkan string tunggal, dan angka yang digabungkan akan
dikonversi otomatis ke format string. Ini dapat mencakup angka, yang
akan dicetak dalam format saat ini.
\end{eulercomment}
\begin{eulerprompt}
>"pi = "+pi
\end{eulerprompt}
\begin{euleroutput}
  pi = 3.14159265359
\end{euleroutput}
\eulerheading{Pengindeksan}
\begin{eulercomment}
Seringkali, ini akan berfungsi seperti di R.

Namun EMT akan menafsirkan indeks negatif dari belakang vektor,
sementara R menafsirkan x[n] sebagai x tanpa elemen ke-n.
\end{eulercomment}
\begin{eulerprompt}
>x, x[1:3], x[-2]
\end{eulerprompt}
\begin{euleroutput}
  [10.4,  5.6,  3.1,  6.4,  21.7]
  [10.4,  5.6,  3.1]
  6.4
\end{euleroutput}
\begin{eulerprompt}
>x, x[1:5], x[-3]
\end{eulerprompt}
\begin{euleroutput}
  [10.4,  5.6,  3.1,  6.4,  21.7]
  [10.4,  5.6,  3.1,  6.4,  21.7]
  3.1
\end{euleroutput}
\begin{eulercomment}
Untuk meniru perilaku R di EMT, kita dapat menggunakan fungsi
drop(x,n)
\end{eulercomment}
\begin{eulerprompt}
>drop(x,2)
\end{eulerprompt}
\begin{euleroutput}
  [10.4,  3.1,  6.4,  21.7]
\end{euleroutput}
\begin{eulercomment}
Vektor logika tidak diperlakukan berbeda sebagai indeks di EMT,
berbeda dengan R. Anda perlu mengekstrak elemen bukan nol terlebih
dahulu di EMT.
\end{eulercomment}
\begin{eulerprompt}
>x, x>5, x[nonzeros(x>5)]
\end{eulerprompt}
\begin{euleroutput}
  [10.4,  5.6,  3.1,  6.4,  21.7]
  [1,  1,  0,  1,  1]
  [10.4,  5.6,  6.4,  21.7]
\end{euleroutput}
\begin{eulercomment}
Sama seperti di R, vektor indeks dapat berisi pengulangan.
\end{eulercomment}
\begin{eulerprompt}
>x[[1,2,2,1]]
\end{eulerprompt}
\begin{euleroutput}
  [10.4,  5.6,  5.6,  10.4]
\end{euleroutput}
\eulerheading{Tipe Data}
\begin{eulercomment}
EMT memiliki lebih banyak tipe data tetap daripada R. Jelasnya, di R
terdapat vektor yang berkembang. Anda dapat mengatur vektor numerik
kosong v dan memberikan nilai ke elemen v[17]. Hal ini tidak mungkin
dilakukan di EMT.

Berikut ini agak tidak efisien.
\end{eulercomment}
\begin{eulerprompt}
>v=[]; for i=1 to 10000; v=v|i; end;
\end{eulerprompt}
\begin{eulercomment}
kenapa cara ini kurang efisien? karna setiap elemen baru di tambahkan
EMT harus menyalin selurus isi v kembali ke variabel v.


Semakin efisien vektor telah ditentukan sebelumnya.
\end{eulercomment}
\begin{eulerprompt}
>v=zeros(10000); for i=1 to 10000; v[i]=i; end;
\end{eulerprompt}
\begin{eulercomment}
Untuk mengubah tipe data di EMT, Anda dapat menggunakan fungsi seperti
kompleks().
\end{eulercomment}
\begin{eulerprompt}
>complex(1:4)
\end{eulerprompt}
\begin{euleroutput}
  [ 1+0i ,  2+0i ,  3+0i ,  4+0i  ]
\end{euleroutput}
\begin{eulercomment}
Konversi ke string hanya dimungkinkan untuk tipe data dasar. Format
saat ini digunakan untuk penggabungan string sederhana. Tapi ada
fungsi seperti print() atau frac().

Untuk vektor, Anda dapat dengan mudah menulis fungsi Anda sendiri.
\end{eulercomment}
\begin{eulerprompt}
>function tostr (v) ...
\end{eulerprompt}
\begin{eulerudf}
  s="[";
  loop 1 to length(v);
     s=s+print(v[#],2,0);
     if #<length(v) then s=s+","; endif;
  end;
  return s+"]";
  endfunction
\end{eulerudf}
\begin{eulercomment}
- Variabel s diinisialisasi sebagai string "[ " untuk menyimpan hasil
akhir. Awalnya, tanda kurung buka [ ditambahkan ke variabel s sebagai
pembuka.\\
- loop 1 to length(v); menjalankan perulangan dari elemen pertama
hingga elemen terakhir dalam v. Fungsi length(v) mengembalikan panjang
atau jumlah elemen dalam vektor v.\\
- print(v[#], 2, 0); adalah fungsi format yang mengonversi elemen
vektor v pada posisi saat ini (v[#]) menjadi string.\\
parameter 2 menunjukkan bahwa dua digit setelah titik desimal akan
ditampilkan, sementara 0 memastikan bahwa angka ditampilkan tanpa
tambahan simbol atau format lainnya.\\
- Bagian if #\textless{}length(v) memeriksa apakah elemen saat ini bukan elemen
terakhir. Jika benar, maka koma , akan ditambahkan ke variabel s untuk
memisahkan elemen.\\
- Setelah loop selesai, tanda kurung tutup ] ditambahkan ke string s,
dan string ini kemudian dikembalikan sebagai output.
\end{eulercomment}
\begin{eulerprompt}
>tostr(linspace(0,1,10));
\end{eulerprompt}
\begin{eulercomment}
Untuk komunikasi dengan Maxima, terdapat fungsi convertmxm(), yang
juga dapat digunakan untuk memformat vektor untuk keluaran.
\end{eulercomment}
\begin{eulerprompt}
>convertmxm(1:10);
\end{eulerprompt}
\begin{eulercomment}
Untuk Latex perintah tex dapat digunakan untuk mendapatkan perintah
Latex.
\end{eulercomment}
\begin{eulerprompt}
>tex(&[1,2,3]);
\end{eulerprompt}
\begin{eulerformula}
\[
\left[ 1 , 2 , 3 \right]
\]
\end{eulerformula}
\begin{eulercomment}
\begin{eulercomment}
\eulerheading{Faktor dan Tabel}
\begin{eulercomment}
Dalam pengantar R ada contoh yang disebut faktor.

Berikut ini adalah daftar wilayah 30 negara bagian.
\end{eulercomment}
\begin{eulerprompt}
>austates = ["tas", "sa", "qld", "nsw", "nsw", "nt", "wa", "wa", ...
>"qld", "vic", "nsw", "vic", "qld", "qld", "sa", "tas", ...
>"sa", "nt", "wa", "vic", "qld", "nsw", "nsw", "wa", ...
>"sa", "act", "nsw", "vic", "vic", "act"];
\end{eulerprompt}
\begin{eulercomment}
Perintah diatas digunakan untuk mendefinisikan sebuah array (array
sendiri adalah sekumpulan variabel yang memiliki tipe data yang sama)
karena pada data tersebut ada beberapa nama negara bagian yang
terulang. Array ini berisi singkatan untuk negara bagian dan teritori
di Australia.

Asumsikan, kita memiliki pendapatan yang sesuai di setiap negara
bagian.
\end{eulercomment}
\begin{eulerprompt}
>incomes = [60, 49, 40, 61, 64, 60, 59, 54, 62, 69, 70, 42, 56, ...
>61, 61, 61, 58, 51, 48, 65, 49, 49, 41, 48, 52, 46, ...
>59, 46, 58, 43];
\end{eulerprompt}
\begin{eulercomment}
Sekarang mari kita coba mencari nilai mean dan median dari data
pendapatan tersebut menggunakan perintah mean(incomes) dan
median(incomes)
\end{eulercomment}
\begin{eulerprompt}
>mean(incomes)
\end{eulerprompt}
\begin{euleroutput}
  54.7333333333
\end{euleroutput}
\begin{eulerprompt}
>median(incomes)
\end{eulerprompt}
\begin{euleroutput}
  57
\end{euleroutput}
\begin{eulercomment}
Sekarang, kami ingin menghitung rata-rata pendapatan di suatu wilayah.
Menjadi program statistik, R memiliki faktor() dan tappy() untuk ini.

EMT dapat melakukan hal ini dengan menemukan indeks wilayah dalam
daftar wilayah unik.
\end{eulercomment}
\begin{eulerprompt}
>auterr=sort(unique(austates)); f=indexofsorted(auterr,austates)
\end{eulerprompt}
\begin{euleroutput}
  [6,  5,  4,  2,  2,  3,  8,  8,  4,  7,  2,  7,  4,  4,  5,  6,  5,  3,
  8,  7,  4,  2,  2,  8,  5,  1,  2,  7,  7,  1]
\end{euleroutput}
\begin{eulercomment}
Pada titik itu, kita dapat menulis fungsi perulangan kita sendiri
untuk melakukan sesuatu hanya untuk satu faktor.

Atau kita bisa meniru fungsi tapply() dengan cara berikut.
\end{eulercomment}
\begin{eulerprompt}
>function map tappl (i; f$:call, cat, x) ...
\end{eulerprompt}
\begin{eulerudf}
  u=sort(unique(cat));
  f=indexof(u,cat);
  return f$(x[nonzeros(f==indexof(u,i))]);
  endfunction
\end{eulerudf}
\begin{eulercomment}
i: Parameter pertama biasanya adalah nilai yang digunakan untuk
pencocokan atau pemetaan.\\
f\textdollar{}:call: Parameter kedua, yang kemungkinan besar adalah sebuah fungsi
yang dipanggil dalam kode tersebut. f\textdollar{} di sini merujuk pada fungsi
yang diterima sebagai input.\\
cat: Parameter ketiga adalah array atau vektor yang berisi kategori
yang akan diproses.\\
x: Parameter keempat adalah array atau vektor yang akan diproses atau
diubah berdasarkan pemetaan kategori yang dilakukan.

Ini agak tidak efisien, karena menghitung wilayah unik untuk setiap i,
tetapi berhasil.
\end{eulercomment}
\begin{eulerprompt}
>tappl(auterr,"mean",austates,incomes)
\end{eulerprompt}
\begin{euleroutput}
  [44.5,  57.3333333333,  55.5,  53.6,  55,  60.5,  56,  52.25]
\end{euleroutput}
\begin{eulercomment}
Perhatikan bahwa ini berfungsi untuk setiap vektor wilayah.
\end{eulercomment}
\begin{eulerprompt}
>tappl(["act","nsw"],"mean",austates,incomes)
\end{eulerprompt}
\begin{euleroutput}
  [44.5,  57.3333333333]
\end{euleroutput}
\begin{eulercomment}
Sekarang, paket statistik EMT mendefinisikan tabel seperti di R.
Fungsi readtable() dan writetable() dapat digunakan untuk input dan
output.

Sehingga kita bisa mencetak rata-rata pendapatan negara di daerah
secara bersahabat.
\end{eulercomment}
\begin{eulerprompt}
>writetable(tappl(auterr,"mean",austates,incomes),labc=auterr,wc=7)
\end{eulerprompt}
\begin{euleroutput}
      act    nsw     nt    qld     sa    tas    vic     wa
     44.5  57.33   55.5   53.6     55   60.5     56  52.25
\end{euleroutput}
\begin{eulercomment}
Fungsi writetable digunakan untuk menampilkan data dalam bentuk tabel
yang terstruktur dengan label kolom dan lebar kolom yang dapat
disesuaikan.\\
Dengan labc=auterr, berarti menetapkan label kolom untuk tabel
tersebut berdasarkan kategori yang ada di auterr(yang sudah diurutkan
sesuai abjad).\\
wc(width of columns)=7 berarti setiap kolom dalam tabel akan memiliki
lebar minimal 7 karakter.\\
sebagai contoh 44.5 itu memiliki 4 karakter (termasuk titik desimal).\\
karena data dalam kolom lebih pendek dari 7 karakter, kolom tersebut
diberi ruang ekstra untuk tampilan yang rapi.

Kita juga bisa mencoba meniru perilaku R sepenuhnya.

Faktor-faktor tersebut harus disimpan dengan jelas dalam kumpulan
beserta jenis dan kategorinya (negara bagian dan teritori dalam contoh
kita). Untuk EMT, kami menambahkan indeks yang telah dihitung
sebelumnya.
\end{eulercomment}
\begin{eulerprompt}
>function makef (t) ...
\end{eulerprompt}
\begin{eulerudf}
  ## Factor data
  ## Returns a collection with data t, unique data, indices.
  ## See: tapply
  u=sort(unique(t));
  return \{\{t,u,indexofsorted(u,t)\}\};
  endfunction
\end{eulerudf}
\begin{eulerprompt}
>statef=makef(austates);
\end{eulerprompt}
\begin{eulercomment}
Perintah statef = makef(austates); digunakan untuk mengolah data yang
ada dalam variabel austates, dan mengidentifikasi elemen unik yang ada
dalam data tersebut.

Sekarang elemen ketiga dari koleksi akan berisi indeks.
\end{eulercomment}
\begin{eulerprompt}
>statef[3]
\end{eulerprompt}
\begin{euleroutput}
  [6,  5,  4,  2,  2,  3,  8,  8,  4,  7,  2,  7,  4,  4,  5,  6,  5,  3,
  8,  7,  4,  2,  2,  8,  5,  1,  2,  7,  7,  1]
\end{euleroutput}
\begin{eulercomment}
statef[3] adalah elemen ketiga dari koleksi yang dikembalikan oleh
fungsi makef, yaitu indeks posisi dari elemen-elemen dalam austates
yang sudah dipetakan ke urutan dalam u (data unik yang terurut).\\
statef[3] akan mengembalikan indeks posisi dari setiap elemen dalam
austates berdasarkan urutan yang ada di u.

Sekarang kita bisa meniru tapply() dengan cara berikut. Ini akan
mengembalikan tabel sebagai kumpulan data tabel dan judul kolom.
\end{eulercomment}
\begin{eulerprompt}
>function tapply (t:vector,tf,f$:call) ...
\end{eulerprompt}
\begin{eulerudf}
  ## Makes a table of data and factors
  ## tf : output of makef()
  ## See: makef
  uf=tf[2]; f=tf[3]; x=zeros(length(uf));
  for i=1 to length(uf);
     ind=nonzeros(f==i);
     if length(ind)==0 then x[i]=NAN;
     else x[i]=f$(t[ind]);
     endif;
  end;
  return \{\{x,uf\}\};
  endfunction
\end{eulerudf}
\begin{eulercomment}
Kami tidak menambahkan banyak pengecekan tipe di sini. Satu-satunya
tindakan pencegahan menyangkut kategori (faktor) yang tidak memiliki
data. Tetapi kita harus memeriksa panjang t yang benar dan kebenaran
pengumpulan tf.

Tabel ini dapat dicetak sebagai tabel dengan writetable().
\end{eulercomment}
\begin{eulerprompt}
>writetable(tapply(incomes,statef,"mean"),wc=7)
\end{eulerprompt}
\begin{euleroutput}
      act    nsw     nt    qld     sa    tas    vic     wa
     44.5  57.33   55.5   53.6     55   60.5     56  52.25
\end{euleroutput}
\eulerheading{Array}
\begin{eulercomment}
EMT hanya memiliki dua dimensi untuk array. Tipe datanya disebut
matriks. Namun, akan mudah untuk menulis fungsi untuk dimensi yang
lebih tinggi atau perpustakaan C untuk ini.

R memiliki lebih dari dua dimensi. Di R array adalah vektor dengan
bidang dimensi.

Dalam EMT, vektor adalah matriks dengan satu baris. Itu dapat dibuat
menjadi matriks dengan redim().
\end{eulercomment}
\begin{eulerprompt}
>shortformat; X=redim(1:20,4,5)
\end{eulerprompt}
\begin{euleroutput}
          1         2         3         4         5 
          6         7         8         9        10 
         11        12        13        14        15 
         16        17        18        19        20 
\end{euleroutput}
\begin{eulercomment}
Fungsi shortformat digunakan untuk mengatur format tampilan angka agar
lebih ringkas dan mudah dibaca.\\
Perintah diatas digunakan untuk membuat matrik X dari angka 1 sampai
20 dengan ketentuan matriks dengan 4 baris dan 5 kolom.

Ekstraksi baris dan kolom, atau sub-matriks, mirip dengan R.
\end{eulercomment}
\begin{eulerprompt}
>X[,2:3]
\end{eulerprompt}
\begin{euleroutput}
          2         3 
          7         8 
         12        13 
         17        18 
\end{euleroutput}
\begin{eulercomment}
Perintah diatas digunakan untuk menampilkan matriks X kolom kedua
sampai ketiga.
\end{eulercomment}
\begin{eulerprompt}
>X[,3:5]
\end{eulerprompt}
\begin{euleroutput}
          3         4         5 
          8         9        10 
         13        14        15 
         18        19        20 
\end{euleroutput}
\begin{eulercomment}
Namun, di R dimungkinkan untuk menyetel daftar indeks vektor tertentu
ke suatu nilai. Hal yang sama mungkin terjadi di EMT hanya dengan satu
putaran.
\end{eulercomment}
\begin{eulerprompt}
>function setmatrixvalue (M, i, j, v) ...
\end{eulerprompt}
\begin{eulerudf}
  loop 1 to max(length(i),length(j),length(v))
     M[i\{#\},j\{#\}] = v\{#\};
  end;
  endfunction
\end{eulerudf}
\begin{eulercomment}
Perintah setmatrixvalue(M, i, j, v) adalah fungsi yang digunakan untuk
mengubah nilai elemen-elemen dalam matriks berdasarkan indeks
tertentu.\\
M: Matriks yang akan dimodifikasi.

i: Indeks baris atau posisi baris dalam matriks M yang ingin diubah.

j: Indeks kolom atau posisi kolom dalam matriks M yang ingin diubah.

v: Nilai yang akan dimasukkan ke dalam elemen-elemen matriks M pada
posisi yang ditentukan oleh indeks i dan j.

Kami mendemonstrasikan ini untuk menunjukkan bahwa matriks dilewatkan
dengan referensi di EMT. Jika Anda tidak ingin mengubah matriks M
asli, Anda perlu menyalinnya ke dalam fungsi.
\end{eulercomment}
\begin{eulerprompt}
>setmatrixvalue(X,1:3,3:-1:1,0); X,
\end{eulerprompt}
\begin{euleroutput}
          1         2         0         4         5 
          6         0         8         9        10 
          0        12        13        14        15 
         16        17        18        19        20 
\end{euleroutput}
\begin{eulercomment}
Perkalian luar dalam EMT hanya dapat dilakukan antar vektor. Ini
otomatis karena bahasa matriks. Satu vektor harus berupa vektor kolom
dan vektor lainnya harus berupa vektor baris.
\end{eulercomment}
\begin{eulerprompt}
>(1:5)*(1:5)'
\end{eulerprompt}
\begin{euleroutput}
          1         2         3         4         5 
          2         4         6         8        10 
          3         6         9        12        15 
          4         8        12        16        20 
          5        10        15        20        25 
\end{euleroutput}
\begin{eulercomment}
1:5: Ini adalah vektor baris yang berisi angka-angka dari 1 hingga 5\\
(1:5)': Tanda ' di sini menunjukkan transposisi dari vektor baris 1:5.
Dengan kata lain, ini mengubah vektor baris menjadi vektor kolom.

Dalam PDF pendahuluan untuk R terdapat contoh yang menghitung
distribusi ab-cd untuk a,b,c,d yang dipilih dari 0 hingga n secara
acak. Solusi dalam R adalah membentuk matriks 4 dimensi dan
menjalankan table() di atasnya.

Tentu saja, hal ini dapat dicapai dengan satu putaran. Tapi loop tidak
efektif di EMT atau R. Di EMT, kita bisa menulis loop di C dan itu
akan menjadi solusi tercepat.

Namun kita ingin meniru perilaku R. Untuk melakukannya, kita perlu
meratakan perkalian ab dan membuat matriks ab-cd.
\end{eulercomment}
\begin{eulerprompt}
>a=0:6; b=a'; p=flatten(a*b); q=flatten(p-p'); ...
>u=sort(unique(q)); f=getmultiplicities(u,q); ...
>statplot(u,f,"h"):
\end{eulerprompt}
\eulerimg{15}{images/Tugas Individu Pekan13-14_Muhammad Lutfi Ramadhan_23030630021-077.png}
\begin{eulercomment}
Selain multiplisitas eksak, EMT dapat menghitung frekuensi dalam
vektor.
\end{eulercomment}
\begin{eulerprompt}
>getfrequencies(q,-50:10:50)
\end{eulerprompt}
\begin{euleroutput}
  [0,  23,  132,  316,  602,  801,  333,  141,  53,  0]
\end{euleroutput}
\begin{eulercomment}
Perintah diatas digunakan untuk menghitung distribusi frekuensi
nilai-nilai dalam vektor q dalam rentang dari -50 hingga 50, dengan
interval 10. Fungsi ini menghitung berapa banyak nilai dalam q yang
jatuh dalam setiap interval: [-50, -40), [-40, -30), ..., [40, 50).

Cara paling mudah untuk memplotnya sebagai distribusi adalah sebagai
berikut.
\end{eulercomment}
\begin{eulerprompt}
>plot2d(q,distribution=11):
\end{eulerprompt}
\eulerimg{15}{images/Tugas Individu Pekan13-14_Muhammad Lutfi Ramadhan_23030630021-078.png}
\begin{eulercomment}
Namun dimungkinkan juga untuk menghitung terlebih dahulu penghitungan
dalam interval yang dipilih sebelumnya. Tentu saja, berikut ini
menggunakan getfrequencies() secara internal.

Karena fungsi histo() mengembalikan frekuensi, kita perlu
menskalakannya sehingga integral di bawah grafik batang adalah 1.
\end{eulercomment}
\begin{eulerprompt}
>\{x,y\}=histo(q,v=-55:10:55); y=y/sum(y)/differences(x); ...
>plot2d(x,y,>bar,style="/"):
\end{eulerprompt}
\eulerimg{15}{images/Tugas Individu Pekan13-14_Muhammad Lutfi Ramadhan_23030630021-079.png}
\begin{eulercomment}
\begin{eulercomment}
\eulerheading{Daftar}
\begin{eulercomment}
EMT memiliki dua jenis daftar. Salah satunya adalah daftar global yang
bisa berubah, dan yang lainnya adalah tipe daftar yang tidak bisa
diubah. Kami tidak peduli dengan daftar global di sini.

Tipe daftar yang tidak dapat diubah disebut koleksi di EMT. Ini
berperilaku seperti struktur di C, tetapi elemennya hanya diberi nomor
dan tidak diberi nama.

1. Membuat list dan mengakses elemen dalam list
\end{eulercomment}
\begin{eulerprompt}
>L=\{\{"Fred","Flintstone",40,[1990,1992]\}\}
\end{eulerprompt}
\begin{euleroutput}
  Fred
  Flintstone
  40
  [1990,  1992]
\end{euleroutput}
\begin{eulercomment}
Perintah diatas digunakan untuk membuat list L dengan nama depan Fred,
nama belakang Flintstone, usia 40, dan tahun 1990, 1992.\\
Namun untuk tahun tersebut tidak dapat dipastikan apa arti dari
tahun-tahun tersebut, bisa saja tahun kelahiran dan kematian, tahun
pendidikan, tahun pekerjaan, atau yang lainnya.

Saat ini unsur-unsur tersebut tidak memiliki nama, meskipun nama dapat
ditetapkan untuk tujuan khusus. Mereka diakses dengan nomor.
\end{eulercomment}
\begin{eulerprompt}
>(L[4])[2]
\end{eulerprompt}
\begin{euleroutput}
  1992
\end{euleroutput}
\begin{eulercomment}
Perintah diatas digunakan untuk menampilkan list L keempat urutan
kedua. Karena pada list L keempat berisi tahun yang dimana terdapat 2
tahun, tahun pertama adalah 1990 dan tahun kedua adalah 1992. Perintah
tersebut ingin menampilkan tahun kedua, maka outputnya adalah 1992.

2. Menggabungkan dua list
\end{eulercomment}
\begin{eulerprompt}
>A := [1,2,3]
\end{eulerprompt}
\begin{euleroutput}
  [1,  2,  3]
\end{euleroutput}
\begin{eulerprompt}
>B := [4,5,6]
\end{eulerprompt}
\begin{euleroutput}
  [4,  5,  6]
\end{euleroutput}
\begin{eulerprompt}
>C := [A, B]
\end{eulerprompt}
\begin{euleroutput}
  [1,  2,  3,  4,  5,  6]
\end{euleroutput}
\begin{eulercomment}
3. Mengubah elemen dalam list
\end{eulercomment}
\begin{eulerprompt}
>D := [7,8,9,10]
\end{eulerprompt}
\begin{euleroutput}
  [7,  8,  9,  10]
\end{euleroutput}
\begin{eulerprompt}
>D[3] := 99
\end{eulerprompt}
\begin{euleroutput}
  [7,  8,  99,  10]
\end{euleroutput}
\begin{eulercomment}
4. menghitung panjang list
\end{eulercomment}
\begin{eulerprompt}
>E := [10,20,30,40,50,60,70]
\end{eulerprompt}
\begin{euleroutput}
  [10,  20,  30,  40,  50,  60,  70]
\end{euleroutput}
\begin{eulerprompt}
>len := length(E)
\end{eulerprompt}
\begin{euleroutput}
  7
\end{euleroutput}
\begin{eulercomment}
\begin{eulercomment}
\eulerheading{File Input dan Output (Membaca dan Menulis Data)}
\begin{eulercomment}
Anda sering kali ingin mengimpor matriks data dari sumber lain ke EMT.
Tutorial ini memberi tahu Anda tentang banyak cara untuk mencapai hal
ini. Fungsi sederhananya adalah writematrix() dan readmatrix().

Mari kita tunjukkan cara membaca dan menulis vektor real ke file.
\end{eulercomment}
\begin{eulerprompt}
>a=random(1,100); mean(a), dev(a),
\end{eulerprompt}
\begin{euleroutput}
  0.50877
  0.29823
\end{euleroutput}
\begin{eulerformula}
\[
mean= \frac{1}{n} \sum_{i=1}^n x_i
\]
\end{eulerformula}
\begin{eulerformula}
\[
dev= \sqrt{\frac{1}{n-1}\sum_{i=1}^n(x_i-x)^2}
\]
\end{eulerformula}
\begin{eulercomment}
Untuk menulis data ke file, kita menggunakan fungsi writematrix().

Karena pengenalan ini kemungkinan besar ada di direktori, di mana
pengguna tidak memiliki akses tulis, kami menulis data ke direktori
home pengguna. Untuk buku catatan sendiri, hal ini tidak diperlukan,
karena file data akan ditulis ke dalam direktori yang sama.
\end{eulercomment}
\begin{eulerprompt}
>filename="test.dat";
\end{eulerprompt}
\begin{eulercomment}
Sekarang kita menulis vektor kolom a' ke file. Ini menghasilkan satu
nomor di setiap baris file.
\end{eulercomment}
\begin{eulerprompt}
>writematrix(a',filename)
\end{eulerprompt}
\begin{eulercomment}
Untuk membaca data, kita menggunakan readmatrix()
\end{eulercomment}
\begin{eulerprompt}
>a=readmatrix(filename)'
\end{eulerprompt}
\begin{euleroutput}
  [0.067788,  0.0069551,  0.79177,  0.46697,  0.88572,  0.62323,  0.3563,
  0.13821,  0.81998,  0.63064,  0.2313,  0.94722,  0.050549,  0.73706,
  0.3564,  0.45624,  0.39517,  0.024241,  0.78341,  0.66949,  0.66402,
  0.92648,  0.16134,  0.22421,  0.67846,  0.17555,  0.74348,  0.13851,
  0.62084,  0.51664,  0.47372,  0.57745,  0.054384,  0.50955,  0.73305,
  0.81468,  0.46121,  0.33495,  0.99671,  0.47853,  0.48392,  0.68367,
  0.69906,  0.54403,  0.87231,  0.11038,  0.63528,  0.44927,  0.38662,
  0.75775,  0.90199,  0.10859,  0.37499,  0.0076804,  0.69771,  0.69744,
  0.70032,  0.32777,  0.71683,  0.33858,  0.95522,  0.8847,  0.62253,
  0.78447,  0.88177,  0.38756,  0.0015708,  0.1791,  0.9956,  0.10918,
  0.20354,  0.38863,  0.69405,  0.78769,  0.26061,  0.049016,  0.042538,
  0.90893,  0.96385,  0.27297,  0.97447,  0.2176,  0.37186,  0.3532,
  0.7827,  0.077767,  0.58132,  0.11367,  0.38659,  0.60442,  0.40113,
  0.92264,  0.33496,  0.088732,  0.51624,  0.18083,  0.79184,  0.60923,
  0.9795,  0.99769]
\end{euleroutput}
\begin{eulercomment}
Dan hapus file tersebut.
\end{eulercomment}
\begin{eulerprompt}
>fileremove(filename);
>mean(a), dev(a),
\end{eulerprompt}
\begin{euleroutput}
  0.50877
  0.29823
\end{euleroutput}
\begin{eulercomment}
Fungsi writematrix() atau writetable() dapat dikonfigurasi untuk
bahasa lain.

Misalnya, jika Anda memiliki sistem Indonesia (titik desimal dengan
koma), Excel Anda memerlukan nilai dengan koma desimal yang dipisahkan
dengan titik koma dalam file csv (defaultnya adalah nilai yang
dipisahkan koma). File berikut "test.csv" akan muncul di folder saat
ini Anda.
\end{eulercomment}
\begin{eulerprompt}
>filename="test.csv"; ...
>writematrix(random(5,3),file=filename,separator=",")
\end{eulerprompt}
\begin{eulercomment}
Anda sekarang dapat membuka file ini dengan Excel bahasa Indonesia
secara langsung.
\end{eulercomment}
\begin{eulerprompt}
>fileremove(filename);
\end{eulerprompt}
\begin{eulercomment}
Terkadang kita memiliki string dengan token seperti berikut.
\end{eulercomment}
\begin{eulerprompt}
>s1:="f m m f m m m f f f m m f";  ...
>s2:="f f f m m f f";
\end{eulerprompt}
\begin{eulercomment}
Untuk melakukan tokenisasi ini, kami mendefinisikan vektor token.
\end{eulercomment}
\begin{eulerprompt}
>tok:=["f","m"]
\end{eulerprompt}
\begin{euleroutput}
  f
  m
\end{euleroutput}
\begin{eulercomment}
Kemudian kita dapat menghitung berapa kali setiap token muncul dalam
string, dan memasukkan hasilnya ke dalam tabel.
\end{eulercomment}
\begin{eulerprompt}
>M:=getmultiplicities(tok,strtokens(s1))_ ...
>  getmultiplicities(tok,strtokens(s2));
\end{eulerprompt}
\begin{eulercomment}
Tulis tabel dengan header token.
\end{eulercomment}
\begin{eulerprompt}
>writetable(M,labc=tok,labr=1:2,wc=8)
\end{eulerprompt}
\begin{euleroutput}
                 f       m
         1       6       7
         2       5       2
\end{euleroutput}
\begin{eulercomment}
Untuk statika, EMT dapat membaca dan menulis tabel.
\end{eulercomment}
\begin{eulerprompt}
>file="test.dat"; open(file,"w"); ...
>writeln("A,B,C"); writematrix(random(3,3)); ...
>close();
\end{eulerprompt}
\begin{eulercomment}
The file looks like this.
\end{eulercomment}
\begin{eulerprompt}
>printfile(file)
\end{eulerprompt}
\begin{euleroutput}
  A,B,C
  0.8910393909264115,0.7342404864777072,0.1286626471841359
  0.6367455442079208,0.4280982647968235,0.6285685540308836
  0.5384287521086838,0.8624719326897125,0.3709559205599377
  
\end{euleroutput}
\begin{eulercomment}
Fungsi readtable() dalam bentuknya yang paling sederhana dapat membaca
ini dan mengembalikan kumpulan nilai dan baris judul.
\end{eulercomment}
\begin{eulerprompt}
>L=readtable(file,>list);
\end{eulerprompt}
\begin{eulercomment}
Koleksi ini dapat dicetak dengan writetable() ke buku catatan, atau ke
file.
\end{eulercomment}
\begin{eulerprompt}
>writetable(L,wc=10,dc=5)
\end{eulerprompt}
\begin{euleroutput}
           A         B         C
     0.89104   0.73424   0.12866
     0.63675    0.4281   0.62857
     0.53843   0.86247   0.37096
\end{euleroutput}
\begin{eulercomment}
Matriks nilai adalah elemen pertama dari L. Perhatikan bahwa mean() di
EMT menghitung nilai rata-rata baris matriks.
\end{eulercomment}
\begin{eulerprompt}
>mean(L[1])
\end{eulerprompt}
\begin{euleroutput}
    0.58465 
    0.56447 
    0.59062 
\end{euleroutput}
\eulerheading{File CSV}
\begin{eulercomment}
Pertama, mari kita menulis matriks ke dalam file. Untuk outputnya,
kami membuat file di direktori kerja saat ini.
\end{eulercomment}
\begin{eulerprompt}
>file="test.csv";  ...
>M=random(3,3); writematrix(M,file);
\end{eulerprompt}
\begin{eulercomment}
Here is the content of this file.
\end{eulercomment}
\begin{eulerprompt}
>printfile(file)
\end{eulerprompt}
\begin{euleroutput}
  0.8819682672467561,0.8665461589845007,0.6002792241162314
  0.8917809421203007,0.5190138486754974,0.9963167624952852
  0.1287989709283749,0.4810705319470703,0.6487943554052712
  
\end{euleroutput}
\begin{eulercomment}
CVS ini dapat dibuka pada sistem berbahasa Inggris ke Excel dengan
klik dua kali. Jika Anda mendapatkan file seperti itu di sistem
Jerman, Anda perlu mengimpor data ke Excel dengan memperhatikan titik
desimal.

Namun titik desimal juga merupakan format default untuk EMT. Anda
dapat membaca matriks dari file dengan readmatrix().
\end{eulercomment}
\begin{eulerprompt}
>readmatrix(file)
\end{eulerprompt}
\begin{euleroutput}
    0.88197   0.86655   0.60028 
    0.89178   0.51901   0.99632 
     0.1288   0.48107   0.64879 
\end{euleroutput}
\begin{eulercomment}
Dimungkinkan untuk menulis beberapa matriks ke satu file. Perintah
open() dapat membuka file untuk ditulis dengan parameter "w".
Standarnya adalah "r" untuk membaca.
\end{eulercomment}
\begin{eulerprompt}
>open(file,"w"); writematrix(M); writematrix(M'); close();
\end{eulerprompt}
\begin{eulercomment}
Matriks dipisahkan oleh garis kosong. Untuk membaca matriks, buka file
dan panggil readmatrix() beberapa kali.
\end{eulercomment}
\begin{eulerprompt}
>open(file); A=readmatrix(); B=readmatrix(); A==B, close();
\end{eulerprompt}
\begin{euleroutput}
          1         0         0 
          0         1         0 
          0         0         1 
\end{euleroutput}
\begin{eulercomment}
Di Excel atau spreadsheet serupa, Anda dapat mengekspor matriks
sebagai CSV (nilai yang dipisahkan koma). Di Excel 2007, gunakan "save
as" dan "other format", lalu pilih "CSV". Pastikan tabel saat ini
hanya berisi data yang ingin Anda ekspor.

Ini sebuah contoh.
\end{eulercomment}
\begin{eulerprompt}
> printfile("excel-data.csv")
\end{eulerprompt}
\begin{euleroutput}
  Could not open the file
  excel-data.csv
  for reading!
  Try "trace errors" to inspect local variables after errors.
  printfile:
      open(filename,"r");
\end{euleroutput}
\begin{eulercomment}
Seperti yang Anda lihat, sistem bahasa Jerman saya menggunakan titik
koma sebagai pemisah dan koma desimal. Anda dapat mengubahnya di
pengaturan sistem atau di Excel, tetapi hal ini tidak diperlukan untuk
membaca matriks menjadi EMT.

Cara termudah untuk membaca ini ke dalam Euler adalah readmatrix().
Semua koma diganti dengan titik dengan parameter \textgreater{}koma. Untuk CSV
bahasa Inggris, hilangkan saja parameter ini.
\end{eulercomment}
\begin{eulerprompt}
>M=readmatrix("excel-data.csv",>comma)
\end{eulerprompt}
\begin{euleroutput}
  Could not open the file
  excel-data.csv
  for reading!
  Try "trace errors" to inspect local variables after errors.
  readmatrix:
      if filename<>"" then open(filename,"r"); endif;
\end{euleroutput}
\begin{eulercomment}
Let us plot this.
\end{eulercomment}
\begin{eulerprompt}
>plot2d(M'[1],M'[2:3],>points,color=[red,green]'):
\end{eulerprompt}
\eulerimg{15}{images/Tugas Individu Pekan13-14_Muhammad Lutfi Ramadhan_23030630021-082.png}
\begin{eulercomment}
Ada cara yang lebih mendasar untuk membaca data dari suatu file. Anda
dapat membuka file dan membaca angka baris demi baris. Fungsi
getvectorline() akan membaca angka dari sebaris data. Secara default,
ini mengharapkan titik desimal. Tapi bisa juga menggunakan koma
desimal, jika Anda memanggil setdecimaldot(",") sebelum Anda
menggunakan fungsi ini.

Fungsi berikut adalah contohnya. Itu akan berhenti di akhir file atau
baris kosong.
\end{eulercomment}
\begin{eulerprompt}
>function myload (file) ...
\end{eulerprompt}
\begin{eulerudf}
  open(file);
  M=[];
  repeat
     until eof();
     v=getvectorline(3);
     if length(v)>0 then M=M_v; else break; endif;
  end;
  return M;
  close(file);
  endfunction
\end{eulerudf}
\begin{eulerprompt}
>myload(file)
\end{eulerprompt}
\begin{euleroutput}
    0.88197         0   0.86655         0   0.60028 
    0.89178         0   0.51901         0   0.99632 
     0.1288         0   0.48107         0   0.64879 
\end{euleroutput}
\begin{eulercomment}
Dimungkinkan juga untuk membaca semua angka dalam file itu dengan
getvector().
\end{eulercomment}
\begin{eulerprompt}
>open(file); v=getvector(10000); close(); redim(v[1:9],3,3)
\end{eulerprompt}
\begin{euleroutput}
    0.88197         0   0.86655 
          0   0.60028   0.89178 
          0   0.51901         0 
\end{euleroutput}
\begin{eulercomment}
Oleh karena itu sangat mudah untuk menyimpan suatu vektor nilai, satu
nilai di setiap baris dan membaca kembali vektor ini.
\end{eulercomment}
\begin{eulerprompt}
>v=random(1000); mean(v)
\end{eulerprompt}
\begin{euleroutput}
  0.49728
\end{euleroutput}
\begin{eulerprompt}
>writematrix(v',file); mean(readmatrix(file)')
\end{eulerprompt}
\begin{euleroutput}
  0.49728
\end{euleroutput}
\eulerheading{Menggunakan Tabel}
\begin{eulercomment}
Tabel dapat digunakan untuk membaca atau menulis data numerik.
Misalnya, kita menulis tabel dengan header baris dan kolom ke sebuah
file.
\end{eulercomment}
\begin{eulerprompt}
>file="test.tab"; M=random(3,3);  ...
>open(file,"w");  ...
>writetable(M,separator=",",labc=["one","two","three"]);  ...
>close(); ...
>printfile(file)
\end{eulerprompt}
\begin{euleroutput}
  one,two,three
        0.65,      0.66,      0.41
        0.34,      0.55,      0.74
        0.24,      0.12,      0.93
\end{euleroutput}
\begin{eulercomment}
Ini dapat diimpor ke Excel.

Untuk membaca file di EMT, kami menggunakan readtable().
\end{eulercomment}
\begin{eulerprompt}
>\{M,headings\}=readtable(file,>clabs); ...
>writetable(M,labc=headings)
\end{eulerprompt}
\begin{euleroutput}
         one       two     three
        0.65      0.66      0.41
        0.34      0.55      0.74
        0.24      0.12      0.93
\end{euleroutput}
\eulerheading{Menganalisis Garis}
\begin{eulercomment}
Pada subbab ini sering digunakan untuk memproses atau mengekstrak data
dari teks yang berformat khusus, seperti data tabel dallam HTML. Anda
bahkan dapat mengevaluasi setiap baris dengan tangan. Misalkan, kita
memiliki baris dengan format berikut.\\
\end{eulercomment}
\begin{eulerformula}
\[
2020-11-03, Tue, 1'114.05
\]
\end{eulerformula}
\begin{eulerprompt}
>line="2020-11-03,Tue,1'114.05"
\end{eulerprompt}
\begin{euleroutput}
  2020-11-03,Tue,1'114.05
\end{euleroutput}
\begin{eulercomment}
Pertama, kita akan memisahkan string line menjadi bagian-bagian yang
lebih kecil, yang dikenal sebagai "token".
\end{eulercomment}
\begin{eulerprompt}
>vt=strtokens(line)
\end{eulerprompt}
\begin{euleroutput}
  2020-11-03
  Tue
  1'114.05
\end{euleroutput}
\begin{eulercomment}
Kemudian kita dapat mengevaluasi setiap elemen garis menggunakan
evaluasi yang sesuai.
\end{eulercomment}
\begin{eulerprompt}
>day(vt[1]);  ...
>indexof(["mon","tue","wed","thu","fri","sat","sun"],tolower(vt[2]));  ...
>strrepl(vt[3],"'","")();
\end{eulerprompt}
\begin{eulercomment}
Dengan menggunakan ekspresi reguler, dimungkinkan untuk mengekstrak
hampir semua informasi dari sebaris data.

Selanjutnya, kita akan melihat bagaimana mengekstrak data string yang
berisi markup HTML menggunakan ekspresi reguler.
\end{eulercomment}
\begin{eulerprompt}
>line="<tr><td>1145.45</td><td>5.6</td><td>-4.5</td><tr>";
\end{eulerprompt}
\begin{eulercomment}
Untuk mengekstraknya, kami menggunakan ekspresi reguler, yang mencari

\end{eulercomment}
\begin{eulerttcomment}
  - tanda kurung tutup >, untuk mengindikasikan bahwa kita akan
\end{eulerttcomment}
\begin{eulercomment}
mencari awal dari elemen yang ada di dalam tag.\\
\end{eulercomment}
\begin{eulerttcomment}
  - string apa pun yang tidak mengandung tanda kurung akan mencocokkan
\end{eulerttcomment}
\begin{eulercomment}
elemen di dalam tag \textless{}td\textgreater{}.\\
\end{eulercomment}
\begin{eulerttcomment}
  - braket pembuka dan penutup menggunakan solusi terpendek,dengan tag
\end{eulerttcomment}
\begin{eulercomment}
pembuka (\textless{}td\textgreater{}) dan penutup (\textless{}/td\textgreater{}).\\
\end{eulercomment}
\begin{eulerttcomment}
  - sekali lagi string apa pun yang tidak mengandung tanda kurung,ini
\end{eulerttcomment}
\begin{eulercomment}
akan menjamin bahwa kita akan mengambil isi yang relevan di dalam
tagnya.\\
\end{eulercomment}
\begin{eulerttcomment}
  - dan tanda kurung buka < menandai bahwa ini adalah akhir dari tag
\end{eulerttcomment}
\begin{eulercomment}
dan awal dari tag baru.

Mencari pola tertentu dalam string line yang menggunakan ekspresi
reguler.
\end{eulercomment}
\begin{eulerprompt}
>\{pos,s,vt\}=strxfind(line,">([^<>]+)<.+?>([^<>]+)<");
\end{eulerprompt}
\begin{eulercomment}
Hasilnya adalah posisi kecocokan, string yang cocok, dan vektor string
untuk sub-kecocokan.

Kita akan mengeksekusi elemen-elemen di dalam array atau list vt satu
per satu dalam sebuah perulangan.
\end{eulercomment}
\begin{eulerprompt}
>for k=1:length(vt); vt[k](), end;
\end{eulerprompt}
\begin{euleroutput}
  1145.5
  5.6
\end{euleroutput}
\begin{eulercomment}
Berikut adalah fungsi yang membaca semua item numerik antara \textless{}td\textgreater{} dan
\textless{}/td\textgreater{}.
\end{eulercomment}
\begin{eulerprompt}
>function readtd (line) ...
\end{eulerprompt}
\begin{eulerudf}
  v=[]; cp=0;
  repeat
     \{pos,s,vt\}=strxfind(line,"<td.*?>(.+?)</td>",cp);
     until pos==0;
     if length(vt)>0 then v=v|vt[1]; endif;
     cp=pos+strlen(s);
  end;
  return v;
  endfunction
\end{eulerudf}
\begin{eulercomment}
Kita akan mengekstrak dan menampilkan semua nilai yang berada di
antara tag \textless{}td\textgreater{}...\textless{}/td\textgreater{} dalam baris,dan mencari apakah nilai tersebut
numerik atau bukan.
\end{eulercomment}
\begin{eulerprompt}
>readtd(line+"<td>non-numerical</td>")
\end{eulerprompt}
\begin{euleroutput}
  1145.45
  5.6
  -4.5
  non-numerical
\end{euleroutput}
\eulerheading{Membaca dari Web}
\begin{eulercomment}
Situs web atau file dengan URL dapat dibuka di EMT dan dapat dibaca
baris demi baris.

Dalam contoh, kita membaca versi terkini dari situs EMT. Kami
menggunakan ekspresi reguler untuk memindai "Versi ..." dalam sebuah
judul.
\end{eulercomment}
\begin{eulerprompt}
>function readversion () ...
\end{eulerprompt}
\begin{eulerudf}
  urlopen("http://www.euler-math-toolbox.de/Programs/Changes.html");
  repeat
    until urleof();
    s=urlgetline();
    k=strfind(s,"Version ",1);
    if k>0 then substring(s,k,strfind(s,"<",k)-1), break; endif;
  end;
  urlclose();
  endfunction
\end{eulerudf}
\begin{eulerprompt}
>readversion
\end{eulerprompt}
\begin{euleroutput}
  Version 2024-01-12
\end{euleroutput}
\begin{eulercomment}
Contoh lain membaca URL dengan EMT\\
"https://mywebsite.com/version.h"
\end{eulercomment}
\begin{eulerprompt}
>function readversionmywebsite () ...
\end{eulerprompt}
\begin{eulerudf}
  urlopen("https://mywebsite.com/version.h");
  repeat
     until urleof();
     s=urlgetline();
     k=strfind(s,"Release",1);
     if k>0 then substring(s,k,strfind(s,"<",k)-1); break; endif;
  end;
  urlclose();
  endfunction
\end{eulerudf}
\begin{eulerprompt}
>readversionmywebsite
\end{eulerprompt}
\begin{eulercomment}
Karena string "Release" tidak ada di dalam file version.h, maka
strfind(s, "Release", 1) akan mengembalikan nilai nol atau tidak
menghasilkan indeks yang diperlukan untuk proses pencarian.

\begin{eulercomment}
\eulerheading{Input dan Output Variabel}
\begin{eulercomment}
Anda dapat menulis variabel dalam bentuk definisi Euler ke file atau
ke baris perintah.
\end{eulercomment}
\begin{eulerprompt}
>writevar(pi,"mypi");
\end{eulerprompt}
\begin{euleroutput}
  mypi = 3.141592653589793;
\end{euleroutput}
\begin{eulercomment}
Untuk pengujian, kami membuat file Euler di direktori kerja EMT.
\end{eulercomment}
\begin{eulerprompt}
>file="tes.e"; ...
>writevar(random(2,2),"M",file); ...
>printfile(file,3)
\end{eulerprompt}
\begin{euleroutput}
  M = [ ..
  0.4376044954156419, 0.1788663823174511;
  0.1555276728327629, 0.9887471739405346];
\end{euleroutput}
\begin{eulercomment}
Sekarang kita dapat memuat file tersebut. Ini akan mendefinisikan
matriks M.
\end{eulercomment}
\begin{eulerprompt}
>load(file); show M,
\end{eulerprompt}
\begin{euleroutput}
  M = 
     0.4376   0.17887 
    0.15553   0.98875 
\end{euleroutput}
\begin{eulercomment}
Omong-omong, jika writevar() digunakan pada suatu variabel, definisi
variabel dengan nama variabel tersebut akan dicetak.
\end{eulercomment}
\begin{eulerprompt}
>writevar(M); writevar(inch$)
\end{eulerprompt}
\begin{euleroutput}
  M = [ ..
  0.4376044954156419, 0.1788663823174511;
  0.1555276728327629, 0.9887471739405346];
  inch$ = 0.0254;
\end{euleroutput}
\begin{eulercomment}
Kita juga bisa membuka file baru atau menambahkan file yang sudah ada.
Dalam contoh kita menambahkan file yang dibuat sebelumnya.
\end{eulercomment}
\begin{eulerprompt}
>open(file,"a"); ...
>writevar(random(2,2),"M1"); ...
>writevar(random(3,1),"M2"); ...
>close();
>load(file); show M1; show M2;
\end{eulerprompt}
\begin{euleroutput}
  M1 = 
    0.91404   0.92963 
    0.47914   0.44517 
  M2 = 
  0.0097853 
    0.78846 
    0.53184 
\end{euleroutput}
\begin{eulercomment}
Untuk menghapus file apa pun, gunakan fileremove().
\end{eulercomment}
\begin{eulerprompt}
>fileremove(file);
\end{eulerprompt}
\begin{eulercomment}
Vektor baris dalam suatu file tidak memerlukan koma, jika setiap angka
berada pada baris baru. Mari kita buat file seperti itu, tulis setiap
baris satu per satu dengan writeln().
\end{eulercomment}
\begin{eulerprompt}
>open(file,"w"); writeln("M = ["); ...
>for i=1 to 5; writeln(""+random()); end; ...
>writeln("];"); close(); ...
>printfile(file)
\end{eulerprompt}
\begin{euleroutput}
  M = [
  0.133712072096
  0.779233039
  0.250150421757
  0.938737688955
  0.874792276582
  ];
\end{euleroutput}
\begin{eulerprompt}
>load(file); M
\end{eulerprompt}
\begin{euleroutput}
  [0.13371,  0.77923,  0.25015,  0.93874,  0.87479]
\end{euleroutput}
\eulersubheading{LATIHAN}
\begin{eulercomment}
1. Misalkan anda memiliki vektor x=[2,4,6,8,10]\\
a. buatkan vektor yang menggabungkan vektor x,angka0dan vektorx lagi\\
b. tentukan apakah setiap elemen vektor x lebih besar dari 5(hasil
logika 1 untuk benar dan 0 untuk salah)

\end{eulercomment}
\begin{eulerprompt}
>x:=[2,4,6,8,10]; [x,0,x]
\end{eulerprompt}
\begin{euleroutput}
  [2,  4,  6,  8,  10,  0,  2,  4,  6,  8,  10]
\end{euleroutput}
\begin{eulerprompt}
>x>5, %*x
\end{eulerprompt}
\begin{euleroutput}
  [0,  0,  1,  1,  1]
  [0,  0,  6,  8,  10]
\end{euleroutput}
\begin{eulercomment}
2. Tentukan matriks X dengan elemen-elemen yang berurutan dari 1
hingga 20 dan susunlah elemen tersebut menjadi matriks berukuran 5x4.
\end{eulercomment}
\begin{eulerprompt}
>shortformat; X=redim(1:20,5,4)
\end{eulerprompt}
\begin{euleroutput}
          1         2         3         4 
          5         6         7         8 
          9        10        11        12 
         13        14        15        16 
         17        18        19        20 
\end{euleroutput}
\begin{eulercomment}
3.Seorang analis memiliki data penjualan harian selama 5
hari(150,200,250,300,350) yang disimpan dalam bentuk vektor sebagai
berikut:\\
a. mean(rata-rata)\\
b. deviasi standar
\end{eulercomment}
\begin{eulerprompt}
>penjualan=[150,200,250,300,350]
\end{eulerprompt}
\begin{euleroutput}
  [150,  200,  250,  300,  350]
\end{euleroutput}
\begin{eulercomment}
atau anda bisa memanggil data yang sudah dibuat 
\end{eulercomment}
\begin{eulerprompt}
>filename="penjualan.dat";
>writematrix(penjualan',filename)
>penjualan=readmatrix(filename)'
\end{eulerprompt}
\begin{euleroutput}
  [150,  200,  250,  300,  350]
\end{euleroutput}
\begin{eulerprompt}
>mean(penjualan)
\end{eulerprompt}
\begin{euleroutput}
  250
\end{euleroutput}
\begin{eulerprompt}
>dev(penjualan)
\end{eulerprompt}
\begin{euleroutput}
  79.057
\end{euleroutput}
\begin{eulercomment}
4. Buat fungsi yang membuka URL\\
"https://en.wikipedia.org/wiki/Euler\_(software)"\\
dan mencari kata "Versi" di dalam URL tersebut, dan tampilkan
hasilnya.
\end{eulercomment}
\begin{eulerprompt}
>function readversionwebsite () ...
\end{eulerprompt}
\begin{eulerudf}
  urlopen("https://en.wikipedia.org/wiki/Euler_(software)");
  repeat
     until urleof();
     s=urlgetline();
     k=strfind(s,"version",1);
     if k>0 then substring(s,k,strfind(s,"<",k)-1), break; endif;
  end;
  urlclose();
  endfunction
\end{eulerudf}
\begin{eulerprompt}
>readversion
\end{eulerprompt}
\begin{euleroutput}
  Version 2024-01-12
\end{euleroutput}
\begin{eulercomment}
5.Diberikan data pengukuran tinggi badan pada kelas matematika B
adalah sebagai berikut:

\end{eulercomment}
\begin{eulerttcomment}
           | Rentang Tinggi (cm) | Jumlah Orang |
           |---------------------|--------------|
           | 155.5 - 159.5       |      22      |
           | 159.5 - 163.5       |      71      |
           | 163.5 - 167.5       |     136      |
           | 167.5 - 171.5       |     169      |
           | 171.5 - 175.5       |     139      |
           | 175.5 - 179.5       |      71      |
           | 179.5 - 183.5       |      32      |
           | 183.5 - 187.5       |       8      |
\end{eulerttcomment}
\begin{eulercomment}

a.)Hitung rata-rata dan deviasi standar dari distribusi tinggi badan
ini.\\
b.)Plot distribusi frekuensi data (diagram batang).\\
c.)Tambahkan kurva distribusi normal untuk dibandingkan dengan data.
\end{eulercomment}
\begin{eulerprompt}
>r = 155.5:4:187.5  //Rentang ukuran tinggi badan
\end{eulerprompt}
\begin{euleroutput}
  [155.5,  159.5,  163.5,  167.5,  171.5,  175.5,  179.5,  183.5,  187.5]
\end{euleroutput}
\begin{eulerprompt}
>v = [22, 71, 136, 169, 139, 71, 32, 8] //Jumlah orang dalam tiap rentang
\end{eulerprompt}
\begin{euleroutput}
  [22,  71,  136,  169,  139,  71,  32,  8]
\end{euleroutput}
\begin{eulerprompt}
>l=fold(r,[0.5,0.5])  //Menghitung titik tengah dari setiap rentang tinggi badan
\end{eulerprompt}
\begin{euleroutput}
  [157.5,  161.5,  165.5,  169.5,  173.5,  177.5,  181.5,  185.5]
\end{euleroutput}
\begin{eulerprompt}
>\{m,d\}=meandev(l,v); m, d,  //Hitung rata-rata dan deviasi standar
\end{eulerprompt}
\begin{euleroutput}
  169.9
  5.9891
\end{euleroutput}
\begin{eulerprompt}
>plot2d(r, v, a=150, b=200, c=0, d=190, bar=1, style="\(\backslash\)/"):
\end{eulerprompt}
\eulerimg{15}{images/Tugas Individu Pekan13-14_Muhammad Lutfi Ramadhan_23030630021-083.png}
\begin{eulerprompt}
>plot2d("qnormal(x, m, d) * sum(v) * 4", ...
>xmin=min(r), xmax=max(r), thickness=3, add=1):
\end{eulerprompt}
\eulerimg{15}{images/Tugas Individu Pekan13-14_Muhammad Lutfi Ramadhan_23030630021-084.png}
\begin{eulerprompt}
>&remvalue();
\end{eulerprompt}
\begin{eulercomment}
6. Sebuah survei dilakukan untuk mengetahui jumlah jam belajar siswa
SMA dalam satu minggu. Berikut data jam belajar dari 10 siswa: 8, 10,
7, 6, 9, 10, 11, 9, 8, 12.\\
a) Hitung nilai rata-rata dari data di atas\\
b) Tentukan median dari data tersebut.
\end{eulercomment}
\begin{eulerprompt}
>M=[8,10,7,6,9,10,11,9,8,12];
>mean(M)
\end{eulerprompt}
\begin{euleroutput}
  9
\end{euleroutput}
\begin{eulerprompt}
>median(M)
\end{eulerprompt}
\begin{euleroutput}
  9
\end{euleroutput}
\begin{eulercomment}
7. Anda diberikan data yang menunjukkan jumlah penjualan barang selama
12 bulan dalam satu tahun berturut-turut 120, 135, 150, 160, 170, 180,
190, 210, 200, 220, 230, 240.\\
a) Buatlah plot garis dari data penjualan barang tersebut.\\
b) Hitung rata-rata penjualan perbulan.
\end{eulercomment}
\begin{eulerprompt}
>X=[120,135,150,160,170,180,190,210,200,220,230,240]
\end{eulerprompt}
\begin{euleroutput}
  [120,  135,  150,  160,  170,  180,  190,  210,  200,  220,  230,  240]
\end{euleroutput}
\begin{eulerprompt}
>Y=[1,2,3,4,5,6,7,8,9,10,11,12]
\end{eulerprompt}
\begin{euleroutput}
  [1,  2,  3,  4,  5,  6,  7,  8,  9,  10,  11,  12]
\end{euleroutput}
\begin{eulerprompt}
>statplot(Y,X,"l"):
\end{eulerprompt}
\eulerimg{15}{images/Tugas Individu Pekan13-14_Muhammad Lutfi Ramadhan_23030630021-085.png}
\begin{eulerprompt}
>mean(X)
\end{eulerprompt}
\begin{euleroutput}
  183.75
\end{euleroutput}
\end{eulernotebook}
\end{document}
