\documentclass[a4paper,10pt]{article}
\usepackage{eumat}

\begin{document}
\begin{eulernotebook}
\begin{eulercomment}
Nama: Muhammad Lutfi Ramadhan\\
NIM: 23030630021\\
Kelas: Matematika B 2023

\begin{eulercomment}
\eulerheading{EMT untuk Perhitungan Aljabar}
\begin{eulercomment}
Pada notebook ini Anda belajar menggunakan EMT untuk melakukan
berbagai perhitungan terkait dengan materi atau topik dalam Aljabar.
Kegiatan yang harus Anda lakukan adalah sebagai berikut:

- Membaca secara cermat dan teliti notebook ini;\\
- Menerjemahkan teks bahasa Inggris ke bahasa Indonesia;\\
- Mencoba contoh-contoh perhitungan (perintah EMT) dengan cara meng
ENTER setiap perintah EMT yang ada (pindahkan kursor ke baris
perintah)\\
- Jika perlu Anda dapat memodifikasi perintah yang ada dan memberikan
keterangan/penjelasan tambahan terkait hasilnya.\\
- Menyisipkan baris-baris perintah baru untuk mengerjakan soal-soal
Aljabar dari file PDF yang saya berikan;\\
- Memberi catatan hasilnya.\\
- Jika perlu tuliskan soalnya pada teks notebook (menggunakan format
LaTeX).\\
- Gunakan tampilan hasil semua perhitungan yang eksak atau simbolik
dengan format LaTeX. (Seperti contoh-contoh pada notebook ini.)

\end{eulercomment}
\eulersubheading{Contoh pertama}
\begin{eulercomment}
Menyederhanakan bentuk aljabar:

\end{eulercomment}
\begin{eulerformula}
\[
5x^{-4}y^3\times -8x^5y^{-6}
\]
\end{eulerformula}
\begin{eulercomment}
\end{eulercomment}
\begin{eulerprompt}
>$&5*x^(-4)*y^3*-8*x^5*y^(-6)
\end{eulerprompt}
\begin{eulerformula}
\[
-\frac{40\,x}{y^3}
\]
\end{eulerformula}
\begin{eulercomment}
Menyederhanakan fungsi :\\
\end{eulercomment}
\begin{eulerformula}
\[
5y^2+3x^6-7y^2+2x^2
\]
\end{eulerformula}
\begin{eulerprompt}
>$&5*y^2+3*x^6-7*y^2+2*x^2
\end{eulerprompt}
\begin{eulerformula}
\[
-2\,y^2+3\,x^6+2\,x^2
\]
\end{eulerformula}
\begin{eulercomment}
Menjabarkan:

\end{eulercomment}
\begin{eulerformula}
\[
(5x^{-4}+y^3) (-8x^5-y^{-6})
\]
\end{eulerformula}
\begin{eulerprompt}
>$&showev('expand((5*x^\{-4\}+y^3)*(-8*x^5-y^\{-6\})))
\end{eulerprompt}
\begin{eulerformula}
\[
{\it expand}\left(\left(y^3+5\,x^{\left \{-4 \right \}}\right)\,  \left(-y^{\left \{-6 \right \}}-8\,x^5\right)\right)=-y^{\left \{-6   \right \}+3}-5\,x^{\left \{-4 \right \}}\,y^{\left \{-6 \right \}}-  8\,x^5\,y^3-40\,x^{\left \{-4 \right \}+5}
\]
\end{eulerformula}
\eulersubheading{Baris Perintah}
\begin{eulercomment}
Baris perintah Euler terdiri dari satu atau beberapa perintah Euler
diikuti dengan titik koma ";" atau koma ",". Titik koma mencegah
pencetakan hasil. Koma setelah perintah terakhir dapat dihilangkan.

Baris perintah berikut hanya akan mencetak hasil ekspresi, bukan tugas
atau perintah format.
\end{eulercomment}
\begin{eulerprompt}
>r:=3; h:=4; pi*r^2*h/3
\end{eulerprompt}
\begin{euleroutput}
  37.6991118431
\end{euleroutput}
\begin{eulercomment}
Perintah harus dipisahkan dengan yang kosong. Baris perintah berikut
mencetak dua hasilnya.
\end{eulercomment}
\begin{eulerprompt}
>pi*2*r*h, %+2*pi*r*h // Ingat tanda % menyatakan hasil perhitungan terakhir sebelumnya.
\end{eulerprompt}
\begin{euleroutput}
  75.3982236862
  150.796447372
\end{euleroutput}
\begin{eulercomment}
Baris perintah dieksekusi dalam urutan yang ditekan pengguna kembali.
Jadi Anda mendapatkan nilai baru setiap kali Anda menjalankan baris
kedua.
\end{eulercomment}
\begin{eulerprompt}
>x := 4;
>x := cos(x) // nilai cosinus (x dalam radian)
\end{eulerprompt}
\begin{euleroutput}
  -0.653643620864
\end{euleroutput}
\begin{eulerprompt}
>x := cos(x)
\end{eulerprompt}
\begin{euleroutput}
  0.793873449226
\end{euleroutput}
\begin{eulercomment}
Jika dua garis terhubung dengan "..." kedua garis akan selalu
dieksekusi secara bersamaan.
\end{eulercomment}
\begin{eulerprompt}
>x := 3.5; ...
>x := (x+2/x)/2, x := (x+2/x)/2, x := (x+2/x)/2,
\end{eulerprompt}
\begin{euleroutput}
  2.03571428571
  1.50908521303
  1.41719571011
\end{euleroutput}
\begin{eulercomment}
Ini juga merupakan cara yang baik untuk menyebarkan long command pada
dua atau lebih baris. Anda dapat menekan Ctrl+Return untuk membagi
garis menjadi dua pada posisi kursor saat ini, atau Ctrl+Back untuk
menggabungkan garis.

Sedangkan untuk fold semua multi-garis tekan Ctrl + L. Kemudian
garis-garis berikutnya hanya akan terlihat, jika salah satunya
memiliki fokus. Untuk fold satu multi-baris, mulailah baris pertama
dengan "\%+".
\end{eulercomment}
\begin{eulerprompt}
>%+ x=4+5; ...
\end{eulerprompt}
\begin{eulercomment}
Garis yang diawali dengan \%\% tidak akan terlihat sama sekali.
\end{eulercomment}
\begin{euleroutput}
  81
\end{euleroutput}
\begin{eulercomment}
Euler Math Toolbox mendukung loop di baris perintah, selama mereka
masuk ke dalam satu baris atau multi-baris. Dalam program, pembatasan
ini tidak berlaku, tentu saja. Untuk informasi lebih lanjut lihat
pengantar berikut.
\end{eulercomment}
\begin{eulerprompt}
>x=6; for i=1 to 10; x := (x+2/x)/2, end; // menghitung akar 2
\end{eulerprompt}
\begin{euleroutput}
  3.16666666667
  1.89912280702
  1.47612029496
  1.4155117098
  1.41421415763
  1.41421356237
  1.41421356237
  1.41421356237
  1.41421356237
  1.41421356237
\end{euleroutput}
\begin{eulercomment}
Tidak apa-apa untuk menggunakan multi-line. Pastikan baris diakhiri
dengan "...".
\end{eulercomment}
\begin{eulerprompt}
>x := 2.5; // comments go here before the ...
>repeat xnew:=(x+2/x)/2; until xnew~=x; ...
>   x := xnew; ...
>end; ...
>x,
\end{eulerprompt}
\begin{euleroutput}
  1.41421356237
\end{euleroutput}
\begin{eulercomment}
Struktur bersyarat juga berfungsi.
\end{eulercomment}
\begin{eulerprompt}
>if E^pi>pi^E; then "Halo Rakyatku!", endif;
\end{eulerprompt}
\begin{euleroutput}
  Halo Rakyatku!
\end{euleroutput}
\begin{eulercomment}
Saat Anda menjalankan perintah, kursor dapat berada di posisi mana pun
di baris perintah. Anda dapat kembali ke perintah sebelumnya atau
melompat ke perintah berikutnya dengan tombol panah. Atau Anda dapat
mengklik ke bagian komentar di atas perintah untuk menuju ke perintah.

Saat Anda menggerakkan kursor di sepanjang garis, pasangan tanda
kurung atau kurung buka dan tutup akan disorot. Dan juga, perhatikan
baris status. Setelah kurung buka fungsi sqrt(), baris status akan
menampilkan teks bantuan untuk fungsi tersebut. Jalankan perintah
dengan tombol kembali.
\end{eulercomment}
\begin{eulerprompt}
>sqrt(sin(45°)/cos(60°))
\end{eulerprompt}
\begin{euleroutput}
  1.189207115
\end{euleroutput}
\begin{eulercomment}
Untuk melihat bantuan untuk perintah terbaru, buka jendela bantuan
dengan F1. Di sana, Anda dapat memasukkan teks untuk dicari. Pada
baris kosong, bantuan untuk jendela bantuan akan ditampilkan. Anda
dapat menekan escape untuk menghapus garis, atau untuk menutup jendela
bantuan.

Anda dapat mengklik dua kali pada perintah apa pun untuk membuka
bantuan untuk perintah ini. Coba klik dua kali perintah exp di bawah
ini di baris perintah.
\end{eulercomment}
\begin{eulerprompt}
>exp(log(5.7))
\end{eulerprompt}
\begin{euleroutput}
  5.7
\end{euleroutput}
\begin{eulercomment}
\end{eulercomment}
\eulersubheading{Sintaks Dasar}
\begin{eulercomment}
Euler Math Toolbox tahu fungsi matematika yang biasa digunakan.
Seperti yang Anda lihat di atas, fungsi trigonometri bekerja dalam
radian atau derajat. Untuk mengonversi ke derajat, tambahkan simbol
derajat (dengan tombol F7) ke dalam nilainya, atau gunakan fungsi
rad(x). Fungsi akar kuadrat disebut sqrt dalam Euler. Tentu saja,
x\textasciicircum{}(1/2) juga memungkinkan.

Untuk menyetel variabel, gunakan "=" atau ":=". Demi kejelasan,
pengantar ini menggunakan bentuk yang terakhir/terbaru. Spasi tidak
menjadi masalah. Tetapi ruang antara perintah diharapkan untuk ada.

Beberapa perintah dalam satu baris dipisahkan dengan "," atau ";".
Titik koma menekan output dari perintah. Di akhir baris perintah ","
diasumsikan, jika ";" hilang.
\end{eulercomment}
\begin{eulerprompt}
>g:=10.73; t:=4.2; 1/2*g*t^2
\end{eulerprompt}
\begin{euleroutput}
  94.6386
\end{euleroutput}
\begin{eulercomment}
EMT menggunakan sintaks pemrograman untuk ekspresi. Untuk mengetik

\end{eulercomment}
\begin{eulerformula}
\[
e^2 \cdot \left( \frac{1}{9+5 \log(0.7)}+\frac{4}{9} \right)
\]
\end{eulerformula}
\begin{eulercomment}
Anda harus mengatur tanda kurung dengan benar dan menggunakan "/"
untuk pecahan. Perhatikan tanda kurung yang disorot untuk bantuan.
Perhatikan bahwa konstanta Euler e diberi nama E dalam EMT.
\end{eulercomment}
\begin{eulerprompt}
>E^2*(1/(9+5*log(0.7))+4/9)
\end{eulerprompt}
\begin{euleroutput}
  4.30791848586
\end{euleroutput}
\begin{eulercomment}
Untuk menghitung ekspresi rumit seperti

\end{eulercomment}
\begin{eulerformula}
\[
\left(\frac{\frac58 + \frac76 + 3}{\frac37 + \frac59}\right)^2 \pi
\]
\end{eulerformula}
\begin{eulercomment}
Anda harus memasukkannya dalam bentuk baris.
\end{eulercomment}
\begin{eulerprompt}
>((5/8 + 7/6 + 3) / (3/7 + 5/9))^2 * pi
\end{eulerprompt}
\begin{euleroutput}
  74.4767625442
\end{euleroutput}
\begin{eulercomment}
Letakkan tanda kurung dengan hati-hati di sekitar sub-ekspresi yang
perlu dihitung terlebih dahulu. EMT membantu Anda dengan menyorot
ekspresi bahwa braket penutup selesai. Anda juga harus memasukkan nama
"pi" untuk huruf Yunani pi.

Hasil dari perhitungan ini adalah bilangan floating point. Secara
default dicetak dengan akurasi sekitar 12 digit. Di baris perintah
berikut, kita juga belajar bagaimana kita bisa merujuk ke hasil
sebelumnya dalam baris yang sama.
\end{eulercomment}
\begin{eulerprompt}
>2/3+5/7, fraction %
\end{eulerprompt}
\begin{euleroutput}
  1.38095238095
  29/21
\end{euleroutput}
\begin{eulercomment}
Perintah Euler dapat berupa ekspresi atau perintah primitif. Ekspresi
terbuat dari operator dan fungsi. Jika diperlukan, hal tersebut harus
berisi tanda kurung untuk memaksa urutan eksekusi yang benar. Jika
ragu, memasang braket atau tanda kurung adalah ide yang bagus.
Perhatikan bahwa EMT menunjukkan tanda kurung buka dan tutup saat
mengedit baris perintah.
\end{eulercomment}
\begin{eulerprompt}
>(cos(pi/4)+2)^3*(sin(pi/4)+5)^2
\end{eulerprompt}
\begin{euleroutput}
  646.172032434
\end{euleroutput}
\begin{eulercomment}
Operator numerik Euler meliputi

\end{eulercomment}
\begin{eulerttcomment}
  + unary atau operator plus
  - unary atau operator minus
  * operator perkalian
  / operator pecahan
  . produk matriks
  a^b daya untuk positif a atau bilangan bulat b (a**b juga berfungsi)
  n! operator faktorial
\end{eulerttcomment}
\begin{eulercomment}

dan masih banyak lagi.

Berikut adalah beberapa fungsi yang mungkin Anda butuhkan. Ada banyak
lagi.

\end{eulercomment}
\begin{eulerttcomment}
  sin, cos, tan, atan, asin, acos, rad, deg
  log, exp, log10, sqrt, logbase
  bin, logbin, logfac, mod, lantai, ceil, bulat, abs, tanda
  conj, re, im, arg, conj, nyata, kompleks
  beta, betai, gamma, complexgamma, ellrf, ellf, ellrd, elle
  bitand, bitor, bitxor, bitnot
\end{eulerttcomment}
\begin{eulercomment}

Beberapa perintah memiliki alias, mis. ln untuk log.
\end{eulercomment}
\begin{eulerprompt}
>ln(E^4), arctan(tan(0.75)), logbase(30,10)
\end{eulerprompt}
\begin{euleroutput}
  4
  0.75
  1.47712125472
\end{euleroutput}
\begin{eulerprompt}
>sin(60°)
\end{eulerprompt}
\begin{euleroutput}
  0.866025403784
\end{euleroutput}
\begin{eulercomment}
Pastikan untuk menggunakan tanda kurung (kurung bulat), setiap kali
ada keraguan tentang urutan eksekusi! Berikut ini tidak sama dengan
(2\textasciicircum{}3)\textasciicircum{}4, yang merupakan default untuk 2\textasciicircum{}3\textasciicircum{}4 di EMT (beberapa sistem
numerik melakukannya dengan cara lain).
\end{eulercomment}
\begin{eulerprompt}
>2^3^4, (2^3)^4, 2^(3^4)
\end{eulerprompt}
\begin{euleroutput}
  2.41785163923e+24
  4096
  2.41785163923e+24
\end{euleroutput}
\eulersubheading{Bilangan Asli}
\begin{eulercomment}
Tipe data utama dalam Euler adalah bilangan real. Real
direpresentasikan dalam format IEEE dengan akurasi sekitar 16 digit
desimal.
\end{eulercomment}
\begin{eulerprompt}
>longest(23/3)
\end{eulerprompt}
\begin{euleroutput}
        7.666666666666667 
\end{euleroutput}
\begin{eulercomment}
Representasi ganda internal membutuhkan 8 byte.\\
Representasi ganda adalah format penyimpanan untuk floating-point yang
menggunakan 64 bit(8 byte)
\end{eulercomment}
\begin{eulerprompt}
>printdual(23/3)
\end{eulerprompt}
\begin{euleroutput}
  1.1110101010101010101010101010101010101010101010101011*2^2
\end{euleroutput}
\begin{eulerprompt}
>printhex(1/7)
\end{eulerprompt}
\begin{euleroutput}
  2.4924924924924*16^-1
\end{euleroutput}
\begin{eulercomment}
Perbedaan 'printdual' dan 'printhex' adalah 'printdual' yakni mencetak
representasi internal dari sebuah bilangan floating-point dalam format
presisi ganda (pendekatan yang sangat dekat dengan nilai aslinya
tetapi tidak persis sama.) meskipun ia tergantung pada konteks bahasa
pemograman tertentu. sedangkan 'printhex' yakni representasi dari
nilai floating-point dalam bentuk heksadesimal(basis 16), heksadesimal
ini adalah cara yang lebih ringkas untuk menampilkan nilai biner
karena setiap digit heksadesimal mempresentasikan empat digit biner.\\
\end{eulercomment}
\eulersubheading{}
\begin{eulercomment}
\end{eulercomment}
\eulersubheading{String }
\begin{eulercomment}
Sebuah string dalam Euler didefinisikan dengan "..."
\end{eulercomment}
\begin{eulerprompt}
>"A string can contain anything."
\end{eulerprompt}
\begin{euleroutput}
  A string can contain anything.
\end{euleroutput}
\begin{eulercomment}
String dapat digabungkan dengan \textbar{} atau dengan +. Ini juga berfungsi
dengan angka, yang dikonversi menjadi string dalam kasus itu.
\end{eulercomment}
\begin{eulerprompt}
>"Terjadi Gempa Mag pada hari Senin 26 Agustus 2024 dengan pusat gempa berada di laut " +95+ " km barat daya Gunungkidul."
\end{eulerprompt}
\begin{euleroutput}
  Terjadi Gempa Mag pada hari Senin 26 Agustus 2024 dengan pusat gempa berada di laut 95 km barat daya Gunungkidul.
\end{euleroutput}
\begin{eulercomment}
Pada String fungsi print mengonversi angka menjadi string. Ini dapat
mengambil sejumlah digit dan sejumlah tempat (0 untuk keluaran padat),
dan secara optimal satu unit
\end{eulercomment}
\begin{eulerprompt}
>"Golden Ratio : " + print((1+sqrt(5))/2,5,0)
\end{eulerprompt}
\begin{euleroutput}
  Golden Ratio : 1.61803
\end{euleroutput}
\begin{eulercomment}
Terdapat spesial string 'none', yang tidak dicetak.
\end{eulercomment}
\begin{eulerprompt}
>none
\end{eulerprompt}
\begin{eulercomment}
Untuk mengonversi string menjadi angka, cukup mengevaluasinya. Ini\\
bekerja untuk ekspresi juga (lihat dibawah).
\end{eulercomment}
\begin{eulerprompt}
>"1234.5567"()
\end{eulerprompt}
\begin{euleroutput}
  1234.5567
\end{euleroutput}
\begin{eulercomment}
Untuk mendefinisikan vektor string, gunakan notasi vektor [...]
\end{eulercomment}
\begin{eulerprompt}
>v:= ["Indonesia","Malaysia","Brunei Darussalam"]
\end{eulerprompt}
\begin{euleroutput}
  Indonesia
  Malaysia
  Brunei Darussalam
\end{euleroutput}
\begin{eulercomment}
Vektor pada string kosong dilambangkan dengan [none]. Dan vektor
string dapat digabungkan dengan '\textbar{}'.
\end{eulercomment}
\begin{eulerprompt}
>w:= [none] ; w|v|v
\end{eulerprompt}
\begin{euleroutput}
  Indonesia
  Malaysia
  Brunei Darussalam
  Indonesia
  Malaysia
  Brunei Darussalam
\end{euleroutput}
\begin{eulercomment}
String dapat berisi karakter Unicode. Secara internal, string ini\\
berisi kode UTF-8. untuk menghasilkan string seperti itu, gunakan\\
u"..." dan salah satu entitas HTML. String Unicode dapat digabungkan\\
seperti string lainnya.
\end{eulercomment}
\begin{eulerprompt}
>u"&beta; = " + 90 + u"&deg; " // pdfLaTeX mungkin gagal menampilkan secara benar
\end{eulerprompt}
\begin{euleroutput}
  β = 90° 
\end{euleroutput}
\begin{eulercomment}
Dalam komentar, entitas yang sama seperti alpha; beta; dll dapat\\
digunakan untuk lateks. \\
Ada beberapa fungsi untuk membuat atau menganalisis string unicode.\\
Fungsi strtochsr() akan mengenali string Unicode, dan menerjemahkannya\\
dengan benar.
\end{eulercomment}
\begin{eulerprompt}
>v=strtochar(u"&Auml; is a German letter")
\end{eulerprompt}
\begin{euleroutput}
  [196,  32,  105,  115,  32,  97,  32,  71,  101,  114,  109,  97,  110,
  32,  108,  101,  116,  116,  101,  114]
\end{euleroutput}
\begin{eulercomment}
Perintah ini menghasilkan array atau daftar angka berupa vektor angka
yang mewakili karakter dalam string dalam bentuk kode Unicode.\\
Fungsi kebalikannya adalah chartoutf().
\end{eulercomment}
\begin{eulerprompt}
>v[1]=strtochar(u"&Auml;")[1]; chartoutf(v)
\end{eulerprompt}
\begin{euleroutput}
  Ä is a German letter
\end{euleroutput}
\begin{eulercomment}
Fungsi utf()dapat menerjemahkan string dengan entitas dalam variabel
menjadi string Unicode.
\end{eulercomment}
\begin{eulerprompt}
>a="We have &alpha;=&beta;."; utf(a)// PdfLaTeX mengkin gagal menampilkannya
\end{eulerprompt}
\begin{euleroutput}
  We have α=β.
\end{euleroutput}
\begin{eulercomment}
Memungkinkan juga untuk menggunakan entitas numerik.
\end{eulercomment}
\begin{eulerprompt}
>u"&#196;lphabet"
\end{eulerprompt}
\begin{euleroutput}
  Älphabet
\end{euleroutput}
\eulersubheading{Nilai Boolean}
\begin{eulercomment}
Nilai boolean direpresentasikan dengan 1=true atau 0=false dalam
euler. String dapat dibandingkan, seperti halnya angka.
\end{eulercomment}
\begin{eulerprompt}
>"saya">"aku", 6==3
\end{eulerprompt}
\begin{euleroutput}
  1
  0
\end{euleroutput}
\begin{eulerprompt}
>5>1, "mobil"=="motor"
\end{eulerprompt}
\begin{euleroutput}
  1
  0
\end{euleroutput}
\begin{eulercomment}
"dan" adalah operator "\&\&" dan "atau" adalah operator "\textbar{}\textbar{}", seperti
dalam bahasa C. (Kata-kata "dan" dan "atau" hanya dapat digunakan
dalam kondisi "jika".
\end{eulercomment}
\begin{eulerprompt}
>2<E || E<3
\end{eulerprompt}
\begin{euleroutput}
  1
\end{euleroutput}
\begin{eulerprompt}
>6>E && E<2
\end{eulerprompt}
\begin{euleroutput}
  0
\end{euleroutput}
\begin{eulercomment}
Operator Boolean mematuhi aturan bahasa matriks
\end{eulercomment}
\begin{eulerprompt}
>(2:9)>3, nonzeros (%)
\end{eulerprompt}
\begin{euleroutput}
  [0,  0,  1,  1,  1,  1,  1,  1]
  [3,  4,  5,  6,  7,  8]
\end{euleroutput}
\begin{eulercomment}
Kita dapat menggunakan fungsi bukan nol() untuk mengekstrak elemen
tertentu dari vektor. Dalam contoh,menggunakan isprima bersyarat(n).
\end{eulercomment}
\begin{eulerprompt}
>N=5 | 7:2:50 // N berisi elemen 5 dan bilangan bilangan ganjil dari 7:50
\end{eulerprompt}
\begin{euleroutput}
  [5,  7,  9,  11,  13,  15,  17,  19,  21,  23,  25,  27,  29,  31,  33,
  35,  37,  39,  41,  43,  45,  47,  49]
\end{euleroutput}
\begin{eulerprompt}
>N[nonzeros(isprime(N))] //pilih anggota anggota N yang prima
\end{eulerprompt}
\begin{euleroutput}
  [5,  7,  11,  13,  17,  19,  23,  29,  31,  37,  41,  43,  47]
\end{euleroutput}
\eulersubheading{Output Formats}
\begin{eulercomment}
Default output formats EMT adalah 12 digit. Untuk memastikan yang kita
lihat adalah bentuk default, maka perlu direset format.
\end{eulercomment}
\begin{eulerprompt}
>defformat; pi
\end{eulerprompt}
\begin{euleroutput}
  3.14159265359
\end{euleroutput}
\begin{eulercomment}
Secara internal, EMT menggunakan standar IEEE (Institute of Electrical
and Electronics Engineers) untuk bilangan ganda dengan sekitar 16
digit desimal. Untuk melihat bentuk digit penuh, gunakan perintah
"longestformat" atau gunakan operator "longest" untuk memunculkannya.
\end{eulercomment}
\begin{eulerprompt}
>longest pi
\end{eulerprompt}
\begin{euleroutput}
        3.141592653589793 
\end{euleroutput}
\begin{eulerprompt}
>longestformat; pi
\end{eulerprompt}
\begin{euleroutput}
  3.141592653589793
\end{euleroutput}
\begin{eulercomment}
Berikut ini adalah repesentasi heksadesimal internal dari bilangan
ganda.
\end{eulercomment}
\begin{eulerprompt}
>printhex(pi)
\end{eulerprompt}
\begin{euleroutput}
  3.243F6A8885A30*16^0
\end{euleroutput}
\begin{eulercomment}
Heksadesimal adalah sistem bilangan yang menggunakan basis 16. Di mana
angka 0 hingga 9 (untuk mewakili nilai 0 hingga 9) dan huruf A hingga
F (untuk mewakili nilai 10 hingga 15).

Format standarnya adalah 12.

\end{eulercomment}
\begin{eulerprompt}
>format(12,5); 1/3, pi, sin(1)
\end{eulerprompt}
\begin{euleroutput}
      0.33333 
      3.14159 
      0.84147 
\end{euleroutput}
\begin{eulerprompt}
>format(12); 1/7
\end{eulerprompt}
\begin{euleroutput}
  0.1428571
\end{euleroutput}
\begin{eulercomment}
Format output dapat diubah secara permanen dengan perintah format.
\end{eulercomment}
\begin{eulerprompt}
>format(12,5); 1/9, pi, cos(1)
\end{eulerprompt}
\begin{euleroutput}
      0.11111 
      3.14159 
      0.54030 
\end{euleroutput}
\begin{eulercomment}
Fungsi seperti "shortestformat", "shortformat", "longformat" bekerja
untuk vektor dengan cara berikut.
\end{eulercomment}
\begin{eulerprompt}
>shortestformat; random(3,8)
\end{eulerprompt}
\begin{euleroutput}
    0.66    0.2   0.89   0.28   0.53   0.31   0.44    0.3 
    0.28   0.88   0.27    0.7   0.22   0.45   0.31   0.91 
    0.19   0.46  0.095    0.6   0.43   0.73   0.47   0.32 
\end{euleroutput}
\begin{eulercomment}
Format standar untuk skalar adalah 12, tetapi ini dapat diubah.
\end{eulercomment}
\begin{eulerprompt}
>setscalarformat(7); pi
\end{eulerprompt}
\begin{euleroutput}
      3.14159 
\end{euleroutput}
\begin{eulercomment}
Begitu juga dengan fungsi "longestformat" mengatur format skalar.
\end{eulercomment}
\begin{eulerprompt}
>longestformat; pi
\end{eulerprompt}
\begin{euleroutput}
  3.141592653589793
\end{euleroutput}
\begin{eulercomment}
Notes: beberapa format output yang penting.

\end{eulercomment}
\begin{eulerttcomment}
 shortestformat shortformat longformat, longestformat
 format(length,digits) goodformat(length)
 fracformat(length)
 defformat
\end{eulerttcomment}
\begin{eulercomment}

Akurasi internal EMT adalah sekitar 16 digit desimal mengikuti standar
dari IEEE. Angka disimpan dalam format internal. Namun, format output
EMT dapat diatur secara fleksibel.

\end{eulercomment}
\begin{eulerprompt}
>longestformat; pi,
\end{eulerprompt}
\begin{euleroutput}
  3.141592653589793
\end{euleroutput}
\begin{eulerprompt}
>format(10,5); pi
\end{eulerprompt}
\begin{euleroutput}
    3.14159 
\end{euleroutput}
\begin{eulercomment}
Standarnya adalah defformat().
\end{eulercomment}
\begin{eulerprompt}
>defformat; // default
\end{eulerprompt}
\begin{eulercomment}
Ada operator pendek yang hanya mencetak satu nilai. Operator
"terpanjang" akan mencetak semua digit angka yang valid.
\end{eulercomment}
\begin{eulerprompt}
>longest pi^2/2
\end{eulerprompt}
\begin{euleroutput}
        4.934802200544679 
\end{euleroutput}
\begin{eulercomment}
Ada juga operator pendek untuk mencetak hasil dalam format pecahan.
Kami sudah menggunakannya di atas.
\end{eulercomment}
\begin{eulerprompt}
>fraction 2+3/10+7/14+21/7
\end{eulerprompt}
\begin{euleroutput}
  29/5
\end{euleroutput}
\begin{eulerprompt}
>fraction 5*7/2/3*2/5
\end{eulerprompt}
\begin{euleroutput}
  7/3
\end{euleroutput}
\begin{eulercomment}
Digunakan untuk menampilkan ke bentuk pecahan sederhana.

Karena format internal menggunakan cara biner untuk menyimpan angka,
nilai 0,1 tidak akan direpresentasikan dengan tepat. Kesalahan
bertambah sedikit, sep
\end{eulercomment}
\begin{eulerprompt}
>longest 0.2+0.25+0.3+0.3+0.25+0.2+0.5-2.6
\end{eulerprompt}
\begin{euleroutput}
      -0.6000000000000001 
\end{euleroutput}
\begin{eulercomment}
Perintah ini menunjukkan presisi penuh dari operasi aritmetika yang
melibatkan angka-angka kecil, dan bagaimana kesalahan akumulatif bisa
muncul dalam perhitungan biner.

\end{eulercomment}
\eulersubheading{Expressions}
\begin{eulercomment}
String atau nama dapat digunakan untuk menyimpan ekspresi matematika,
yang dapat dievaluasi oleh EMT. Untuk ini, gunakan tanda kurung
setelah ekspresi. Jika Anda bermaksud menggunakan string sebagai
ekspresi, gunakan konvensi untuk menamakannya "fx" atau "fxy" dll.
Ekspresi lebih diutamakan daripada fungsi. Variabel global dapat
digunakan dalam evaluasi
\end{eulercomment}
\begin{eulerprompt}
>k:=5; fx:="pi*k"; fx()
\end{eulerprompt}
\begin{euleroutput}
  15.70796326794897
\end{euleroutput}
\begin{eulercomment}
Ekspresi akan selalu menggunakan global variable, bahkan jika ada
variabel dalam fungsi dengan nama yang sama.

\end{eulercomment}
\begin{eulerprompt}
>fx:="a*cos(x)"; fx(10,a=0.8)
\end{eulerprompt}
\begin{euleroutput}
  -0.671257223261162
\end{euleroutput}
\begin{eulercomment}
Menggunakan parameter yang ditetapkan ke x, y, z, dll. Jika tidak,
evaluasi ekspresi dalam fungsi dapat memberikan hasil yang
membingungkan bagi pengguna yang memanggil fungsi tersebut.

\end{eulercomment}
\begin{eulerprompt}
>at:=6; function f(expr,x,at) := expr(x); ...
>f("at*x^4",2,3) // computes 6*2^4 not 3*2^4
\end{eulerprompt}
\begin{euleroutput}
  96
\end{euleroutput}
\begin{eulercomment}
Menggunakan global variable pada fungsi, dimana "at" merupakan global
variables. Jika ingin menggunakan nilai lain untuk "at" perlu
menambahkan "at=value".

Untuk referensi, kami berkomentar bahwa koleksi panggilan (dibahas di
tempat lain) dapat berisi ekspresi. Jadi kita bisa membuat contoh di
atas sebagai berikut
\end{eulercomment}
\begin{eulerprompt}
>ut:=5; function f(expr,x,p) := expr(x,ut=p); ...
>f("ut*x^3",4,2) // computes 2*4^3 not 5*4^3
\end{eulerprompt}
\begin{euleroutput}
  128
\end{euleroutput}
\begin{eulercomment}
Walaupun "ut" sebagai global variable sudah didefinisikan, tetapi
didefinisikan kembali pada ekspresi fungsinya dimana "ut=p" sehingga
nilainya berganti dari yang awalnya ut=5 menjadi ut=p=2.

\end{eulercomment}
\begin{eulerprompt}
>f &= x^2
\end{eulerprompt}
\begin{euleroutput}
  
                                     2
                                    x
  
\end{euleroutput}
\begin{eulerprompt}
>function f(x) := x^4
>f(2)
\end{eulerprompt}
\begin{euleroutput}
  16
\end{euleroutput}
\begin{eulercomment}
Ekspresi dalam x sering digunakan seperti fungsi.\\
Mendefinisikan fungsi dengan nama yang sama seperti ekspresi simbolik
global (f \&=)menghapus nilai variabel sebelumnya untuk menghindari
kebingungan antara ekspresi simbolik dan fungsi.

Dengan cara konvensi, ekspresi simbolik atau numerik harus diberi nama
fx, fxy dll. Skema penamaan ini tidak boleh digunakan untuk fungsi.
\end{eulercomment}
\begin{eulerprompt}
>fx &= diff(x^x,x); $&fx
\end{eulerprompt}
\begin{eulerformula}
\[
x^{x}\,\left(\log x+1\right)
\]
\end{eulerformula}
\begin{eulercomment}
Bentuk khusus dari ekspresi memungkinkan variabel apa pun sebagai
parameter tanpa nama untuk evaluasi ekspresi, bukan hanya "x", "y"
dll. Untuk ini, mulai ekspresi dengan "@(variabel) ...".
\end{eulercomment}
\begin{eulerprompt}
> "@(a,b) a^3+b^2", %(2,4)
\end{eulerprompt}
\begin{euleroutput}
  @(a,b) a^3+b^2
  24
\end{euleroutput}
\begin{eulercomment}
Ini memungkinkan untuk memanipulasi ekspresi dalam variabel lain untuk
fungsi EMT yang membutuhkan ekspresi dalam "x".\\
Cara paling dasar untuk mendefinisikan fungsi sederhana adalah dengan
menyimpan rumusnya dalam ekspresi simbolis atau numerik. Jika
variabel utama adalah x, ekspresi dapat dievaluasi seperti fungsi.
Seperti yang Anda lihat dalam contoh berikut, variabel global terlihat
selama evaluasi.
\end{eulercomment}
\begin{eulerprompt}
>fx &= 2*x-3*t; ...
>t=2.5; fx(0.8)
\end{eulerprompt}
\begin{euleroutput}
  -5.9
\end{euleroutput}
\begin{eulercomment}
Semua variabel lain dalam ekspresi dapat ditentukan dalam evaluasi
menggunakan parameter yang ditetapkan.

\end{eulercomment}
\begin{eulerprompt}
>fx(1,t=1.5)
\end{eulerprompt}
\begin{euleroutput}
  -2.5
\end{euleroutput}
\begin{eulercomment}
Sebuah ekspresi tidak perlu simbolis. Ini diperlukan, jika ekspresi
berisi fungsi, yang hanya diketahui di kernel numerik, bukan di
Maxima.

\end{eulercomment}
\eulersubheading{Symbolic Mathematics}
\begin{eulercomment}
EMT melakukan matematika simbolis dengan bantuan Maxima. Untuk
detailnya, mulailah dengan tutorial berikut, atau telusuri referensi
untuk Maxima. Para ahli di Maxima harus mencatat bahwa ada perbedaan
sintaks antara sintaks asli Maxima dan sintaks default ekspresi
simbolik di EMT.\\
Matematika simbolik terintegrasi dengan mulus ke dalam Euler dengan \&.
Ekspresi apa pun yang dimulai dengan \& adalah ekspresi simbolis. Itu
dievaluasi dan dicetak oleh Maxima. Pertama-tama, Maxima memiliki
aritmatika "tak terbatas" yang dapat menangani angka yang sangat
besar.
\end{eulercomment}
\begin{eulerprompt}
>$&35!
\end{eulerprompt}
\begin{eulercomment}
Dengan cara ini, kita dapat menghitung hasil yang besar dengan tepat.

Mari kita hitung!

\end{eulercomment}
\begin{eulerformula}
\[
C(35,15)=\frac{35!}{20!\cdot15!}
\]
\end{eulerformula}
\begin{eulercomment}
\end{eulercomment}
\begin{eulerprompt}
>$& 35!/(20!*15!) // nilai C(35,15)
\end{eulerprompt}
\begin{eulerformula}
\[
3247943160
\]
\end{eulerformula}
\begin{eulercomment}
Maxima memiliki fungsi yang lebih efisien untuk ini (seperti halnya
bagian numerik dari EMT).

\end{eulercomment}
\begin{eulerprompt}
>$binomial(35,15) //menghitung C(35,15) menggunakan fungsi binomial()
\end{eulerprompt}
\begin{eulerformula}
\[
3247943160
\]
\end{eulerformula}
\begin{eulercomment}
Untuk mempelajari lebih lanjut tentang fungsi tertentu klik dua kali
di atasnya. Misalnya, coba klik dua kali pada "\&binomial" di baris
perintah sebelumnya. Ini membuka dokumentasi Maxima seperti yang
disediakan oleh penulis program itu.

Anda akan belajar bahwa yang berikut ini juga berfungsi.

\end{eulercomment}
\begin{eulerformula}
\[
C(x,3)=\frac{x!}{(x-3)!3!}=\frac{(x-2)(x-1)x}{6}
\]
\end{eulerformula}
\begin{eulercomment}
\end{eulercomment}
\begin{eulerprompt}
>$binomial(x,3) // C(x,3)
\end{eulerprompt}
\begin{eulerformula}
\[
\frac{\left(x-2\right)\,\left(x-1\right)\,x}{6}
\]
\end{eulerformula}
\begin{eulercomment}
Jika Anda ingin mengganti x dengan nilai tertentu, gunakan "with".
\end{eulercomment}
\begin{eulerprompt}
>$&binomial(x,3) with x=5 // substitusi x=5 ke C(x,3)
\end{eulerprompt}
\begin{eulerformula}
\[
10
\]
\end{eulerformula}
\begin{eulercomment}
Dengan begitu Anda dapat menggunakan solusi persamaan dalam persamaan
lain. Ekspresi simbolik dicetak oleh Maxima dalam bentuk 2D. Alasan
untuk ini adalah bendera simbolis khusus dalam string. Seperti yang
akan Anda lihat pada contoh sebelumnya dan berikut, jika Anda telah
menginstal LaTeX, Anda dapat mencetak ekspresi simbolis dengan Lateks.
Jika tidak, perintah berikut akan mengeluarkan pesan kesalahan. Untuk
mencetak ekspresi simbolis dengan LaTeX, gunakan \textdollar{} di depan \& (atau
Anda dapat menghilangkan \&) sebelum perintah. Jangan menjalankan
perintah Maxima dengan \textdollar{}, jika Anda tidak menginstal LaTeX.
\end{eulercomment}
\begin{eulerprompt}
>$(3+x)/(x^2+1)
\end{eulerprompt}
\begin{eulerformula}
\[
\frac{x+3}{x^2+1}
\]
\end{eulerformula}
\begin{eulercomment}
Ekspresi simbolik diuraikan oleh Euler. Jika Anda membutuhkan sintaks
yang kompleks dalam satu ekspresi, Anda dapat menyertakan ekspresi
dalam "...". Untuk menggunakan lebih dari ekspresi sederhana adalah
mungkin, tetapi sangat tidak disarankan
\end{eulercomment}
\begin{eulerprompt}
>&"v := 5; v^2"
\end{eulerprompt}
\begin{euleroutput}
  
                                    25
  
\end{euleroutput}
\begin{eulercomment}
Untuk kelengkapan, kami menyatakan bahwa ekspresi simbolik dapat
digunakan dalam program, tetapi perlu diapit dalam tanda kutip. Selain
itu, jauh lebih efektif untuk memanggil Maxima pada waktu kompilasi
jika memungkinkan.
\end{eulercomment}
\begin{eulerprompt}
>$&expand((1+x)^4), $&factor(diff(%,x)) // diff: turunan, factor: faktor
\end{eulerprompt}
\begin{eulerformula}
\[
4\,\left(x+1\right)^3
\]
\end{eulerformula}
\eulerimg{0}{images/Tugas Individu Pekan3-4_Muhammad Lutfi Ramadhan_23030630021-015-large.png}
\begin{eulercomment}
Sekali lagi, \% mengacu pada hasil sebelumnya.\\
Untuk mempermudah, kami menyimpan solusi ke variabel simbolik.
Variabel simbolik didefinisikan dengan "\&=".
\end{eulercomment}
\begin{eulerprompt}
>fx &= (x+1)/(x^4+1); $&fx
\end{eulerprompt}
\begin{eulerformula}
\[
\frac{x+1}{x^4+1}
\]
\end{eulerformula}
\begin{eulercomment}
Ekspresi simbolik dapat digunakan dalam ekspresi simbolik lainnya.
\end{eulercomment}
\begin{eulerprompt}
>$&factor(diff(fx,x))
\end{eulerprompt}
\begin{eulerformula}
\[
\frac{-3\,x^4-4\,x^3+1}{\left(x^4+1\right)^2}
\]
\end{eulerformula}
\begin{eulercomment}
Masukan langsung dari perintah Maxima juga tersedia. Mulai baris
perintah dengan "::". Sintaks Maxima disesuaikan dengan sintaks EMT
(disebut "compability mode").
\end{eulercomment}
\begin{eulerprompt}
>&factor(20!)
\end{eulerprompt}
\begin{euleroutput}
  
                           2432902008176640000
  
\end{euleroutput}
\begin{eulerprompt}
>::: factor(10!)
\end{eulerprompt}
\begin{euleroutput}
  
                                 8  4  2
                                2  3  5  7
  
\end{euleroutput}
\begin{eulerprompt}
>:: factor(20!)
\end{eulerprompt}
\begin{euleroutput}
  
                          18  8  4  2
                         2   3  5  7  11 13 17 19
  
\end{euleroutput}
\begin{eulercomment}
Jika Anda ahli dalam Maxima, Anda mungkin ingin menggunakan sintaks
asli Maxima. Anda dapat melakukannya dengan ":::".
\end{eulercomment}
\begin{eulerprompt}
>::: av:g$ av^2;
\end{eulerprompt}
\begin{euleroutput}
  
                                     2
                                    g
  
\end{euleroutput}
\begin{eulerprompt}
>fx &= x^3*exp(x), $fx
\end{eulerprompt}
\begin{euleroutput}
  
                                   3  x
                                  x  E
  
\end{euleroutput}
\begin{eulerformula}
\[
x^3\,e^{x}
\]
\end{eulerformula}
\begin{eulercomment}
Variabel tersebut dapat digunakan dalam ekspresi simbolik lainnya.
Perhatikan, bahwa dalam perintah berikut sisi kanan \&= dievaluasi
sebelum penugasan ke Fx.
\end{eulercomment}
\begin{eulerprompt}
>&(fx with x=5), $%, &float(%)
\end{eulerprompt}
\begin{euleroutput}
  
                                       5
                                  125 E
  
\end{euleroutput}
\begin{eulerformula}
\[
125\,e^5
\]
\end{eulerformula}
\begin{euleroutput}
  
                            18551.64488782208
  
\end{euleroutput}
\begin{eulerprompt}
>fx(5)
\end{eulerprompt}
\begin{euleroutput}
  18551.6448878
\end{euleroutput}
\begin{eulercomment}
Untuk evaluasi ekspresi dengan nilai variabel tertentu, Anda dapat
menggunakan operator "with". Baris perintah berikut juga menunjukkan
bahwa Maxima dapat mengevaluasi ekspresi secara numerik dengan
float().
\end{eulercomment}
\begin{eulerprompt}
>&(fx with x=10)-(fx with x=5), &float(%)
\end{eulerprompt}
\begin{euleroutput}
  
                                  10        5
                            1000 E   - 125 E
  
  
                           2.20079141499189e+7
  
\end{euleroutput}
\begin{eulerprompt}
>$factor(diff(fx,x,2))
\end{eulerprompt}
\begin{eulerformula}
\[
x\,\left(x^2+6\,x+6\right)\,e^{x}
\]
\end{eulerformula}
\begin{eulercomment}
Untuk mendapatkan kode Lateks untuk ekspresi, Anda dapat menggunakan
perintah "tex".
\end{eulercomment}
\begin{eulerprompt}
>tex(fx)
\end{eulerprompt}
\begin{euleroutput}
  x^3\(\backslash\),e^\{x\}
\end{euleroutput}
\begin{eulercomment}
Ekspresi simbolik dapat dievaluasi seperti ekspresi numerik.
\end{eulercomment}
\begin{eulerprompt}
>fx(0.5)
\end{eulerprompt}
\begin{euleroutput}
  0.206090158838
\end{euleroutput}
\begin{eulercomment}
Dalam ekspresi simbolis, ini tidak berfungsi, karena Maxima tidak
mendukungnya. Sebagai gantinya, gunakan sintaks "with" (bentuk yang
lebih bagus dari perintah at(...) dari Maxima.
\end{eulercomment}
\begin{eulerprompt}
>$&fx with x=1/2
\end{eulerprompt}
\begin{eulerformula}
\[
\frac{\sqrt{e}}{8}
\]
\end{eulerformula}
\begin{eulercomment}
Penugasan juga bisa bersifat simbolis.
\end{eulercomment}
\begin{eulerprompt}
>$&fx with x=1+t
\end{eulerprompt}
\begin{eulerformula}
\[
\left(t+1\right)^3\,e^{t+1}
\]
\end{eulerformula}
\begin{eulercomment}
Perintah solve memecahkan ekspresi simbolik untuk variabel di Maxima.
Hasilnya adalah vektor solusi.
\end{eulercomment}
\begin{eulerprompt}
>$&solve(x^2+x=4,x)
\end{eulerprompt}
\begin{eulerformula}
\[
\left[ x=\frac{-\sqrt{17}-1}{2} , x=\frac{\sqrt{17}-1}{2} \right] 
\]
\end{eulerformula}
\begin{eulercomment}
Bandingkan dengan perintah numerik "selesaikan" di Euler, yang
membutuhkan nilai awal, dan secara opsional nilai target.
\end{eulercomment}
\begin{eulerprompt}
>solve("x^2+x",1,y=4)
\end{eulerprompt}
\begin{euleroutput}
  1.56155281281
\end{euleroutput}
\begin{eulercomment}
Nilai numerik dari solusi simbolik dapat dihitung dengan evaluasi
hasil simbolis. Euler akan membaca tugas x= dll. Jika Anda tidak
memerlukan hasil numerik untuk perhitungan lebih lanjut, Anda juga
dapat membiarkan Maxima menemukan nilai numerik.
\end{eulercomment}
\begin{eulerprompt}
>sol &= solve(x^2+3*x=9,x); $&sol, sol(), $&float(sol)
\end{eulerprompt}
\begin{eulerformula}
\[
\left[ x=\frac{-3\,\sqrt{5}-3}{2} , x=\frac{3\,\sqrt{5}-3}{2}   \right] 
\]
\end{eulerformula}
\begin{euleroutput}
  [-4.8541,  1.8541]
\end{euleroutput}
\begin{eulerformula}
\[
\left[ x=-4.854101966249685 , x=1.854101966249685 \right] 
\]
\end{eulerformula}
\begin{eulercomment}
Untuk mencetak ekspresi simbolis dengan LaTeX, gunakan \textdollar{} di depan \&
(atau dapat menghilangkan \&) sebelum perintah. Jangan menjalankan
perintah Maxima dengan \textdollar{}, jika tidak menginstal LaTeX.

Untuk mendapatkan solusi simbolis tertentu, seseorang dapat\\
menggunakan "with" dan index.
\end{eulercomment}
\begin{eulerprompt}
>$&solve(x^2+3*x=1,x), x2 &= x with %[2]; $&x2
\end{eulerprompt}
\begin{eulerformula}
\[
\frac{\sqrt{13}-3}{2}
\]
\end{eulerformula}
\eulerimg{1}{images/Tugas Individu Pekan3-4_Muhammad Lutfi Ramadhan_23030630021-027-large.png}
\begin{eulercomment}
Untuk menyelesaikan sistem persamaan,gunakan vektor persamaan.
Hasilnya adalah vektor solusi.
\end{eulercomment}
\begin{eulerprompt}
>sol &= solve([x+y=5,x^2+y^2=15],[x,y]); $&sol, $&x*y with sol[1]
\end{eulerprompt}
\begin{eulerformula}
\[
\frac{\left(5-\sqrt{5}\right)\,\left(\sqrt{5}+5\right)}{4}
\]
\end{eulerformula}
\eulerimg{1}{images/Tugas Individu Pekan3-4_Muhammad Lutfi Ramadhan_23030630021-029-large.png}
\begin{eulercomment}
Ekspresi simbolis dapat memiliki bendera, yang menunjukkan perlakuan
khusus di Maxima. Beberapa flag dapat digunakan sebagai perintah juga,
yang lain tidak. Bendera ditambahkan dengan "\textbar{}" (bentuk yang lebih
bagus dari "ev(...,flags)").
\end{eulercomment}
\begin{eulerprompt}
>$& diff((x^3-1)/(x+1),x) //turunan bentuk pecahan
\end{eulerprompt}
\begin{eulerformula}
\[
\frac{3\,x^2}{x+1}-\frac{x^3-1}{\left(x+1\right)^2}
\]
\end{eulerformula}
\begin{eulerprompt}
>$& diff((x^3-1)/(x+1),x) | ratsimp //menyederhanakan pecahan
\end{eulerprompt}
\begin{eulerformula}
\[
\frac{2\,x^3+3\,x^2+1}{x^2+2\,x+1}
\]
\end{eulerformula}
\begin{eulerprompt}
>$&factor(%)
\end{eulerprompt}
\begin{eulerformula}
\[
\frac{2\,x^3+3\,x^2+1}{\left(x+1\right)^2}
\]
\end{eulerformula}
\eulersubheading{Functions}
\begin{eulercomment}
Dalam EMT, fungsi adalah program yang didefinisikan dengan perintah
"fungsi". Ini bisa berupa fungsi satu baris atau fungsi multibaris.\\
Fungsi satu baris dapat berupa numerik atau simbolis. Fungsi satu
baris numerik didefinisikan oleh ":=".
\end{eulercomment}
\begin{eulerprompt}
>function f(x) := x*sqrt(x^2+1)
\end{eulerprompt}
\begin{eulercomment}
Untuk gambaran umum, kami menunjukkan semua kemungkinan definisi untuk
fungsi satu baris. Suatu fungsi dapat dievaluasi sama seperti fungsi
Euler bawaan lainnya.
\end{eulercomment}
\begin{eulerprompt}
>f(2)
\end{eulerprompt}
\begin{euleroutput}
  4.472135955
\end{euleroutput}
\begin{eulercomment}
Fungsi ini akan bekerja untuk vektor juga, dengan mematuhi bahasa
matriks Euler, karena ekspresi yang digunakan dalam fungsi
divektorkan.
\end{eulercomment}
\begin{eulerprompt}
>f(0:0.1:1)
\end{eulerprompt}
\begin{euleroutput}
  [0,  0.100499,  0.203961,  0.313209,  0.430813,  0.559017,  0.699714,
  0.854459,  1.0245,  1.21083,  1.41421]
\end{euleroutput}
\begin{eulercomment}
Fungsi dapat diplot. Alih-alih ekspresi, kita hanya perlu memberikan
nama fungsi.\\
Berbeda dengan ekspresi simbolik atau numerik, nama fungsi harus
diberikan dalam string.
\end{eulercomment}
\begin{eulerprompt}
>solve("f",1,y=1)
\end{eulerprompt}
\begin{euleroutput}
  0.786151377757
\end{euleroutput}
\begin{eulercomment}
Secara default, jika Anda perlu menimpa fungsi bawaan, Anda harus
menambahkan kata kunci "overwrite". Overwriting fungsi bawaan
berbahaya dan dapat menyebabkan masalah untuk fungsi lain tergantung
pada fungsi tersebut.\\
Anda masih dapat memanggil fungsi bawaan sebagai "\_...", jika itu
adalah fungsi di inti Euler.
\end{eulercomment}
\begin{eulerprompt}
>function overwrite sin(x) := _sin(x°) // tentukan kembai sinus dalam derajat
>sin(45)
\end{eulerprompt}
\begin{euleroutput}
  0.707106781187
\end{euleroutput}
\begin{eulercomment}
Jika ingin menghapus definisi dari sin dan mendefinisikannya ulang,
menggunakan perintah "forget"
\end{eulercomment}
\begin{eulerprompt}
>forget sin; sin(pi/4)
\end{eulerprompt}
\begin{euleroutput}
  0.707106781187
\end{euleroutput}
\eulersubheading{Default Parameters}
\begin{eulercomment}
Parameter default adalah fungsi parameter yang memiliki nilai awal.\\
Fungsi numerik dapat memiliki parameter default.
\end{eulercomment}
\begin{eulerprompt}
>function f(x,a=1) := a*x^2
\end{eulerprompt}
\begin{eulercomment}
Menghilangkan parameter ini menggunakan nilai default.
\end{eulercomment}
\begin{eulerprompt}
>f(4)
\end{eulerprompt}
\begin{euleroutput}
  16
\end{euleroutput}
\begin{eulercomment}
Menimpa default value.
\end{eulercomment}
\begin{eulerprompt}
>f(4,5)
\end{eulerprompt}
\begin{euleroutput}
  80
\end{euleroutput}
\begin{eulercomment}
Parameter yang ditetapkan menimpanya juga. Ini digunakan oleh banyak
fungsi Euler seperti plot2d, plot3d.
\end{eulercomment}
\begin{eulerprompt}
>f(4,a=1)
\end{eulerprompt}
\begin{euleroutput}
  16
\end{euleroutput}
\begin{eulercomment}
Jika suatu variabel bukan parameter, itu pasti global. Fungsi satu
baris dapat melihat variabel global.
\end{eulercomment}
\begin{eulerprompt}
>function f(x):= a*x^2
>a=6; f(2)
\end{eulerprompt}
\begin{euleroutput}
  24
\end{euleroutput}
\begin{eulercomment}
Tetapi parameter yang ditetapkan menimpa global value.\\
Jika argumen tidak ada dalam daftar parameter yang telah ditentukan
sebelumnya, argumen tersebut harus dideklarasikan dengan ":="!
\end{eulercomment}
\begin{eulerprompt}
>f(2,a:=5)
\end{eulerprompt}
\begin{euleroutput}
  20
\end{euleroutput}
\begin{eulercomment}
Fungsi simbolis didefinisikan dengan "\textdollar{}\&=". Fungsi simbolis
didefinisikan dalam Euler dan maxima,dan bekerja keduanya. Ekspresi
yang mendefinisikan dijalankan melalui Maxima sebelum di definisi.
\end{eulercomment}
\begin{eulerprompt}
>function g(x) &= x^3-x*exp(-x); $&g(x)
\end{eulerprompt}
\begin{eulerformula}
\[
x^3-x\,e^ {- x }
\]
\end{eulerformula}
\begin{eulercomment}
Fungsi simbolik dapat digunakan dalam ekspresi simbolik
\end{eulercomment}
\begin{eulerprompt}
>$&diff(g(x),x), $&% with x=4/3 //1. turunan pertama dari g(x), 2. memasukkan nilai x=4/3
\end{eulerprompt}
\begin{eulerformula}
\[
\frac{e^ {- \frac{4}{3} }}{3}+\frac{16}{3}
\]
\end{eulerformula}
\eulerimg{1}{images/Tugas Individu Pekan3-4_Muhammad Lutfi Ramadhan_23030630021-035-large.png}
\begin{eulercomment}
Itu juga dapat digunakan dalam ekspresi numerik. Tentu saja, ini hanya
akan berfungsi jika EMT dapat mengintrepertasikan semua yang ada di
dalam fungsi tersebut.
\end{eulercomment}
\begin{eulerprompt}
>g(5+g(1))
\end{eulerprompt}
\begin{euleroutput}
  178.635099908
\end{euleroutput}
\begin{eulercomment}
Itu juga dapat digunakan untuk mendefinisikan fungsi atau ekspresi
simbolik lainnya.
\end{eulercomment}
\begin{eulerprompt}
>function G(x) &= factor(integrate(g(x),x)); $&G(c) //integrate : mengintegralkan
\end{eulerprompt}
\begin{eulerformula}
\[
\frac{e^ {- c }\,\left(c^4\,e^{c}+4\,c+4\right)}{4}
\]
\end{eulerformula}
\begin{eulerprompt}
>solve(&g(x),0.5)
\end{eulerprompt}
\begin{euleroutput}
  0.703467422498
\end{euleroutput}
\begin{eulercomment}
Berikut ini juga berfungsi, karena Euler menggunakan ekspresi simbolis
dalam fungsi g, jika tidak menemukan variabel simbolik g, dan jika ada
fungsi simbolis g.
\end{eulercomment}
\begin{eulerprompt}
>solve(&g,0.5)
\end{eulerprompt}
\begin{euleroutput}
  0.703467422498
\end{euleroutput}
\begin{eulercomment}
Dengan \&= fungsinya simbolis, dan dapat digunakan dalam ekspresi
simbolik lainnya. Contohnya dalam integral tak tentu sebagai berikut.
\end{eulercomment}
\begin{eulerprompt}
>function P(x,n) &= (2*x-1)^n; $&P(x,n)
\end{eulerprompt}
\begin{eulerformula}
\[
\left(2\,x-1\right)^{n}
\]
\end{eulerformula}
\begin{eulerprompt}
>function Q(x,n) &= (x+2)^n; $&Q(x,n)
\end{eulerprompt}
\begin{eulerformula}
\[
\left(x+2\right)^{n}
\]
\end{eulerformula}
\begin{eulerprompt}
>$&P(x,4), $&expand(%)
\end{eulerprompt}
\begin{eulerformula}
\[
16\,x^4-32\,x^3+24\,x^2-8\,x+1
\]
\end{eulerformula}
\eulerimg{0}{images/Tugas Individu Pekan3-4_Muhammad Lutfi Ramadhan_23030630021-040-large.png}
\begin{eulerprompt}
>P(3,4)
\end{eulerprompt}
\begin{euleroutput}
  625
\end{euleroutput}
\begin{eulerprompt}
>$&P(x,4)+Q(x,3), $&expand(%)
\end{eulerprompt}
\begin{eulerformula}
\[
16\,x^4-31\,x^3+30\,x^2+4\,x+9
\]
\end{eulerformula}
\eulerimg{0}{images/Tugas Individu Pekan3-4_Muhammad Lutfi Ramadhan_23030630021-042-large.png}
\begin{eulerprompt}
>$&P(x,4)-Q(x,3), $&expand(%), $&factor(%)
\end{eulerprompt}
\begin{eulerformula}
\[
16\,x^4-33\,x^3+18\,x^2-20\,x-7
\]
\end{eulerformula}
\eulerimg{0}{images/Tugas Individu Pekan3-4_Muhammad Lutfi Ramadhan_23030630021-044-large.png}
\eulerimg{0}{images/Tugas Individu Pekan3-4_Muhammad Lutfi Ramadhan_23030630021-045-large.png}
\begin{eulerprompt}
>$&P(x,4)*Q(x,3), $&expand(%), $&factor(%)
\end{eulerprompt}
\begin{eulerformula}
\[
\left(x+2\right)^3\,\left(2\,x-1\right)^4
\]
\end{eulerformula}
\eulerimg{0}{images/Tugas Individu Pekan3-4_Muhammad Lutfi Ramadhan_23030630021-047-large.png}
\eulerimg{0}{images/Tugas Individu Pekan3-4_Muhammad Lutfi Ramadhan_23030630021-048-large.png}
\begin{eulerprompt}
>$&P(x,4)/Q(x,1), $&expand(%), $&factor(%)
\end{eulerprompt}
\begin{eulerformula}
\[
\frac{\left(2\,x-1\right)^4}{x+2}
\]
\end{eulerformula}
\eulerimg{1}{images/Tugas Individu Pekan3-4_Muhammad Lutfi Ramadhan_23030630021-050-large.png}
\eulerimg{1}{images/Tugas Individu Pekan3-4_Muhammad Lutfi Ramadhan_23030630021-051-large.png}
\begin{eulerprompt}
>function f(x) &= x^3-2; $&f(x)
\end{eulerprompt}
\begin{eulerformula}
\[
x^3-2
\]
\end{eulerformula}
\begin{eulercomment}
Dengan \&= maka fungsi adalah simbolik, dan dapat digunakan di ekpresi
simbolik lainnya.
\end{eulercomment}
\begin{eulerprompt}
>$&integrate(f(x),x)
\end{eulerprompt}
\begin{eulerformula}
\[
\frac{x^4}{4}-2\,x
\]
\end{eulerformula}
\begin{eulercomment}
Dengan := fungsinya numerik. Contoh yang baik adalah integral tak
tentu seperti\\
\end{eulercomment}
\begin{eulerformula}
\[
f(x) = \int_1^x t^t \, dt,
\]
\end{eulerformula}
\begin{eulercomment}
yang tidak dapat dinilai secara simbolis.\\
Jika kita mendefinisikan kembali fungsi dengan kata kunci "map" dapat
digunakan untuk vektor x. Secara internal, fungsi dipanggil untuk
semua nilai x satu kali, dan hasilnya disimpan dalam vektor.
\end{eulercomment}
\begin{eulerprompt}
>function map f(x) := integrate ("x^x",1,x)
>f(0:0.5:2)
\end{eulerprompt}
\begin{euleroutput}
  [-0.783431,  -0.410816,  0,  0.676863,  2.05045]
\end{euleroutput}
\begin{eulercomment}
Fungsi dapat memiliki nilai default untuk parameter.
\end{eulercomment}
\begin{eulerprompt}
>function mylog (x,base=10) := ln(x)/ln(base);
\end{eulerprompt}
\begin{eulercomment}
Sekarang fungsi dapat dipanggil dengan menggunakan suatau parameter
"base" maupun tidak.
\end{eulercomment}
\begin{eulerprompt}
>mylog(100), mylog(2^6.7,2)
\end{eulerprompt}
\begin{euleroutput}
  2
  6.7
\end{euleroutput}
\begin{eulercomment}
Selain itu, dimungkinkan untuk menggunakan parameter yang ditetapkan.
\end{eulercomment}
\begin{eulerprompt}
>mylog(E^2,base=E)
\end{eulerprompt}
\begin{euleroutput}
  2
\end{euleroutput}
\begin{eulercomment}
Sering kali, kita ingin menggunakan fungsi untuk vektor di satu
tempat, dan untuk elemen individual di tempat lain. Ini tapat terjadi
dengan vektor parameter.
\end{eulercomment}
\begin{eulerprompt}
>function f([a,b]) &= a^2+b^2-a*b+b; $&f(a,b), $&f(x,y)
\end{eulerprompt}
\begin{eulerformula}
\[
y^2-x\,y+y+x^2
\]
\end{eulerformula}
\eulerimg{0}{images/Tugas Individu Pekan3-4_Muhammad Lutfi Ramadhan_23030630021-056-large.png}
\begin{eulercomment}
Fungsi simbolik seperti itu dapat digunakan untuk variabel simbolik.\\
etapi fungsi ini juga dapat digunakan untuk vektor numerik.
\end{eulercomment}
\begin{eulerprompt}
>v=[3,4]; f(v)
\end{eulerprompt}
\begin{euleroutput}
  17
\end{euleroutput}
\begin{eulercomment}
Ada juga fungsi yang murni simbolis, yang tidak dapat digunakan secara
numerik.
\end{eulercomment}
\begin{eulerprompt}
>function lapl(expr,x,y) &&= diff(expr,x,2)+diff(expr,y,2)//turunan parsial kedua
\end{eulerprompt}
\begin{euleroutput}
  
                   diff(expr, y, 2) + diff(expr, x, 2)
  
\end{euleroutput}
\begin{eulerprompt}
>$&realpart((x+I*y)^4), $&lapl(%,x,y)
\end{eulerprompt}
\begin{eulerformula}
\[
0
\]
\end{eulerformula}
\eulerimg{0}{images/Tugas Individu Pekan3-4_Muhammad Lutfi Ramadhan_23030630021-058-large.png}
\begin{eulercomment}
Tetapi tentu saja, mereka dapat digunakan dalam ekspresi simbolis atau
dalam definisi fungsi simbolis.
\end{eulercomment}
\begin{eulerprompt}
>function f(x,y) &= factor(lapl((x+y^2)^5,x,y)); $&f(x,y)
\end{eulerprompt}
\begin{eulerformula}
\[
10\,\left(y^2+x\right)^3\,\left(9\,y^2+x+2\right)
\]
\end{eulerformula}
\begin{eulercomment}
Ringkasam:\\
- \&= mendefinisikan fungsi simbolis,\\
- := mendefinisikan fungsi numerik,\\
- \&\&= mendefinisikan fungsi simbolis murni

\end{eulercomment}
\eulersubheading{Memecahkan Ekspresi }
\begin{eulercomment}
Ekspresi dapat diselesaikan secara numerik dan simbolis.\\
Untuk menyelesaikan ekspresi sederhana dari satu variabel, kita dapat
menggunakan fungsi solve(). Perlu\\
nilai awal untuk memulai pencarian. Secara internal, solve()
menggunakan metode secant.
\end{eulercomment}
\begin{eulerprompt}
>solve("x^2-2",1)
\end{eulerprompt}
\begin{euleroutput}
  1.41421356237
\end{euleroutput}
\begin{eulercomment}
Ini juga berfungsi untuk fungsi simbolik, perhatikan fungsi berikut
ini.
\end{eulercomment}
\begin{eulerprompt}
>$&solve(x^2=2,x)
\end{eulerprompt}
\begin{eulerformula}
\[
\left[ x=-\sqrt{2} , x=\sqrt{2} \right] 
\]
\end{eulerformula}
\begin{eulerprompt}
>$&solve(x^2-2,x)
\end{eulerprompt}
\begin{eulerformula}
\[
\left[ x=-\sqrt{2} , x=\sqrt{2} \right] 
\]
\end{eulerformula}
\begin{eulerprompt}
>$&solve(a*x^2+b*x+c=0,x)
\end{eulerprompt}
\begin{eulerformula}
\[
\left[ x=\frac{-\sqrt{b^2-4\,a\,c}-b}{2\,a} , x=\frac{\sqrt{b^2-4\,  a\,c}-b}{2\,a} \right] 
\]
\end{eulerformula}
\begin{eulerprompt}
>$&solve([a*x+b*y=c,d*x+e*y=f],[x,y])
\end{eulerprompt}
\begin{eulerformula}
\[
\left[ \left[ x=\frac{b\,f-c\,e}{b\,d-a\,e} , y=\frac{c\,d-a\,f}{b  \,d-a\,e} \right]  \right] 
\]
\end{eulerformula}
\begin{eulercomment}
Sekarang kita mencari titik, di mana polinomialnya adalah 2. Dalam
solve(), nilai target default y=0 dapat\\
diubah dengan variabel yang ditetapkan.\\
Kami menggunakan y=2 dan memeriksa dengan mengevaluasi polinomial pada
hasil sebelumnya.
\end{eulercomment}
\begin{eulerprompt}
>px &= 4*x^8+x^7-x^4-x; $&px
\end{eulerprompt}
\begin{eulerformula}
\[
4\,x^8+x^7-x^4-x
\]
\end{eulerformula}
\begin{eulerprompt}
>solve(px,1,y=2), px(%)
\end{eulerprompt}
\begin{euleroutput}
  0.966715594851
  2
\end{euleroutput}
\begin{eulercomment}
Memecahkan ekspresi simbolis dalam bentuk simbolis mengembalikan
daftar solusi. Kami menggunakan\\
pemecah simbolik solve() yang disediakan oleh Maxima.
\end{eulercomment}
\begin{eulerprompt}
>sol &= solve(x^2-x-1,x); $&sol
\end{eulerprompt}
\begin{eulerformula}
\[
\left[ x=\frac{1-\sqrt{5}}{2} , x=\frac{\sqrt{5}+1}{2} \right] 
\]
\end{eulerformula}
\begin{eulercomment}
Cara termudah untuk mendapatkan nilai numerik adalah dengan
mengevaluasi solusi secara numerik seperti\\
ekspresi.
\end{eulercomment}
\begin{eulerprompt}
>longest sol()
\end{eulerprompt}
\begin{euleroutput}
      -0.6180339887498949       1.618033988749895 
\end{euleroutput}
\begin{eulercomment}
Untuk menggunakan solusi secara simbolis dalam ekspresi lain, cara
termudah adalah "with".
\end{eulercomment}
\begin{eulerprompt}
>$&x^2 with sol[1], $&expand(x^2-x-1 with sol[2])
\end{eulerprompt}
\begin{eulerformula}
\[
0
\]
\end{eulerformula}
\eulerimg{0}{images/Tugas Individu Pekan3-4_Muhammad Lutfi Ramadhan_23030630021-067-large.png}
\begin{eulercomment}
Memecahkan sistem persamaan secara simbolis dapat dilakukan dengan
vektor persamaan dan solver simbolis solve(). Hadilnya dalam bentuk
persamaan.
\end{eulercomment}
\begin{eulerprompt}
>$&solve([x+y=2,x^3+2*y+x=4],[x,y])
\end{eulerprompt}
\begin{eulerformula}
\[
\left[ \left[ x=-1 , y=3 \right]  , \left[ x=1 , y=1 \right]  ,   \left[ x=0 , y=2 \right]  \right] 
\]
\end{eulerformula}
\begin{eulercomment}
Fungsi f() dapat melihat variabel global. Namun seringkali kita ingin
menggunakan parameter lokal.

\end{eulercomment}
\begin{eulerformula}
\[
a^x-x^a = 0.1
\]
\end{eulerformula}
\begin{eulercomment}
dengan a=3.
\end{eulercomment}
\begin{eulerprompt}
>function f(x,a) := x^a+a^x;
\end{eulerprompt}
\begin{eulercomment}
Salah satu cara untuk mengoper parameter tambahan ke f() adalah dengan
menggunakan sebuah daftar dengan nama fungsi dan parameternya (cara-\\
ara lainnya adalah parameter titik koma).
\end{eulercomment}
\begin{eulerprompt}
>solve(\{\{"f",3\}\},2,y=0.1)
\end{eulerprompt}
\begin{euleroutput}
  -0.710242150858
\end{euleroutput}
\begin{eulercomment}
Ini juga bekerja dengan ekspresi. Tapi daftar elemen yang ada harus
digunakan.
\end{eulercomment}
\begin{eulerprompt}
>solve(\{\{"x^2+a*x",a=3\}\},2,y=0.1)
\end{eulerprompt}
\begin{euleroutput}
  0.0329709716756
\end{euleroutput}
\begin{eulercomment}
\begin{eulercomment}
\eulerheading{Menyelesaikan Pertidaksamaan}
\begin{eulercomment}
Untuk menyelesaikan pertidaksamaan, EMT tidak akan dapat melakukannya,
melainkan dengan bantuan Maxima, artinya secara eksak (simbolik).
Perintah Maxima yang digunakan adalah fourier\_elim(), yang harus
dipanggil dengan perintah "load(fourier\_elim)" terlebih dahulu.

Eliminasi Fourier adalah analog dari eliminasi Gauss untuk linear
(persamaan atau pertidaksamaan). Panggilan fungsi `fourier\_elim([eq1,
eq2, ...], [var1, var2, ...])' melakukan eliminasi Fourier eliminasi
pada pertidaksamaan linear `[eq1, eq2, ...]' dengan berkenaan dengan
variabel `[var1, var2, ...]'; sebagai contoh
\end{eulercomment}
\begin{eulerprompt}
>&load(fourier_elim)
\end{eulerprompt}
\begin{euleroutput}
  
          C:/Program Files/Euler x64/maxima/share/maxima/5.35.1/share/f\(\backslash\)
  ourier_elim/fourier_elim.lisp
  
\end{euleroutput}
\begin{eulerprompt}
>$&fourier_elim([y-x < 5, x - y < 7, 10 < y],[x,y])
\end{eulerprompt}
\begin{eulerformula}
\[
\left[ y-5<x , x<y+7 , 10<y \right] 
\]
\end{eulerformula}
\begin{eulerprompt}
>$&fourier_elim([x^2 - 1>0],[x])
\end{eulerprompt}
\begin{eulerformula}
\[
\left[ 1<x \right] \lor \left[ x<-1 \right] 
\]
\end{eulerformula}
\begin{eulerprompt}
>$&fourier_elim([x^2 - 4<0],[x])
\end{eulerprompt}
\begin{eulerformula}
\[
\left[ -2<x , x<2 \right] 
\]
\end{eulerformula}
\begin{eulerprompt}
>$&fourier_elim([x^2 - 9# 0],[x])
\end{eulerprompt}
\begin{eulerformula}
\[
\left[ -3<x , x<3 \right] \lor \left[ 3<x \right] \lor \left[ x<-3   \right] 
\]
\end{eulerformula}
\begin{eulerprompt}
>$&fourier_elim([x # 10],[x])
\end{eulerprompt}
\begin{eulerformula}
\[
\left[ x<10 \right] \lor \left[ 10<x \right] 
\]
\end{eulerformula}
\begin{eulercomment}
Ketika himpunan penyelesaiannya adalah kosong maka `emptyset', dan
ketika himpunan penyelesaiannya adalah semua bilangan real, maka
'universalset'; sebagai contoh
\end{eulercomment}
\begin{eulerprompt}
>$&fourier_elim([minf < x, x < inf],[x])
\end{eulerprompt}
\begin{eulerformula}
\[
{\it universalset}
\]
\end{eulerformula}
\begin{eulerprompt}
>$&fourier_elim([x < 1, x > 1],[x])
\end{eulerprompt}
\begin{eulerformula}
\[
{\it emptyset}
\]
\end{eulerformula}
\begin{eulercomment}
Untuk persamaan nonlinier, `fourier\_elim' mengembalikan sebuah daftar
persamaan yang disederhanakan:
\end{eulercomment}
\begin{eulerprompt}
>$&fourier_elim([x^3 - 8 > 0],[x])
\end{eulerprompt}
\begin{eulerformula}
\[
\left[ 2<x , x^2+2\,x+4>0 \right] \lor \left[ x<2 , -x^2-2\,x-4>0   \right] 
\]
\end{eulerformula}
\begin{eulerprompt}
>$&fourier_elim([cos(x) < 1/2],[x])
\end{eulerprompt}
\begin{eulerformula}
\[
\left[ 1-2\,\cos x>0 \right] 
\]
\end{eulerformula}
\begin{eulercomment}
Alih-alih sebuah daftar pertidaksamaan,`fourier\_elim' juga dapat
berupa disjungsi atau konjungsi logika:
\end{eulercomment}
\begin{eulerprompt}
>$&fourier_elim((x + y < 5) and (x - y >8),[x,y])
\end{eulerprompt}
\begin{eulerformula}
\[
\left[ y+8<x , x<5-y , y<-\frac{3}{2} \right] 
\]
\end{eulerformula}
\begin{eulerprompt}
>$&fourier_elim([y-x < 5, x - y < 7, 10 < y],[x,y])
\end{eulerprompt}
\begin{eulerformula}
\[
\left[ y-5<x , x<y+7 , 10<y \right] 
\]
\end{eulerformula}
\begin{eulerprompt}
>$&fourier_elim(((x + y < 5) and x < 1) or  (x - y >8),[x,y])
\end{eulerprompt}
\begin{eulerformula}
\[
\left[ y+8<x \right] \lor \left[ x<{\it min}\left(1 , 5-y\right)   \right] 
\]
\end{eulerformula}
\begin{eulercomment}
Fungsi `fourier\_elim' mendukung operator pertidaksamaan `\textless{},\textless{}=, \textgreater{},
\textgreater{}=,#', dan `='.\\
Kode eliminasi Fourier memiliki sebuah preprocessor yang mengubah
beberapa persamaan nonlinier yang melibatkan nilai absolut,
minimum,dan fungsi maksimum menjadi linear dalam persamaan. Selain
itu,preprocessor menangani beberapa ekspresi yang merupakan hasil kali
atau hasil bagi dari suku-suku linier:
\end{eulercomment}
\begin{eulerprompt}
>$&fourier_elim([max(x,y) > 6, x # 8, abs(y-1) > 12],[x,y])
\end{eulerprompt}
\begin{eulerformula}
\[
\left[ 6<x , x<8 , y<-11 \right] \lor \left[ 8<x , y<-11 \right]   \lor \left[ x<8 , 13<y \right] \lor \left[ x=y , 13<y \right] \lor   \left[ 8<x , x<y , 13<y \right] \lor \left[ y<x , 13<y \right] 
\]
\end{eulerformula}
\begin{eulerprompt}
>$&fourier_elim([(x+2)/(x-4) <= 2],[x])
\end{eulerprompt}
\begin{eulerformula}
\[
\left[ x=10 \right] \lor \left[ 10<x \right] \lor \left[ x<4   \right] 
\]
\end{eulerformula}
\eulerheading{Bahasa Matriks}
\begin{eulercomment}
Dalam matematika, matriks adalah susunan[1] bilangan, simbol, atau
ekspresi yang disusun dalam baris dan kolom sehingga membentuk suatu
bangun persegi\\
Vektor dan matriks dimasukkan dengan tanda kurung siku, elemen
dipisahkan dengan koma, baris dipisahkan dengan titik koma.

Matriks 1x2
\end{eulercomment}
\begin{eulerprompt}
>a=[1;2]
\end{eulerprompt}
\begin{euleroutput}
              1 
              2 
\end{euleroutput}
\begin{eulerprompt}
>b=[3,4;5,6]
\end{eulerprompt}
\begin{euleroutput}
              3             4 
              5             6 
\end{euleroutput}
\begin{eulerprompt}
>c=[1,2,3;4,5,6;7,8,9]
\end{eulerprompt}
\begin{euleroutput}
              1             2             3 
              4             5             6 
              7             8             9 
\end{euleroutput}
\begin{eulercomment}
Transpose matriks adalah matriks baru yang diperoleh dengan cara
menukar elemen-elemen baris menjadi elemen kolom atau sebaliknya.
\end{eulercomment}
\begin{eulerprompt}
>a'
\end{eulerprompt}
\begin{euleroutput}
  [1,  2]
\end{euleroutput}
\begin{eulerprompt}
>b'
\end{eulerprompt}
\begin{euleroutput}
              3             5 
              4             6 
\end{euleroutput}
\begin{eulerprompt}
>c'
\end{eulerprompt}
\begin{euleroutput}
              1             4             7 
              2             5             8 
              3             6             9 
\end{euleroutput}
\begin{eulercomment}
Invers matriks adalah matriks baru yang merupakan kebalikan dari
matriks asal
\end{eulercomment}
\begin{eulerprompt}
>inv(b)
\end{eulerprompt}
\begin{euleroutput}
             -3             2 
            2.5          -1.5 
\end{euleroutput}
\begin{eulercomment}
Perkalian matriks sendiri adalah proses mengalikan setiap elemen baris
pada matriks pertama dengan elemen kolom pada matriks kedua.
\end{eulercomment}
\begin{eulerprompt}
>b.a
\end{eulerprompt}
\begin{euleroutput}
             11 
             17 
\end{euleroutput}
\begin{eulercomment}
Perkalian dari matriks dengan invers matriks itu sendiri akan
menghasilkan matriks identitas
\end{eulercomment}
\begin{eulerprompt}
>b.inv(b)
\end{eulerprompt}
\begin{euleroutput}
              1             0 
              0             1 
\end{euleroutput}
\begin{eulercomment}
Perkalian matriks dan perpangkatan matriks
\end{eulercomment}
\begin{eulerprompt}
>b.b
\end{eulerprompt}
\begin{euleroutput}
             29            36 
             45            56 
\end{euleroutput}
\begin{eulerprompt}
>b^2
\end{eulerprompt}
\begin{euleroutput}
              9            16 
             25            36 
\end{euleroutput}
\begin{eulerprompt}
>b.b.b
\end{eulerprompt}
\begin{euleroutput}
            267           332 
            415           516 
\end{euleroutput}
\begin{eulerprompt}
>power(b,3)
\end{eulerprompt}
\begin{euleroutput}
            267           332 
            415           516 
\end{euleroutput}
\begin{eulercomment}
Pembagian matriks
\end{eulercomment}
\begin{eulerprompt}
>a/a
\end{eulerprompt}
\begin{euleroutput}
              1 
              1 
\end{euleroutput}
\begin{eulerprompt}
>a/b
\end{eulerprompt}
\begin{euleroutput}
       0.333333          0.25 
            0.4      0.333333 
\end{euleroutput}
\begin{eulercomment}
Perkalian invers matriks dengan matriks lainny
\end{eulercomment}
\begin{eulerprompt}
>b\(\backslash\)a
\end{eulerprompt}
\begin{euleroutput}
              1 
           -0.5 
\end{euleroutput}
\begin{eulerprompt}
>inv(b).a
\end{eulerprompt}
\begin{euleroutput}
              1 
           -0.5 
\end{euleroutput}
\begin{eulercomment}
Perkalian skalar
\end{eulercomment}
\begin{eulerprompt}
>b*2
\end{eulerprompt}
\begin{euleroutput}
              6             8 
             10            12 
\end{euleroutput}
\begin{eulerprompt}
>2*b
\end{eulerprompt}
\begin{euleroutput}
              6             8 
             10            12 
\end{euleroutput}
\begin{eulerprompt}
>[1,2]*2
\end{eulerprompt}
\begin{euleroutput}
  [2,  4]
\end{euleroutput}
\eulerheading{Fungsi Matriks Lainnya}
\begin{eulercomment}
Untuk membangun matriks, kita dapat menumpuk satu matriks di atas yang
lain. Jika keduanya tidak memiliki jumlah kolom yang sama, kolom yang
lebih pendek akan diisi dengan 0.
\end{eulercomment}
\begin{eulerprompt}
>v=1:3; v_v
\end{eulerprompt}
\begin{euleroutput}
              1             2             3 
              1             2             3 
\end{euleroutput}
\begin{eulerprompt}
>A=random(3,4)
\end{eulerprompt}
\begin{euleroutput}
       0.525184      0.502255      0.168603      0.262253 
       0.866587      0.536137      0.493453      0.601344 
       0.659461      0.967468      0.193151      0.935921 
\end{euleroutput}
\begin{eulerprompt}
>A|1
\end{eulerprompt}
\begin{euleroutput}
       0.525184      0.502255      0.168603      0.262253             1 
       0.866587      0.536137      0.493453      0.601344             1 
       0.659461      0.967468      0.193151      0.935921             1 
\end{euleroutput}
\begin{eulerprompt}
>[v,v]
\end{eulerprompt}
\begin{euleroutput}
  [1,  2,  3,  1,  2,  3]
\end{euleroutput}
\begin{eulerprompt}
>[v;v]
\end{eulerprompt}
\begin{euleroutput}
              1             2             3 
              1             2             3 
\end{euleroutput}
\begin{eulerprompt}
>[v',v']
\end{eulerprompt}
\begin{euleroutput}
              1             1 
              2             2 
              3             3 
\end{euleroutput}
\begin{eulerprompt}
>"[x,x^2]"(v')
\end{eulerprompt}
\begin{euleroutput}
              1             1 
              2             4 
              3             9 
\end{euleroutput}
\begin{eulerprompt}
>length(2:10)
\end{eulerprompt}
\begin{euleroutput}
  9
\end{euleroutput}
\begin{eulerprompt}
>ones(2,2)
\end{eulerprompt}
\begin{euleroutput}
              1             1 
              1             1 
\end{euleroutput}
\begin{eulerprompt}
>zeros(2,2)
\end{eulerprompt}
\begin{euleroutput}
              0             0 
              0             0 
\end{euleroutput}
\begin{eulerprompt}
>ones(5)*6
\end{eulerprompt}
\begin{euleroutput}
  [6,  6,  6,  6,  6]
\end{euleroutput}
\begin{eulerprompt}
>random(1,2)
\end{eulerprompt}
\begin{euleroutput}
  [0.0728753,  0.988966]
\end{euleroutput}
\begin{eulercomment}
Berikut adalah fungsi lain yang berguna, yang merestrukturisasi elemen
matriks menjadi matriks lain.
\end{eulercomment}
\begin{eulerprompt}
>redim(1:9,3,3)
\end{eulerprompt}
\begin{euleroutput}
              1             2             3 
              4             5             6 
              7             8             9 
\end{euleroutput}
\begin{eulerprompt}
>function rep(v,n) := redim(dup(v,n),1,n*cols(v))
>rep(1:3,5)
\end{eulerprompt}
\begin{euleroutput}
  [1,  2,  3,  1,  2,  3,  1,  2,  3,  1,  2,  3,  1,  2,  3]
\end{euleroutput}
\begin{eulerprompt}
>multdup(1:3,3)
\end{eulerprompt}
\begin{euleroutput}
  [1,  1,  1,  2,  2,  2,  3,  3,  3]
\end{euleroutput}
\begin{eulercomment}
Fungsi flipx() dan flipy() mengembalikan urutan baris atau kolom
matriks. Yaitu, fungsi flipx() membalik secara horizontal.\\
Keduanya tidak memiliki jumlah kolom yang sama, kolom yang lebih
pendek akan diisi dengan 0.
\end{eulercomment}
\begin{eulerprompt}
>flipx(1:5)
\end{eulerprompt}
\begin{euleroutput}
  [5,  4,  3,  2,  1]
\end{euleroutput}
\begin{eulerprompt}
>rotleft(1:5)
\end{eulerprompt}
\begin{euleroutput}
  [2,  3,  4,  5,  1]
\end{euleroutput}
\begin{eulerprompt}
>rotright(1:5)
\end{eulerprompt}
\begin{euleroutput}
  [5,  1,  2,  3,  4]
\end{euleroutput}
\begin{eulercomment}
Sebuah fungsi khusus adalah drop(v,i), yang menghilangkan elemen
dengan indeks di i dari vektor v
\end{eulercomment}
\begin{eulerprompt}
>drop(10:20,3)
\end{eulerprompt}
\begin{euleroutput}
  [10,  11,  13,  14,  15,  16,  17,  18,  19,  20]
\end{euleroutput}
\begin{eulercomment}
Ada beberapa fungsi khusus untuk mengatur diagonal atau untuk
menghasilkan matriks diagonal. Kita mulai dengan matriks identitas
\end{eulercomment}
\begin{eulerprompt}
>A=id(3)
\end{eulerprompt}
\begin{euleroutput}
              1             0             0 
              0             1             0 
              0             0             1 
\end{euleroutput}
\eulerheading{Vektorisasi}
\begin{eulercomment}
Hampir semua fungsi di Euler juga berfungsi untuk input matriks dan
vektor, kapan pun ini masuk akal. Misalnya, fungsi sqrt() menghitung
akar kuadrat dari semua elemen vektor atau matriks.

\end{eulercomment}
\begin{eulerprompt}
>sqrt(1:4)
\end{eulerprompt}
\begin{euleroutput}
  [1,  1.41421,  1.73205,  2]
\end{euleroutput}
\begin{eulercomment}
Jadi, kamu dapat dengan mudah membuat tabel nilai. Ini adalah salah
satu cara untuk memplot suatu fungsi(alternatifnya menggunakan
ekspresi).
\end{eulercomment}
\begin{eulerprompt}
>x=3:0.05:6; y=log(x)
\end{eulerprompt}
\begin{euleroutput}
  [1.09861,  1.11514,  1.1314,  1.1474,  1.16315,  1.17865,  1.19392,
  1.20896,  1.22378,  1.23837,  1.25276,  1.26695,  1.28093,  1.29473,
  1.30833,  1.32176,  1.335,  1.34807,  1.36098,  1.37372,  1.38629,
  1.39872,  1.41099,  1.42311,  1.43508,  1.44692,  1.45862,  1.47018,
  1.4816,  1.4929,  1.50408,  1.51513,  1.52606,  1.53687,  1.54756,
  1.55814,  1.56862,  1.57898,  1.58924,  1.59939,  1.60944,  1.61939,
  1.62924,  1.639,  1.64866,  1.65823,  1.66771,  1.6771,  1.6864,
  1.69562,  1.70475,  1.7138,  1.72277,  1.73166,  1.74047,  1.7492,
  1.75786,  1.76644,  1.77495,  1.78339,  1.79176]
\end{euleroutput}
\begin{eulercomment}
Dengan ini dan operator titik dua a:delta:b, vektor nilai fungsi dapat
dihasilkan dengan mudah. Pada contoh berikut, kita membangkitkan
vektor nilai t[i] dengan spasi 0,1 dari -1 hingga 3. Kemudian kita
membangkitkan vektor nilai fungsi.\\
lateks:s=t\textasciicircum{}3-t
\end{eulercomment}
\begin{eulerprompt}
>t=-1:0.1:3; s=t^3-t
\end{eulerprompt}
\begin{euleroutput}
  [0,  0.171,  0.288,  0.357,  0.384,  0.375,  0.336,  0.273,  0.192,
  0.099,  0,  -0.099,  -0.192,  -0.273,  -0.336,  -0.375,  -0.384,
  -0.357,  -0.288,  -0.171,  0,  0.231,  0.528,  0.897,  1.344,  1.875,
  2.496,  3.213,  4.032,  4.959,  6,  7.161,  8.448,  9.867,  11.424,
  13.125,  14.976,  16.983,  19.152,  21.489,  24]
\end{euleroutput}
\begin{eulerprompt}
>shortest (1:7)*(1:7)'
\end{eulerprompt}
\begin{euleroutput}
       1      2      3      4      5      6      7 
       2      4      6      8     10     12     14 
       3      6      9     12     15     18     21 
       4      8     12     16     20     24     28 
       5     10     15     20     25     30     35 
       6     12     18     24     30     36     42 
       7     14     21     28     35     42     49 
\end{euleroutput}
\begin{eulercomment}
Perhatikan, bahwa ini sangat berbeda dari produk matriks. Produk
matriks dilambangkan dengan titik "." di EMT.
\end{eulercomment}
\begin{eulerprompt}
>(1:7).(1:7)'
\end{eulerprompt}
\begin{euleroutput}
  140
\end{euleroutput}
\begin{eulercomment}
Secara default, vektor baris dicetak dalam format yang ringkas.
\end{eulercomment}
\begin{eulerprompt}
>[6,7,8,9]
\end{eulerprompt}
\begin{euleroutput}
  [6,  7,  8,  9]
\end{euleroutput}
\begin{eulercomment}
Untuk matriks operator khusus . menunjukkan perkalian matriks, dan A'
menunjukkan transpos. Matriks 1x1 dapat digunakan seperti bilangan
real.
\end{eulercomment}
\begin{eulerprompt}
>v:=[2,3]; v.v', %^2
\end{eulerprompt}
\begin{euleroutput}
  13
  169
\end{euleroutput}
\begin{eulercomment}
Untuk mentranspos matriks kita menggunakan apostrof
\end{eulercomment}
\begin{eulerprompt}
>v=3:6; v'
\end{eulerprompt}
\begin{euleroutput}
              3 
              4 
              5 
              6 
\end{euleroutput}
\begin{eulerprompt}
>A=[1,2,3,4]; A.v'
\end{eulerprompt}
\begin{euleroutput}
  50
\end{euleroutput}
\begin{eulercomment}
Perhatikan bahwa v masih merupakan vektor baris, Jadi v'.v berbeda
dengan v.v'.
\end{eulercomment}
\begin{eulerprompt}
>v'.v
\end{eulerprompt}
\begin{euleroutput}
              9            12            15            18 
             12            16            20            24 
             15            20            25            30 
             18            24            30            36 
\end{euleroutput}
\begin{eulercomment}
v.v' menghitung norma v kuadrat untuk vektor baris v. Hasilnya adalah
vektor 1x1, yang bekerja seperti bilangan real.
\end{eulercomment}
\begin{eulerprompt}
>v.v'
\end{eulerprompt}
\begin{euleroutput}
  86
\end{euleroutput}
\begin{eulercomment}
Ada juga fungsi norma (bersama dengan banyak fungsi lain dari Aljabar
Linier).
\end{eulercomment}
\begin{eulerprompt}
>norm(v)^3
\end{eulerprompt}
\begin{euleroutput}
  797.531190613
\end{euleroutput}
\begin{eulercomment}
Operator dan fungsi mematuhi bahasa matriks Euler.\\
Berikut ringkasan aturannya.\\
- Fungsi yang diterapkan ke vektor atau matriks diterapkan ke setiap
elemen.\\
- Operator yang beroperasi pada dua matriks dengan ukuran yang sama
diterapkan berpasangan ke elemen matriks.\\
- jika kedua matriks memiliki dimensi yang berbeda, keduanya diperluas
dengan cara yang masuk akal, sehingga memiliki ukuran yang sama.\\
Misalnya, nilai skalar kali vektor mengalikan nilai dengan setiap
elemen vektor. Atau matriks kali vektor (dengan *, bukan.) memperluas
vektor ke ukuran matriks dengan menduplikasikan.\\
Berikut ini adalah kasus sederhana dengan operator\textasciicircum{}.
\end{eulercomment}
\begin{eulerprompt}
>[2,3,6]^2
\end{eulerprompt}
\begin{euleroutput}
  [4,  9,  36]
\end{euleroutput}
\begin{eulercomment}
Berikut adalah kasus yang lebih rumit. Vektor baris dikalikan dengan
vektor kolom mengembang keduanya\\
dengan menduplikasi.
\end{eulercomment}
\begin{eulerprompt}
>v:=[2,3,6]; v*v'
\end{eulerprompt}
\begin{euleroutput}
              4             6            12 
              6             9            18 
             12            18            36 
\end{euleroutput}
\begin{eulercomment}
Perhatikan bahwa produk skalar menggunakan produk matriks, bukan *!
\end{eulercomment}
\begin{eulerprompt}
>v.v'
\end{eulerprompt}
\begin{euleroutput}
  49
\end{euleroutput}
\begin{eulercomment}
Ada banyak fungsi matriks. Kami memberikan daftar singkat. Anda harus
berkonsultasi dengan dokumentasi untuk informasi lebih lanjut tentang
perintah ini.

\end{eulercomment}
\begin{eulerttcomment}
 sum,prod menghitung jumlah dan produk dari baris 
 cumsum,cumprod melakukan hal yang sama secara kumulatif 
 menghitung nilai ekstrem dari setiap baris 
 extrema mengembalikan vektor dengan informasi ekstrim 
 diag(A,i) mengembalikan diagonal ke-i 
 setdiag(A,i,v) mengatur diagonal  ke-i 
 id(n) matriks identitas 
 det(A) penentu 
 charpoly(A) polinomial  karakteristik 
 nilai eigen(A) nilai eigen.
\end{eulerttcomment}
\begin{eulerprompt}
>v*v, sum(v*v), cumsum(v*v)
\end{eulerprompt}
\begin{euleroutput}
  [4,  9,  36]
  49
  [4,  13,  49]
\end{euleroutput}
\begin{eulercomment}
Operator : menghasilkan vektor baris spasi yang sama, opsional dengan
ukuran langkah.
\end{eulercomment}
\begin{eulerprompt}
>3:9, 0:2:8
\end{eulerprompt}
\begin{euleroutput}
  [3,  4,  5,  6,  7,  8,  9]
  [0,  2,  4,  6,  8]
\end{euleroutput}
\begin{eulerprompt}
>[1,2]|[3,4,5], [1,2]_5
\end{eulerprompt}
\begin{euleroutput}
  [1,  2,  3,  4,  5]
              1             2 
              5             5 
\end{euleroutput}
\begin{eulercomment}
Unsur-unsur matriks disebut dengan "A[i,j]".
\end{eulercomment}
\begin{eulerprompt}
>A:=[4,5,6,7;8,9,10,11]; A[2,2]
\end{eulerprompt}
\begin{euleroutput}
  9
\end{euleroutput}
\begin{eulercomment}
Untuk vektor baris atau kolom, v[i] adalah elemen ke-i dari vektor.
Untuk matriks, ini mengembalikan baris ke-i lengkap dari matriks.
\end{eulercomment}
\begin{eulerprompt}
>v:=[1,3,5,7]; v[1], A[1]
\end{eulerprompt}
\begin{euleroutput}
  1
  [4,  5,  6,  7]
\end{euleroutput}
\begin{eulercomment}
Indeks juga bisa menjadi vektor baris dari indeks. : menunjukkan semua
indeks.
\end{eulercomment}
\begin{eulerprompt}
>v[2:4], A[:,2]
\end{eulerprompt}
\begin{euleroutput}
  [3,  5,  7]
              5 
              9 
\end{euleroutput}
\begin{eulercomment}
Bentuk singkat untuk : adalah menghilangkan indeks sepenuhnya.
\end{eulercomment}
\begin{eulerprompt}
>A[,2:4]
\end{eulerprompt}
\begin{euleroutput}
              5             6             7 
              9            10            11 
\end{euleroutput}
\begin{eulercomment}
Untuk tujuan vektorisasi, elemen matriks dapat diakses seolah-olah
mereka adalah vektor.
\end{eulercomment}
\begin{eulerprompt}
>A\{5\}
\end{eulerprompt}
\begin{euleroutput}
  8
\end{euleroutput}
\begin{eulercomment}
Matriks juga dapat diratakan, menggunakan fungsi redim(). Ini
diimplementasikan dalam fungsi flatten().
\end{eulercomment}
\begin{eulerprompt}
>redim(A,1,prod(size(A))), flatten(A)
\end{eulerprompt}
\begin{euleroutput}
  [4,  5,  6,  7,  8,  9,  10,  11]
  [4,  5,  6,  7,  8,  9,  10,  11]
\end{euleroutput}
\begin{eulercomment}
Untuk menggunakan matriks untuk tabel, mari kita reset ke format
default, dan menghitung tabel nilai sinus dan kosinus. Perhatikan
bahwa sudut dalam radian secara default.
\end{eulercomment}
\begin{eulerprompt}
>defformat; w=0°:45°:360°; w=w'; deg(w)
\end{eulerprompt}
\begin{euleroutput}
              0 
             45 
             90 
            135 
            180 
            225 
            270 
            315 
            360 
\end{euleroutput}
\begin{eulercomment}
Sekarang kita menambahkan kolom ke matriks.
\end{eulercomment}
\begin{eulerprompt}
>M = deg(w)|w|cos(w)|sin(w)
\end{eulerprompt}
\begin{euleroutput}
              0             0             1             0 
             45      0.785398      0.707107     0.0137074 
             90        1.5708             0     0.0274121 
            135       2.35619     -0.707107     0.0411118 
            180       3.14159            -1     0.0548037 
            225       3.92699     -0.707107     0.0684853 
            270       4.71239             0      0.082154 
            315       5.49779      0.707107     0.0958073 
            360       6.28319             1      0.109443 
\end{euleroutput}
\begin{eulercomment}
Dengan menggunakan bahasa matriks, kita dapat menghasilkan beberapa
tabel dari beberapa fungsi sekaligus.\\
Dalam contoh berikut, kita menghitung t[j]\textasciicircum{}i untuk i dari 1 hingga n.
Kami mendapatkan matriks, di mana\\
setiap baris adalah tabel t\textasciicircum{}i untuk satu i. Yaitu, matriks memiliki
elemen lateks: a\_\{i,j\} = t\_j\textasciicircum{}i, \textbackslash{}quad 1 \textbackslash{}le j\\
\textbackslash{}le 101, \textbackslash{}quad 1 \textbackslash{}le i \textbackslash{}le n\\
Fungsi yang tidak berfungsi untuk input vektor harus "divektorkan".
Ini dapat dicapai dengan kata kunci\\
"peta" dalam definisi fungsi. Kemudian fungsi tersebut akan dievaluasi
untuk setiap elemen dari parameter\\
vektor.\\
Integrasi numerik terintegrasi() hanya berfungsi untuk batas interval
skalar. Jadi kita perlu membuat vektor.
\end{eulercomment}
\begin{eulerprompt}
>function map f(x) := integrate("x^x",1,x)
\end{eulerprompt}
\begin{eulercomment}
Kata kunci "peta" membuat vektor fungsi. Fungsinya sekarang akan
bekerja untuk vektor bilangan.
\end{eulercomment}
\begin{eulerprompt}
>f([1:3])
\end{eulerprompt}
\begin{euleroutput}
  [0,  2.05045,  13.7251]
\end{euleroutput}
\eulerheading{Sub-Matriks dan Matriks-Elemen}
\begin{eulercomment}
Untuk mengakses elemen matriks, gunakan notasi braket.
\end{eulercomment}
\begin{eulerprompt}
>A=[1,2,3;4,5,6;7,8,9], A[2,3]
\end{eulerprompt}
\begin{euleroutput}
              1             2             3 
              4             5             6 
              7             8             9 
  6
\end{euleroutput}
\begin{eulercomment}
Kita dapat mengakses satu baris matriks yang lengkap.
\end{eulercomment}
\begin{eulerprompt}
>A[1]
\end{eulerprompt}
\begin{euleroutput}
  [1,  2,  3]
\end{euleroutput}
\begin{eulercomment}
Dalam kasus vektor baris atau kolom, ini mengembalikan elemen vektor.
\end{eulercomment}
\begin{eulerprompt}
>v=1:8; v[2]
\end{eulerprompt}
\begin{euleroutput}
  2
\end{euleroutput}
\begin{eulercomment}
Untuk memastikan, Anda mendapatkan baris pertama untuk matriks 1xn dan
mxn, tentukan semua kolom menggunakan indeks kedua kosong.
\end{eulercomment}
\begin{eulerprompt}
>A[3,]
\end{eulerprompt}
\begin{euleroutput}
  [7,  8,  9]
\end{euleroutput}
\begin{eulercomment}
Jika indeks adalah vektor indeks, Euler akan mengembalikan baris
matriks yang sesuai. Di sini kita ingin baris pertama dan kedua dari
A.
\end{eulercomment}
\begin{eulerprompt}
>A[[2,3]]
\end{eulerprompt}
\begin{euleroutput}
              4             5             6 
              7             8             9 
\end{euleroutput}
\begin{eulercomment}
Kita bahkan dapat menyusun ulang A menggunakan vektor indeks.
Tepatnya, kami tidak mengubah A disini, tetapi menghitung versi A yang
disusun ulang.
\end{eulercomment}
\begin{eulerprompt}
>A[[3,2,1]]
\end{eulerprompt}
\begin{euleroutput}
              7             8             9 
              4             5             6 
              1             2             3 
\end{euleroutput}
\begin{eulercomment}
Trik indeks bekerja dengan kolom juga.\\
Contoh ini memilih semua baris A dan kolom kedua dan ketiga.
\end{eulercomment}
\begin{eulerprompt}
>A[1:2,2:3]
\end{eulerprompt}
\begin{euleroutput}
              2             3 
              5             6 
\end{euleroutput}
\begin{eulercomment}
Untuk singkatan ":" menunjukkan semua indeks baris atau kolom.
\end{eulercomment}
\begin{eulerprompt}
>A[:,2]
\end{eulerprompt}
\begin{euleroutput}
              2 
              5 
              8 
\end{euleroutput}
\begin{eulercomment}
Atau, biarkan indeks pertama kosong.
\end{eulercomment}
\begin{eulerprompt}
>A[,1:3]
\end{eulerprompt}
\begin{euleroutput}
              1             2             3 
              4             5             6 
              7             8             9 
\end{euleroutput}
\begin{eulercomment}
Kita juga bisa mendapatkan baris terakhir dari A.
\end{eulercomment}
\begin{eulerprompt}
>A[3]
\end{eulerprompt}
\begin{euleroutput}
  [7,  8,  9]
\end{euleroutput}
\begin{eulercomment}
Sekarang mari kita ubah elemen A dengan menetapkan submatriks A ke
beberapa nilai. Ini sebenarnya mengubah matriks A yang disimpan.
\end{eulercomment}
\begin{eulerprompt}
>A[2,3]=9
\end{eulerprompt}
\begin{euleroutput}
              1             2             3 
              4             5             9 
              7             8             9 
\end{euleroutput}
\begin{eulercomment}
Kami bahkan dapat menetapkan sub-matriks jika memiliki ukuran yang
tepat.
\end{eulercomment}
\begin{eulerprompt}
>A[1:2,1:2]=[4,5;6,7]
\end{eulerprompt}
\begin{euleroutput}
              4             5             3 
              6             7             9 
              7             8             9 
\end{euleroutput}
\begin{eulercomment}
Selain itu, beberapa jalan pintas diperbolehkan.
\end{eulercomment}
\begin{eulerprompt}
>A[1:2,1:2]=-1
\end{eulerprompt}
\begin{euleroutput}
             -1            -1             3 
             -1            -1             9 
              7             8             9 
\end{euleroutput}
\begin{eulercomment}
Peringatan: Indeks di luar batas mengembalikan matriks kosong, atau
pesan kesalahan, tergantung pada pengaturan sistem. Standarnya adalah
pesan kesalahan. Ingat, bagaimanapun, bahwa indeks negatif dapat
digunakan untuk mengakses elemen matriks yang dihitung dari akhir.
\end{eulercomment}
\begin{eulerprompt}
>A[5]
\end{eulerprompt}
\begin{euleroutput}
  Row index 5 out of bounds!
  Error in:
  A[5] ...
      ^
\end{euleroutput}
\eulerheading{Menyortir dan Mengacak}
\begin{eulercomment}
Fungsi sort() mengurutkan vektor baris.
\end{eulercomment}
\begin{eulerprompt}
>sort([2,9,5,7,3,1])
\end{eulerprompt}
\begin{euleroutput}
  [1,  2,  3,  5,  7,  9]
\end{euleroutput}
\begin{eulercomment}
Seringkali perlu untuk mengetahui indeks dari vektor yang diurutkan
dalam vektor aslinya. Ini dapat digunakan untuk menyusun ulang vektor
lain dengan cara yang sama.\\
Mari kita mengacak vektor.
\end{eulercomment}
\begin{eulerprompt}
>v=shuffle(1:8) 
\end{eulerprompt}
\begin{euleroutput}
  [2,  5,  6,  8,  7,  4,  3,  1]
\end{euleroutput}
\begin{eulercomment}
Indeks berisi urutan yang tepat dari v.
\end{eulercomment}
\begin{eulerprompt}
>\{vs,ind\}=sort(v); v[ind]
\end{eulerprompt}
\begin{euleroutput}
  [1,  2,  3,  4,  5,  6,  7,  8]
\end{euleroutput}
\begin{eulerprompt}
>s=["d","f","c","b","aa","g"]
\end{eulerprompt}
\begin{euleroutput}
  d
  f
  c
  b
  aa
  g
\end{euleroutput}
\begin{eulerprompt}
>\{ss,ind\}=sort(s); ss
\end{eulerprompt}
\begin{euleroutput}
  aa
  b
  c
  d
  f
  g
\end{euleroutput}
\begin{eulerprompt}
>ind
\end{eulerprompt}
\begin{euleroutput}
  [8,  1,  7,  6,  2,  3,  5,  4]
\end{euleroutput}
\begin{eulercomment}
Fungsi unik mengembalikan daftar elemen unik vektor yang diurutkan.
\end{eulercomment}
\begin{eulerprompt}
>intrandom(1,10,10), unique(%)
\end{eulerprompt}
\begin{euleroutput}
  [6,  5,  9,  1,  6,  4,  4,  5,  9,  6]
  [1,  4,  5,  6,  9]
\end{euleroutput}
\begin{eulercomment}
Ini bekerja untuk vektor string juga.
\end{eulercomment}
\begin{eulerprompt}
>intrandom(1,5,10), unique(%)
\end{eulerprompt}
\begin{euleroutput}
  [3,  1,  4,  3,  10]
  [1,  3,  4,  10]
\end{euleroutput}
\begin{eulerprompt}
>unique(s)
\end{eulerprompt}
\begin{euleroutput}
  aa
  b
  c
  d
  f
  g
\end{euleroutput}
\begin{euleroutput}
  \(\backslash\)
\end{euleroutput}
\eulersubheading{Aljabar Linier}
\begin{eulercomment}
EMT memiliki banyak fungsi untuk menyelesaikan sistem linier, sistem
sparse, atau masalah regresi.\\
Untuk sistem linier Ax=b,dengan A adalah matriks koefisien, x adalah
vektor solusi yang ingin kita cari dan b adalah vektor hasil yang
diberikan.\\
Anda dapat menggunakan algoritma Gauss, matriks invers atau kecocokan
linier.\\
Operator A\textbackslash{}b menggunakan versi algoritma Gauss.\\
Operator backslash \textbackslash{} digunakan untuk menyelesaikan sistem persamaan
linier ini. Ketika menulis A\textbackslash{}b, perangkat lunak akan menghitung solusi
yang memenuhi persamaan Ax=b.\\
Operator ini secara otomatis menggunakan algoritma eliminasi Gauss
atau metode numerik serupa untuk menemukan solusi.
\end{eulercomment}
\begin{eulerprompt}
>A=[5,6;7,8]; b= [4;3]; A\(\backslash\)b
\end{eulerprompt}
\begin{euleroutput}
             -7 
            6.5 
\end{euleroutput}
\begin{eulercomment}
Untuk contoh lain, kami membuat matriks 100x100 dan jumlah barisnya.
Kemudian kita selesaikan Ax=b\\
menggunakan matriks invers. Kami mengukur kesalahan sebagai deviasi
maksimal semua elemen dari 1,\\
yang tentu saja merupakan solusi yang benar.
\end{eulercomment}
\begin{eulerprompt}
>A=normal(100,100); b=sum(A); longest totalmax(abs(inv(A).b-1))
\end{eulerprompt}
\begin{euleroutput}
    2.873257187729905e-13 
\end{euleroutput}
\begin{eulercomment}
Jika sistem tidak memiliki solusi, kecocokan linier meminimalkan norma\\
kesalahan Ax-b.
\end{eulercomment}
\begin{eulerprompt}
>A=[2,5,7;3,6,8;9,1,7]
\end{eulerprompt}
\begin{euleroutput}
              2             5             7 
              3             6             8 
              9             1             7 
\end{euleroutput}
\begin{eulerprompt}
>det(A)
\end{eulerprompt}
\begin{euleroutput}
  -34
\end{euleroutput}
\eulerheading{Matriks Simbolik}
\begin{eulercomment}
Maxima memiliki matriks simbolis. Tentu saja, Maxima dapat digunakan
untuk masalah aljabar linier sederhana seperti itu.\\
Kita dapat mendefinisikan matriks untuk Euler dan Maxima dengan \&:=,
dan kemudian menggunakannya dalam ekspresi simbolis. Bentuk [...]
biasa untuk mendefinisikan matriks dapat digunakan di Euler untuk
mendefinisikan matriks simbolik.
\end{eulercomment}
\begin{eulerprompt}
>A &= [a,1,1;1,a,1;1,1,a]; $A
\end{eulerprompt}
\begin{eulerformula}
\[
\begin{pmatrix}a & 1 & 1 \\ 1 & a & 1 \\ 1 & 1 & a \\ \end{pmatrix}
\]
\end{eulerformula}
\begin{eulerprompt}
> $&det(A), $&factor(%) 
\end{eulerprompt}
\begin{eulerformula}
\[
\left(a-1\right)^2\,\left(a+2\right)
\]
\end{eulerformula}
\eulerimg{0}{images/Tugas Individu Pekan3-4_Muhammad Lutfi Ramadhan_23030630021-085-large.png}
\begin{eulerprompt}
>$&invert(A) with a=0
\end{eulerprompt}
\begin{eulerformula}
\[
\begin{pmatrix}-\frac{1}{2} & \frac{1}{2} & \frac{1}{2} \\ \frac{1  }{2} & -\frac{1}{2} & \frac{1}{2} \\ \frac{1}{2} & \frac{1}{2} & -  \frac{1}{2} \\ \end{pmatrix}
\]
\end{eulerformula}
\begin{eulerprompt}
>A &= [1,a;b,2]; $A
\end{eulerprompt}
\begin{eulerformula}
\[
\begin{pmatrix}1 & a \\ b & 2 \\ \end{pmatrix}
\]
\end{eulerformula}
\begin{eulercomment}
Seperti semua variabel simbolik, matriks ini dapat digunakan dalam
ekspresi simbolik lainnya.
\end{eulercomment}
\begin{eulerprompt}
>$&det(A-x*ident(2)), $&solve(%,x)
\end{eulerprompt}
\begin{eulerformula}
\[
\left[ x=\frac{3-\sqrt{4\,a\,b+1}}{2} , x=\frac{\sqrt{4\,a\,b+1}+3  }{2} \right] 
\]
\end{eulerformula}
\eulerimg{1}{images/Tugas Individu Pekan3-4_Muhammad Lutfi Ramadhan_23030630021-089-large.png}
\begin{eulercomment}
Nilai eigen juga dapat dihitung secara otomatis. Hasilnya adalah
vektor dengan dua vektor nilai eigen dan multiplisitas.
\end{eulercomment}
\begin{eulerprompt}
>$&eigenvalues([a,1;1,a])
\end{eulerprompt}
\begin{eulerformula}
\[
\left[ \left[ a-1 , a+1 \right]  , \left[ 1 , 1 \right]  \right] 
\]
\end{eulerformula}
\begin{eulercomment}
Untuk mengekstrak vektor eigen tertentu perlu pengindeksan yang
cermat.
\end{eulercomment}
\begin{eulerprompt}
>&eigenvectors([a,1;1,a]), &%[2][1][1]
\end{eulerprompt}
\begin{euleroutput}
  
            [[[a - 1, a + 1], [1, 1]], [[[1, - 1]], [[1, 1]]]]
  
  
                                 [1, - 1]
  
\end{euleroutput}
\begin{eulercomment}
Matriks simbolik dapat dievaluasi dalam Euler secara numerik seperti
ekspresi simbolik lainnya.
\end{eulercomment}
\begin{eulerprompt}
>A(a=6,b=7)
\end{eulerprompt}
\begin{euleroutput}
              1             6 
              7             2 
\end{euleroutput}
\begin{eulercomment}
Dalam ekspresi simbolik, gunakan dengan.
\end{eulercomment}
\begin{eulerprompt}
>$&A with [a=6,b=7]
\end{eulerprompt}
\begin{eulerformula}
\[
\begin{pmatrix}1 & 6 \\ 7 & 2 \\ \end{pmatrix}
\]
\end{eulerformula}
\begin{eulercomment}
Akses ke baris matriks simbolik bekerja seperti halnya dengan matriks\\
numerik.
\end{eulercomment}
\begin{eulerprompt}
>$&A[1]
\end{eulerprompt}
\begin{eulerformula}
\[
\left[ 1 , a \right] 
\]
\end{eulerformula}
\begin{eulercomment}
Ekspresi simbolis dapat berisi tugas. Dan itu mengubah matriks A.
\end{eulercomment}
\begin{eulerprompt}
>&A[1,1]:=t+1; $&A
\end{eulerprompt}
\begin{eulerformula}
\[
\begin{pmatrix}t+1 & a \\ b & 2 \\ \end{pmatrix}
\]
\end{eulerformula}
\begin{eulercomment}
Ada fungsi simbolik di Maxima untuk membuat vektor dan matriks. Untuk
ini, lihat dokumentasi Maxima atau tutorial tentang Maxima di EMT.
\end{eulercomment}
\begin{eulerprompt}
>v &= makelist(1/(i+j),i,1,3); $v
\end{eulerprompt}
\begin{eulerformula}
\[
\left[ \frac{1}{j+1} , \frac{1}{j+2} , \frac{1}{j+3} \right] 
\]
\end{eulerformula}
\begin{eulerprompt}
>B &:= [1,2;3,4]; $B, $&invert(B)
\end{eulerprompt}
\begin{eulerformula}
\[
\begin{pmatrix}-2 & 1 \\ \frac{3}{2} & -\frac{1}{2} \\   \end{pmatrix}
\]
\end{eulerformula}
\eulerimg{1}{images/Tugas Individu Pekan3-4_Muhammad Lutfi Ramadhan_23030630021-096-large.png}
\begin{eulercomment}
Hasilnya dapat dievaluasi secara numerik dalam Euler. Untuk informasi\\
lebih lanjut tentang Maxima, lihat pengantar Maxima.
\end{eulercomment}
\begin{eulerprompt}
>$&invert(B)()
\end{eulerprompt}
\begin{euleroutput}
             -2             1 
            1.5          -0.5 
\end{euleroutput}
\begin{eulercomment}
Euler juga memiliki fungsi xinv() yang kuat, yang membuat upaya lebih\\
besar dan mendapatkan hasil yang lebih tepat.

Perhatikan, bahwa dengan \&:= matriks B telah didefinisikan sebagai\\
simbolik dalam ekspresi simbolik dan sebagai numerik dalam ekspresi\\
numerik. Jadi kita bisa menggunakannya di sini.
\end{eulercomment}
\begin{eulerprompt}
>longest B.xinv(B)
\end{eulerprompt}
\begin{euleroutput}
                        1                       0 
                        0                       1 
\end{euleroutput}
\begin{eulercomment}
Misalnya. nilai eigen dari A dapat dihitung secara numerik.
\end{eulercomment}
\begin{eulerprompt}
>A=[1,2,3;4,5,6;7,8,9]; real(eigenvalues(A))
\end{eulerprompt}
\begin{euleroutput}
  [16.1168,  -1.11684,  0]
\end{euleroutput}
\begin{eulercomment}
Atau secara simbolis. Lihat tutorial tentang Maxima untuk detailnya.
\end{eulercomment}
\begin{eulerprompt}
>$&eigenvalues(@A)  
\end{eulerprompt}
\begin{eulerformula}
\[
\left[ \left[ \frac{15-3\,\sqrt{33}}{2} , \frac{3\,\sqrt{33}+15}{2}   , 0 \right]  , \left[ 1 , 1 , 1 \right]  \right] 
\]
\end{eulerformula}
\begin{eulercomment}
\end{eulercomment}
\eulersubheading{Nilai Numerik dalam Ekspresi simbolis}
\begin{eulercomment}
Ekspresi simbolis hanyalah string yang berisi ekspresi. Jika kita
ingin mendefinisikan nilai baik untuk ekspresi simbolik maupun
ekspresi numerik, kita harus menggunakan "\&:=".
\end{eulercomment}
\begin{eulerprompt}
>A &:= [1,pi;4,5]
\end{eulerprompt}
\begin{euleroutput}
              1       3.14159 
              4             5 
\end{euleroutput}
\begin{eulercomment}
Masih ada perbedaan antara bentuk numerik dan simbolik. Saat\\
mentransfer matriks ke bentuk simbolis, pendekatan fraksional untuk\\
real akan digunakan.
\end{eulercomment}
\begin{eulerprompt}
>$&A 
\end{eulerprompt}
\begin{eulerformula}
\[
\begin{pmatrix}t+1 & a \\ b & 2 \\ \end{pmatrix}
\]
\end{eulerformula}
\begin{eulercomment}
Untuk menghindarinya, ada fungsi "mxmset(variable)".
\end{eulercomment}
\begin{eulerprompt}
>mxmset(A); $&A
\end{eulerprompt}
\begin{eulerformula}
\[
\begin{pmatrix}1 & 2 & 3 \\ 4 & 5 & 6 \\ 7 & 8 & 9 \\ \end{pmatrix}
\]
\end{eulerformula}
\begin{eulercomment}
Maxima juga dapat menghitung dengan angka floating point, dan bahkan\\
dengan angka floating besar dengan 32 digit. Namun, evaluasinya jauh\\
lebih lambat.
\end{eulercomment}
\begin{eulerprompt}
>$&bfloat(sqrt(2)), $&float(sqrt(2))
\end{eulerprompt}
\begin{eulerformula}
\[
1.414213562373095
\]
\end{eulerformula}
\eulerimg{0}{images/Tugas Individu Pekan3-4_Muhammad Lutfi Ramadhan_23030630021-101-large.png}
\begin{eulercomment}
Ketepatan angka floating point besar dapat diubah
\end{eulercomment}
\begin{eulerprompt}
>&fpprec:=100; &bfloat(pi)
\end{eulerprompt}
\begin{euleroutput}
  
          3.14159265358979323846264338327950288419716939937510582097494\(\backslash\)
  4592307816406286208998628034825342117068b0
  
\end{euleroutput}
\begin{eulercomment}
Variabel numerik dapat digunakan dalam ekspresi simbolis apa pun
menggunakan "@var". \\
Perhatikan bahwa ini hanya diperlukan, jika variabel telah\\
didefinisikan dengan ":=" atau "=" sebagai variabel numerik.
\end{eulercomment}
\begin{eulerprompt}
>B:=[1,pi;3,4]; $&det(@B)
\end{eulerprompt}
\begin{eulerformula}
\[
-5.424777960769379
\]
\end{eulerformula}
\eulersubheading{Demo - Suku Bunga}
\begin{eulercomment}
Di bawah ini, kami menggunakan Euler Math Toolbox (EMT) untuk
perhitungan suku bunga. Kami melakukannya secara numerik dan simbolis
untuk menunjukkan kepada Anda bagaimana Euler dapat\\
digunakan untuk memecahkan masalah kehidupan nyata.

Asumsikan Anda memiliki modal awal 4000 (katakanlah dalam dolar).
\end{eulercomment}
\begin{eulerprompt}
>M=4000
\end{eulerprompt}
\begin{euleroutput}
  4000
\end{euleroutput}
\begin{eulercomment}
Sekarang kita asumsikan tingkat bunga 3\% per tahun. Mari kita
tambahkan satu tarif sederhana dan hitung hasilnya.
\end{eulercomment}
\begin{eulerprompt}
>M*1.03
\end{eulerprompt}
\begin{euleroutput}
  4120
\end{euleroutput}
\begin{eulercomment}
Euler akan memahami sintaks berikut juga.
\end{eulercomment}
\begin{eulerprompt}
>M+M*3%
\end{eulerprompt}
\begin{euleroutput}
  4120
\end{euleroutput}
\begin{eulercomment}
Tetapi lebih mudah menggunakan faktornya.
\end{eulercomment}
\begin{eulerprompt}
>q=1+3%, M*q
\end{eulerprompt}
\begin{euleroutput}
  1.03
  4120
\end{euleroutput}
\begin{eulercomment}
Selama 10 tahun, kita cukup mengalikan faktornya dan mendapatkan nilai
akhir dengan suku bunga majemuk.
\end{eulercomment}
\begin{eulerprompt}
>M*q^10
\end{eulerprompt}
\begin{euleroutput}
  5375.66551738
\end{euleroutput}
\begin{eulercomment}
Untuk tujuan kita, kita dapat mengatur format menjadi 2 digit setelah
titik desimal.
\end{eulercomment}
\begin{eulerprompt}
>format(12,2); M*q^10
\end{eulerprompt}
\begin{euleroutput}
      5375.67 
\end{euleroutput}
\begin{eulercomment}
Mari kita cetak yang dibulatkan menjadi 2 digit dalam kalimat lengkap.
\end{eulercomment}
\begin{eulerprompt}
>"Starting from " + M + "$ you get " + round(M*q^10,2) + "$."
\end{eulerprompt}
\begin{euleroutput}
  Starting from 4000$ you get 5375.67$.
\end{euleroutput}
\begin{eulercomment}
Bagaimana jika kita ingin mengetahui hasil antara dari tahun 1 sampai
tahun 9? Untuk ini, bahasa matriks Euler sangat membantu. Anda tidak
harus menulis loop, tetapi cukup masukkan
\end{eulercomment}
\begin{eulerprompt}
>M*q^(0:10)
\end{eulerprompt}
\begin{euleroutput}
  Real 1 x 11 matrix
  
      4000.00     4120.00     4243.60     4370.91     ...
\end{euleroutput}
\begin{eulercomment}
Bagaimana keajaiban ini bekerja? Pertama ekspresi 0:10 mengembalikan
vektor bilangan bulat.
\end{eulercomment}
\begin{eulerprompt}
>short 0:10
\end{eulerprompt}
\begin{euleroutput}
  [0,  1,  2,  3,  4,  5,  6,  7,  8,  9,  10]
\end{euleroutput}
\begin{eulercomment}
Kemudian semua operator dan fungsi dalam Euler dapat diterapkan pada
elemen vektor untuk elemen.
\end{eulercomment}
\begin{eulerprompt}
>short q^(0:10)
\end{eulerprompt}
\begin{euleroutput}
  [1,  1.03,  1.0609,  1.0927,  1.1255,  1.1593,  1.1941,  1.2299,
  1.2668,  1.3048,  1.3439]
\end{euleroutput}
\begin{eulercomment}
adalah vektor faktor qˆ0 sampai qˆ10. Ini dikalikan dengan M, dan kami
mendapatkan vektor nilai.
\end{eulercomment}
\begin{eulerprompt}
>VM=M*q^(0:10)
\end{eulerprompt}
\begin{euleroutput}
  Real 1 x 11 matrix
  
      4000.00     4120.00     4243.60     4370.91     ...
\end{euleroutput}
\begin{eulercomment}
Tentu saja, cara realistis untuk menghitung suku bunga ini adalah
dengan membulatkan ke sen terdekat setelah setiap tahun. Mari kita
tambahkan fungsi untuk ini.
\end{eulercomment}
\begin{eulerprompt}
>function oneyear(M) := round(M * q, 2)
\end{eulerprompt}
\begin{eulercomment}
Mari kita bandingkan dua hasil, dengan dan tanpa pembulatan.
\end{eulercomment}
\begin{eulerprompt}
>longest oneyear(1234.57), longest 1234.57*q
\end{eulerprompt}
\begin{euleroutput}
                  1271.61 
                1271.6071 
\end{euleroutput}
\begin{eulercomment}
Sekarang tidak ada rumus sederhana untuk tahun ke-n, dan kita harus
mengulang selama bertahun-tahun.\\
Euler memberikan banyak solusi untuk ini.\\
Cara termudah adalah iterasi fungsi, yang mengulangi fungsi tertentu
beberapa kali.
\end{eulercomment}
\begin{eulerprompt}
>VMr=iterate("oneyear",4000,10)
\end{eulerprompt}
\begin{euleroutput}
  Real 1 x 11 matrix
  
      4000.00     4120.00     4243.60     4370.91     ...
\end{euleroutput}
\begin{eulercomment}
Kami dapat mencetaknya dengan cara yang ramah, menggunakan format kami
dengan tempat desimal tetap.
\end{eulercomment}
\begin{eulerprompt}
>VMr'
\end{eulerprompt}
\begin{euleroutput}
      4000.00 
      4120.00 
      4243.60 
      4370.91 
      4502.04 
      4637.10 
      4776.21 
      4919.50 
      5067.09 
      5219.10 
      5375.67 
\end{euleroutput}
\begin{eulercomment}
Untuk mendapatkan elemen tertentu dari vektor, kami menggunakan indeks
dalam tanda kurung siku.
\end{eulercomment}
\begin{eulerprompt}
>VMr[2], VMr[1:3]
\end{eulerprompt}
\begin{euleroutput}
      4120.00 
      4000.00     4120.00     4243.60 
\end{euleroutput}
\begin{eulercomment}
Anehnya, kita juga bisa menggunakan vektor indeks. Ingat bahwa 1:3
menghasilkan vektor [1,2,3].\\
Mari kita bandingkan elemen terakhir dari nilai yang dibulatkan dengan
nilai penuh.
\end{eulercomment}
\begin{eulerprompt}
>VMr[-2], VM[-2]
\end{eulerprompt}
\begin{euleroutput}
      5219.10 
      5219.09 
\end{euleroutput}
\begin{eulercomment}
Perbedaannya sangat kecil.

\end{eulercomment}
\eulersubheading{Memecahkan Persamaan}
\begin{eulercomment}
Sekarang kita mengambil fungsi yang lebih maju, yang menambahkan
tingkat uang tertentu setiap tahun.
\end{eulercomment}
\begin{eulerprompt}
>function onepay (M) := M*q+R
\end{eulerprompt}
\begin{eulercomment}
Kita tidak perlu menentukan q atau R untuk definisi fungsi. Hanya jika
kita menjalankan perintah, kita harus mendefinisikan nilai-nilai ini.
Kami memilih R=200.
\end{eulercomment}
\begin{eulerprompt}
>R=200; iterate("onepay",4000,10)
\end{eulerprompt}
\begin{euleroutput}
  Real 1 x 11 matrix
  
      4000.00     4320.00     4649.60     4989.09     ...
\end{euleroutput}
\begin{eulercomment}
Bagaimana jika kita menghapus jumlah yang sama setiap tahun?
\end{eulercomment}
\begin{eulerprompt}
>R=-200; iterate("onepay",4000,10)
\end{eulerprompt}
\begin{euleroutput}
  Real 1 x 11 matrix
  
      4000.00     3920.00     3837.60     3752.73     ...
\end{euleroutput}
\begin{eulercomment}
Kami melihat bahwa uang berkurang. Jelas, jika kita hanya mendapatkan
150 bunga di tahun pertama, tetapi menghapus 200, kita kehilangan uang
setiap tahun.\\
Bagaimana kita bisa menentukan berapa tahun uang itu akan bertahan?
Kita harus menulis loop untuk ini. Cara termudah adalah dengan iterasi
cukup lama.
\end{eulercomment}
\begin{eulerprompt}
>VMR=iterate("onepay",4000,50)
\end{eulerprompt}
\begin{euleroutput}
  Real 1 x 51 matrix
  
      4000.00     3920.00     3837.60     3752.73     ...
\end{euleroutput}
\begin{eulercomment}
Dengan menggunakan bahasa matriks, kita dapat menentukan nilai negatif
pertama dengan cara berikut.
\end{eulercomment}
\begin{eulerprompt}
>min(nonzeros(VMR<0))
\end{eulerprompt}
\begin{euleroutput}
        32.00 
\end{euleroutput}
\begin{eulercomment}
Alasan untuk ini adalah bahwa bukan nol(VKR\textless{}0) mengembalikan vektor
indeks i, di mana VKR[i]\textless{}0, dan min menghitung indeks minimal.\\
Karena vektor selalu dimulai dengan indeks 1, jawabannya adalah 31
tahun.\\
Fungsi iterate() memiliki satu trik lagi. Itu bisa mengambil kondisi
akhir sebagai argumen. Kemudian akan mengembalikan nilai dan jumlah
iterasi.
\end{eulercomment}
\begin{eulerprompt}
>\{x,n\}=iterate("onepay",4000,till="x<0"); x, n
\end{eulerprompt}
\begin{euleroutput}
        -0.21 
        31.00 
\end{euleroutput}
\begin{eulercomment}
Mari kita coba menjawab pertanyaan yang lebih ambigu. Asumsikan kita
tahu bahwa nilainya adalah 0 setelah 50 tahun. Apa yang akan menjadi
tingkat bunga?\\
Ini adalah pertanyaan yang hanya bisa dijawab dengan angka. Di bawah
ini, kita akan mendapatkan formula yang diperlukan. Kemudian Anda akan
melihat bahwa tidak ada formula yang mudah untuk tingkat bunga.\\
Tapi untuk saat ini, kami bertujuan untuk solusi numerik.\\
Langkah pertama adalah mendefinisikan fungsi yang melakukan iterasi
sebanyak n kali. Kami menambahkan semua parameter ke fungsi ini.
\end{eulercomment}
\begin{eulerprompt}
>function f(M,R,P,n) := iterate("x*(1+P/100)+R",M,n;P,R)[-1]
\end{eulerprompt}
\begin{eulercomment}
Iterasinya sama seperti di atas.

\end{eulercomment}
\begin{eulerformula}
\[
x_{n+1} = x_n \cdot \left(1+ \frac{P}{100}\right) + R
\]
\end{eulerformula}
\begin{eulercomment}
Tapi kami tidak lagi menggunakan nilai global R dalam ekspresi kami.
Fungsi seperti iterate() memiliki trik khusus di Euler. Anda dapat
meneruskan nilai variabel dalam ekspresi sebagai parameter titik koma.
Dalam hal ini P dan R.

Selain itu, kami hanya tertarik pada nilai terakhir. Jadi kita ambil
indeks [-1].

Mari kita coba tes.
\end{eulercomment}
\begin{eulerprompt}
>f(4000,-200,3,31)
\end{eulerprompt}
\begin{euleroutput}
        -0.21 
\end{euleroutput}
\begin{eulercomment}
Sekarang kita bisa menyelesaikan masalah kita.
\end{eulercomment}
\begin{eulerprompt}
>solve("f(4000,-200,x,50)",3)
\end{eulerprompt}
\begin{euleroutput}
         4.43 
\end{euleroutput}
\begin{eulercomment}
Rutin memecahkan memecahkan ekspresi=0 untuk variabel x. Jawabannya
adalah 3,15\% per tahun. Kami mengambil nilai awal 3\% untuk algoritma.
Fungsi solve() selalu membutuhkan nilai awal.\\
Kita dapat menggunakan fungsi yang sama untuk menyelesaikan pertanyaan
berikut: \\
Berapa banyak yang dapat kita keluarkan per tahun sehingga modal awal
habis setelah 20 tahun dengan asumsi tingkat bunga 3\% per tahun.
\end{eulercomment}
\begin{eulerprompt}
>solve("f(4000,x,3,20)",-200)
\end{eulerprompt}
\begin{euleroutput}
      -268.86 
\end{euleroutput}
\begin{eulercomment}
Perhatikan bahwa Anda tidak dapat memecahkan jumlah tahun, karena
fungsi kami mengasumsikan n sebagai nilai integer.

\end{eulercomment}
\eulersubheading{Solusi Simbolik untuk Masalah Suku Bunga}
\begin{eulercomment}
Kita dapat menggunakan bagian simbolik dari Euler untuk mempelajari
masalah tersebut. Pertama kita mendefinisikan fungsi onepay() kita
secara simbolis.
\end{eulercomment}
\begin{eulerprompt}
>function op(M) &= M*q+R; $&op(M)
\end{eulerprompt}
\begin{eulerformula}
\[
R+q\,M
\]
\end{eulerformula}
\begin{eulercomment}
Kita sekarang dapat mengulangi ini.
\end{eulercomment}
\begin{eulerprompt}
>$&op(op(op(op(M)))), $&expand(%)
\end{eulerprompt}
\begin{eulerformula}
\[
q^3\,R+q^2\,R+q\,R+R+q^4\,M
\]
\end{eulerformula}
\eulerimg{0}{images/Tugas Individu Pekan3-4_Muhammad Lutfi Ramadhan_23030630021-105-large.png}
\begin{eulercomment}
Kami melihat sebuah pola. Setelah n periode yang kita miliki

\end{eulercomment}
\begin{eulerformula}
\[
M_n = q^n M + R (1+q+\ldots+q^{n-1}) = q^n M + \frac{q^n-1}{q-1} R
\]
\end{eulerformula}
\begin{eulercomment}
Rumusnya adalah rumus untuk jumlah geometri, yang diketahui Maxima.
\end{eulercomment}
\begin{eulerprompt}
>&sum(q^k,k,0,n-1); $& % = ev(%,simpsum)
\end{eulerprompt}
\begin{eulerformula}
\[
\sum_{k=0}^{n-1}{q^{k}}=\frac{q^{n}-1}{q-1}
\]
\end{eulerformula}
\begin{eulercomment}
Ini agak rumit. Jumlahnya dievaluasi dengan bendera ”simpsum” untuk
menguranginya menjadi hasil bagi.\\
Mari kita membuat fungsi untuk ini.

Rumusnya adalah rumus untuk jumlah geometri, yang diketahui Maxima.
\end{eulercomment}
\begin{eulerprompt}
>function fs(M,R,P,n) &= (1+P/100)^n*M + ((1+P/100)^n-1)/(P/100)*R; $&fs(M,R,P,n)
\end{eulerprompt}
\begin{eulerformula}
\[
\frac{100\,\left(\left(\frac{P}{100}+1\right)^{n}-1\right)\,R}{P}+M  \,\left(\frac{P}{100}+1\right)^{n}
\]
\end{eulerformula}
\begin{eulercomment}
Fungsi tersebut melakukan hal yang sama seperti fungsi f kita
sebelumnya. Tapi itu lebih efektif.
\end{eulercomment}
\begin{eulerprompt}
>longest f(4000,-200,3,31), longest fs(4000,-200,3,31)
\end{eulerprompt}
\begin{euleroutput}
      -0.2142542009293322 
      -0.2142542009369208 
\end{euleroutput}
\begin{eulercomment}
Kita sekarang dapat menggunakannya untuk menanyakan waktu n. Kapan
modal kita habis? Dugaan awal kami adalah 30 tahun.\\
\end{eulercomment}
\begin{eulerttcomment}
 fungsi untuk ini.
\end{eulerttcomment}
\begin{eulercomment}

Rumusnya adalah rumus untuk jumlah geometri, yang diketahui Maxima.
\end{eulercomment}
\begin{eulerprompt}
>solve("fs(4000,-330,3,x)",30)
\end{eulerprompt}
\begin{euleroutput}
        15.29 
\end{euleroutput}
\begin{eulercomment}
Jawaban ini mengatakan bahwa itu akan menjadi negatif setelah 21
tahun.\\
Kita juga dapat menggunakan sisi simbolis Euler untuk menghitung
formula pembayaran.\\
Asumsikan kita mendapatkan pinjaman sebesar K, dan membayar n
pembayaran sebesar R (dimulai setelah tahun pertama) meninggalkan sisa
hutang sebesar Kn (pada saat pembayaran terakhir).\\
Rumus untuk ini jelas.
\end{eulercomment}
\begin{eulerprompt}
>equ &= fs(M,R,P,n)=Mn; $&equ
\end{eulerprompt}
\begin{eulerformula}
\[
\frac{100\,\left(\left(\frac{P}{100}+1\right)^{n}-1\right)\,R}{P}+M  \,\left(\frac{P}{100}+1\right)^{n}={\it Mn}
\]
\end{eulerformula}
\begin{eulercomment}
Biasanya rumus ini diberikan dalam bentuk

\end{eulercomment}
\begin{eulerformula}
\[
i = \frac{P}{100}
\]
\end{eulerformula}
\begin{eulerprompt}
>equ &= (equ with P=100*i); $&equ
\end{eulerprompt}
\begin{eulerformula}
\[
\frac{\left(\left(i+1\right)^{n}-1\right)\,R}{i}+\left(i+1\right)^{  n}\,M={\it Mn}
\]
\end{eulerformula}
\begin{eulercomment}
Kita dapat memecahkan tingkat R secara simbolis.
\end{eulercomment}
\begin{eulerprompt}
>$&solve(equ,R)
\end{eulerprompt}
\begin{eulerformula}
\[
\left[ R=\frac{i\,{\it Mn}-i\,\left(i+1\right)^{n}\,M}{\left(i+1  \right)^{n}-1} \right] 
\]
\end{eulerformula}
\begin{eulercomment}
Seperti yang Anda lihat dari rumus, fungsi ini mengembalikan kesalahan
titik mengambang untuk i=0.\\
Euler tetap merencanakannya.\\
Tentu saja, kami memiliki batas berikut.
\end{eulercomment}
\begin{eulerprompt}
>$&limit(R(4000,0,x,10),x,0)
\end{eulerprompt}
\begin{eulerformula}
\[
\lim_{x\rightarrow 0}{R\left(4000 , 0 , x , 10\right)}
\]
\end{eulerformula}
\begin{eulercomment}
Jelas, tanpa bunga kita harus membayar kembali 10 tarif 500.\\
Persamaan juga dapat diselesaikan untuk n. Kelihatannya lebih bagus,
jika kita menerapkan beberapa penyederhanaan untuk itu.
\end{eulercomment}
\begin{eulerprompt}
>fn &= solve(equ,n) | ratsimp; $&fn
\end{eulerprompt}
\begin{eulerformula}
\[
\left[ n=\frac{\log \left(\frac{R+i\,{\it Mn}}{R+i\,M}\right)}{  \log \left(i+1\right)} \right] 
\]
\end{eulerformula}
\begin{eulercomment}
\begin{eulercomment}
\eulerheading{Penyelesaian Soal Aljabar}
\begin{eulercomment}
Nama: Muhammad Lutfi Ramadhan\\
NIM: 23030630021\\
Kelas: Matematika B 2023

\end{eulercomment}
\eulersubheading{R.2 Exercise Set}
\begin{eulercomment}
1. Sederhanakan bentuk eksponen berikut! (Soal nomor 49)\\
\end{eulercomment}
\begin{eulerformula}
\[
\left(\frac{24a^{10}b^{-8}c^7}{12a^6b^{-3}c^5}\right)^{-5}
\]
\end{eulerformula}
\begin{eulercomment}
Penyelesaian:
\end{eulercomment}
\begin{eulerprompt}
>$&((24*a^10*b^(-8)*c^7) / (12*a^6*b^(-3)*c^5))^(-5)
\end{eulerprompt}
\begin{eulerformula}
\[
\frac{b^{25}}{32\,a^{20}\,c^{10}}
\]
\end{eulerformula}
\begin{eulercomment}
2. Sederhanakan bentuk eksponen berikut! (Soal nomor 50)\\
\end{eulercomment}
\begin{eulerformula}
\[
\left(\frac{125p^{12}q^{-14}r^{22}}{25p^8q^6r^{-15}}\right)^{-4}
\]
\end{eulerformula}
\begin{eulercomment}
Penyelesaian:
\end{eulercomment}
\begin{eulerprompt}
>$& ((125*p^12*q^(-14)*r^22)/(25*p^8*q^6*r^(-15)))^(-4)
\end{eulerprompt}
\begin{eulerformula}
\[
\frac{q^{80}}{625\,p^{16}\,r^{148}}
\]
\end{eulerformula}
\begin{eulercomment}
3. Hitunglah! (Soal nomor 90)\\
\end{eulercomment}
\begin{eulerformula}
\[
2^6*2^{-3}/2^{10}/2^{-8}
\]
\end{eulerformula}
\begin{eulercomment}
Penyelesaian:
\end{eulercomment}
\begin{eulerprompt}
>2^6*2^(-3)/2^10/2^(-8)
\end{eulerprompt}
\begin{euleroutput}
         2.00 
\end{euleroutput}
\begin{eulercomment}
4. Hitunglah! (Soal nomor 91)\\
\end{eulercomment}
\begin{eulerformula}
\[
\left(\frac{4(8-6)^2-4*3+2*8}{3^1+19^0}\right)
\]
\end{eulerformula}
\begin{eulercomment}
Penyelesaian:
\end{eulercomment}
\begin{eulerprompt}
>(4*(8-6)^2 - 4*3 + 2*8)/(3^1+19^0)
\end{eulerprompt}
\begin{euleroutput}
         5.00 
\end{euleroutput}
\begin{eulercomment}
5. Hitunglah! (Soal nomor 92)\\
\end{eulercomment}
\begin{eulerformula}
\[
\left(\frac{[4(8-6)^2+4](3-2*8)}{2^2(2^5+5)}\right)
\]
\end{eulerformula}
\begin{eulercomment}
Penyelesaian:
\end{eulercomment}
\begin{eulerprompt}
>((4*(8-6)^2+4)*(3-2*8))/(2^2*(2^5+5))
\end{eulerprompt}
\begin{euleroutput}
        -1.76 
\end{euleroutput}
\begin{eulercomment}
\end{eulercomment}
\eulersubheading{R.3 Exercise Set Perform the Indicated Operations}
\begin{eulercomment}
1. Lakukan operasi yang ditunjukkan! (Soal nomor 28)\\
\end{eulercomment}
\begin{eulerformula}
\[
(z+6)^2
\]
\end{eulerformula}
\begin{eulercomment}
Penyelesaian
\end{eulercomment}
\begin{eulerprompt}
>$&showev('expand((z+6)^2))
\end{eulerprompt}
\begin{eulerformula}
\[
{\it expand}\left(\left(z+6\right)^2\right)=z^2+12\,z+36
\]
\end{eulerformula}
\begin{eulercomment}
2. Lakukan operasi yang ditunjukkan! (Soal 13)\\
\end{eulercomment}
\begin{eulerformula}
\[
(3a^2)(-7a^4)
\]
\end{eulerformula}
\begin{eulercomment}
Penyelesaian:
\end{eulercomment}
\begin{eulerprompt}
>$&showev('expand((3*a^2)*(-7)*a^4))
\end{eulerprompt}
\begin{eulerformula}
\[
{\it expand}\left(-21\,a^6\right)=-21\,a^6
\]
\end{eulerformula}
\begin{eulercomment}
3. Lakukan operasi yang ditunjukkan! (Soal nomor 21)\\
\end{eulercomment}
\begin{eulerformula}
\[
(x+6)(x+3)
\]
\end{eulerformula}
\begin{eulercomment}
Penyelesaian:
\end{eulercomment}
\begin{eulerprompt}
>$&showev('expand((x+6)*(x+3)))
\end{eulerprompt}
\begin{eulerformula}
\[
{\it expand}\left(\left(x+3\right)\,\left(x+6\right)\right)=x^2+9\,  x+18
\]
\end{eulerformula}
\begin{eulercomment}
4. Lakukan operasi yang ditunjukkan! (Soal nomor 40)\\
\end{eulercomment}
\begin{eulerformula}
\[
(2x-7)(2x+7)
\]
\end{eulerformula}
\begin{eulercomment}
Penyelesaian:
\end{eulercomment}
\begin{eulerprompt}
>$&showev('expand((2*x-7)*(2*x+7)))
\end{eulerprompt}
\begin{eulerformula}
\[
{\it expand}\left(\left(2\,x-7\right)\,\left(2\,x+7\right)\right)=4  \,x^2-49
\]
\end{eulerformula}
\begin{eulercomment}
5. Lakukan operasi yang ditunjukkan! (Soal nomor 29)\\
\end{eulercomment}
\begin{eulerformula}
\[
(y-5)^2
\]
\end{eulerformula}
\begin{eulercomment}
Penyelesaian:
\end{eulercomment}
\begin{eulerprompt}
>$&showev('expand((y-5)^2))
\end{eulerprompt}
\begin{eulerformula}
\[
{\it expand}\left(\left(y-5\right)^2\right)=y^2-10\,y+25
\]
\end{eulerformula}
\begin{eulercomment}
\end{eulercomment}
\eulersubheading{R.4 Exercise Set Faktor Trinomial}
\begin{eulercomment}
1. Faktorkan jumlah atau selisih kubus! (Soal nomor 69)\\
\end{eulercomment}
\begin{eulerformula}
\[
m^3-216
\]
\end{eulerformula}
\begin{eulercomment}
Penyelesaian:
\end{eulercomment}
\begin{eulerprompt}
>$&solve(m^3-216)
\end{eulerprompt}
\begin{eulerformula}
\[
\left[ m=3^{\frac{3}{2}}\,i-3 , m=-3^{\frac{3}{2}}\,i-3 , m=6   \right] 
\]
\end{eulerformula}
\begin{eulercomment}
2. Faktorkan jumlah atau selisih kubus! (Soal nomor 70)\\
\end{eulercomment}
\begin{eulerformula}
\[
n^3+1
\]
\end{eulerformula}
\begin{eulercomment}
Penyelesaian:
\end{eulercomment}
\begin{eulerprompt}
>$&solve(n^3+1)
\end{eulerprompt}
\begin{eulerformula}
\[
\left[ n=\frac{1-\sqrt{3}\,i}{2} , n=\frac{\sqrt{3}\,i+1}{2} , n=-1   \right] 
\]
\end{eulerformula}
\begin{eulercomment}
3. Faktorkan kuadrat dari sebuah binomial! (Soal nomor 58)\\
\end{eulercomment}
\begin{eulerformula}
\[
y^2-6y+9
\]
\end{eulerformula}
\begin{eulercomment}
Penyelesaian:
\end{eulercomment}
\begin{eulerprompt}
>$&solve(y^2-6*y+9)
\end{eulerprompt}
\begin{eulerformula}
\[
\left[ y=3 \right] 
\]
\end{eulerformula}
\begin{eulercomment}
4. Faktorkan kuadrat dari sebuah binomial! (Soal nomor 59)\\
\end{eulercomment}
\begin{eulerformula}
\[
9z^2-12z+4
\]
\end{eulerformula}
\begin{eulercomment}
Penyelesaian:
\end{eulercomment}
\begin{eulerprompt}
>$&solve(9*z^2-12*z+4)
\end{eulerprompt}
\begin{eulerformula}
\[
\left[ z=\frac{2}{3} \right] 
\]
\end{eulerformula}
\begin{eulercomment}
5. Faktorkan sepenuhnya! (Soal nomor 110)\\
\end{eulercomment}
\begin{eulerformula}
\[
x^3-x^2-x+1
\]
\end{eulerformula}
\begin{eulercomment}
Penyelesaian:
\end{eulercomment}
\begin{eulerprompt}
>$&solve(x^3-x^2-x+1)
\end{eulerprompt}
\begin{eulerformula}
\[
\left[ x=-1 , x=1 \right] 
\]
\end{eulerformula}
\begin{eulercomment}
\end{eulercomment}
\eulersubheading{R.5 The Basics of Equation Solving}
\begin{eulercomment}
1. Selesaikan! (Soal nomor 37)\\
\end{eulercomment}
\begin{eulerformula}
\[
x^2+5x=0
\]
\end{eulerformula}
\begin{eulercomment}
Penyelesaian:
\end{eulercomment}
\begin{eulerprompt}
>$&solve(x^2+5*x=0,x)
\end{eulerprompt}
\begin{eulerformula}
\[
\left[ x=-5 , x=0 \right] 
\]
\end{eulerformula}
\begin{eulercomment}
2. Selesaikan! (Soal nomor 48)\\
\end{eulercomment}
\begin{eulerformula}
\[
6x^2-7x=10
\]
\end{eulerformula}
\begin{eulercomment}
Penyelesaian:
\end{eulercomment}
\begin{eulerprompt}
>$&solve(6*x^2-7*x=10,x)
\end{eulerprompt}
\begin{eulerformula}
\[
\left[ x=-\frac{5}{6} , x=2 \right] 
\]
\end{eulerformula}
\begin{eulercomment}
3. Selesaikan! (Soal nomor 56)\\
\end{eulercomment}
\begin{eulerformula}
\[
t^2=25
\]
\end{eulerformula}
\begin{eulercomment}
Penyelesaian:
\end{eulercomment}
\begin{eulerprompt}
>$&solve(t^2=25,t)
\end{eulerprompt}
\begin{eulerformula}
\[
\left[ t=-5 , t=5 \right] 
\]
\end{eulerformula}
\begin{eulercomment}
4. Selesaikan! (Soal nomor 41)\\
\end{eulercomment}
\begin{eulerformula}
\[
x^2+100=20x
\]
\end{eulerformula}
\begin{eulercomment}
Penyelesaian:
\end{eulercomment}
\begin{eulerprompt}
>$&solve(x^2+100=20*x,x)
\end{eulerprompt}
\begin{eulerformula}
\[
\left[ x=10 \right] 
\]
\end{eulerformula}
\begin{eulercomment}
5. Selesaikan! (Soal nomor 55)\\
\end{eulercomment}
\begin{eulerformula}
\[
z^2=144
\]
\end{eulerformula}
\begin{eulercomment}
Penyelesaian:
\end{eulercomment}
\begin{eulerprompt}
>$&solve(z^2=144)
\end{eulerprompt}
\begin{eulerformula}
\[
\left[ z=-12 , z=12 \right] 
\]
\end{eulerformula}
\end{eulernotebook}
\end{document}
