\documentclass[a4paper,10pt]{article}
\usepackage{eumat}

\begin{document}
\begin{eulernotebook}
\begin{eulercomment}
Nama : Muhammad Lutfi Ramadhan\\
Kelas : Matematika B 2023\\
NIM : 23030630021

\begin{eulercomment}
\eulerheading{Menggambar Grafik 2D dengan EMT}
\begin{eulercomment}
Notebook ini menjelaskan tentang cara menggambar berbagaikurva dan
grafik 2D dengan software EMT. EMT menyediakan fungsi plot2d() untuk
menggambar berbagai kurva dan grafik dua dimensi (2D).\\
\end{eulercomment}
\eulersubheading{Plot Dasar}
\begin{eulercomment}
Ada beberapa fungsi dasar untuk plot. Ada koordinat layar, yang selalu
berkisar dari 0 hingga 1024 pada setiap sumbu, tidak peduli apakah
layarnya berbentuk persegi atau tidak. Ada juga koordinat plot, yang
dapat diatur dengan setplot(). Pemetaan antara koordinat tergantung
pada jendela plot saat ini. Sebagai contoh, jendela shrinkwindow()
default menyisakan ruang untuk label sumbu dan judul plot.


Dalam contoh ini, kita hanya menggambar beberapa garis acak dengan
berbagai warna. Untuk detail mengenai fungsi ini, pelajari fungsi inti
EMT.
\end{eulercomment}
\begin{eulerprompt}
>clg; // hapus layar
>window(0,0,1024,1024); // gunakan seluruh jendela
>setplot(0,1,0,1); // atur koordinat plot
>hold on; // aktifkan mode overwrite
>n=100; X=random(n,2); Y=random(n,2);  // mendapatkan poin acak
>colors=rgb(random(n),random(n),random(n)); // mendapatkan warna acak
>loop 1 to n; color(colors[#]); plot(X[#],Y[#]); end; // plot
>hold off; // akhiri mode overwrite
>insimg; // masukkan gambar ke dalam notebook
>reset;
\end{eulerprompt}
\begin{eulercomment}
Ini penting untuk menahan grafik karena perintah plot() akan menghapus
jendela plot.

Untuk menghapus semuanya kita menggunakan reset().

Untuk menampilkan gambar hasil plot di layar notebook, perintah
plot2d() dapat diakhiri dengan titik dua (:). Cara lain adalah
perintah plot2d() diakhiri dengan titik koma (;), kemudian menggunakan
perintah insimg() untuk menampilkan gambar hasil plot.

Sebagai contoh lain, kita menggambar sebuah plot sebagai inset
(sisipan) di dalam plot lain. Ini dilakukan dengan mendefinisikan
jendela plot yang lebih kecil. Perlu dicatat bahwa jendela ini tidak
menyediakan ruang untuk label sumbu di luar jendela plot, sehingga
kita perlu menambahkan margin sesuai kebutuhan. Kita menyimpan dan
mengembalikan jendela penuh, serta menahan plot saat kita menggambar
inset.
\end{eulercomment}
\begin{eulerprompt}
>plot2d("x^3-x");
>xw=200; yw=100; ww=300; hw=300;
>ow=window();
>window(xw,yw,xw+ww,yw+hw);
>hold on;
>barclear(xw-50,yw-10,ww+60,ww+60);
>plot2d("x^4-x",grid=6):
>hold off;
>window(ow);
\end{eulerprompt}
\begin{eulercomment}
Plot dengan beberapa gambar dicapai dengan cara yang sama. Ada fungsi
utilitas figure() untuk ini.

\end{eulercomment}
\eulersubheading{Aspek Plot}
\begin{eulercomment}
Plot default menggunakan jendela plot persegi. Kita dapat mengubah ini
dengan fungsi aspect(). Jangan lupa untuk mengatur ulang aspek
setelahnya. Kita juga dapat mengubah pengaturan ini dalam menu dengan
"Set Aspect" ke rasio aspek tertentu atau ke ukuran grafik saat ini.
Anda juga dapat mengubahnya untuk satu plot. Untuk ini, ukuran area
plot saat ini diubah, dan jendela diatur agar label memiliki ruang
yang cukup.
\end{eulercomment}
\begin{eulerprompt}
>aspect(2); // rasio panjang dan lebar 2:1
>plot2d(["sin(x)","cos(x)"],0,2pi):
>aspect();
>reset;
\end{eulerprompt}
\begin{eulercomment}
Fungsi reset() mengembalikan plot ke pengaturan default termasuk rasio
aspek.\\
\begin{eulercomment}
\eulerheading{Plot 2D dalam Euler}
\begin{eulercomment}
EMT Math Toolbox memiliki plot dalam 2D, baik untuk data maupun
fungsi. EMT menggunakan fungsi plot2d. Fungsi ini dapat memplotkan
fungsi dan data.

Dimungkinkan untuk memplot di Maxima menggunakan Gnuplot atau di
Python menggunakan Math Plot Lib.

Euler dapat memplot plot 2D dari:

- Ekspresi,\\
- Fungsi, variabel, atau kurva parameterisasi,\\
- Vektor nilai x-y,\\
- Sekumpulan titik di bidang,\\
- Kurva implisit dengan level atau area level,\\
- Fungsi kompleks,

Gaya plot mencakup berbagai gaya untuk garis dan titik, plot batang,
dan plot berbayang.\\
\begin{eulercomment}
\eulerheading{Plot Ekspresi atau Variabel}
\begin{eulercomment}
Sebuah ekspresi tunggal dalam "x" (misalnya "4*x\textasciicircum{}2") atau nama sebuah
fungsi (misalnya "f") akan menghasilkan grafik fungsi tersebut.

Berikut adalah contoh paling dasar, yang menggunakan rentang default
dan menetapkan rentang y yang sesuai untuk memuat plot fungsi
tersebut.

atatan: Jika Anda mengakhiri baris perintah dengan titik dua (:), plot
akan dimasukkan ke dalam jendela teks. Jika tidak, tekan TAB untuk
melihat plot jika jendela plot tertutup.
\end{eulercomment}
\begin{eulerprompt}
>plot2d("x^2"):
>aspect(1.5); plot2d("x^3-x"):
>a:=5.6; plot2d("exp(-a*x^2)/a"); insimg(30); // menampilkan gambar hasil plot setinggi 25 baris
\end{eulerprompt}
\begin{eulercomment}
Dari beberapa contoh sebelumnya Anda dapat melihat bahwa aslinya
gambar plot menggunakan sumbu X dengan rentang nilai dari -2 sampai
dengan 2. Untuk mengubah rentang nilai X dan Y, Anda dapat menambahkan
nilai-nilai batas X (dan Y) di belakang ekspresi yang digambar.

Rentang plot diatur dengan parameter yang ditetapkan berikut:

- a,b: rentang x (default -2,2)\\
- c,d: rentang y (default: skala dengan nilai)\\
- r: alternatifnya, radius di sekitar pusat plot\\
- cx,cy: koordinat pusat plot (default 0,0)
\end{eulercomment}
\begin{eulerprompt}
>plot2d("x^3-x",-1,2):
>plot2d("sin(x)",-2*pi,2*pi): // plot sin(x) pada interval [-2pi, 2pi]
>plot2d("cos(x)","sin(3*x)",xmin=0,xmax=2pi):
\end{eulerprompt}
\begin{eulercomment}
Alternatif untuk titik dua (:) adalah perintah insimg(lines), yang
menyisipkan plot sesuai jumlah baris teks yang ditentukan.

Dalam opsi lainnya, plot dapat diatur agar muncul:\\
- di jendela yang dapat diubah ukurannya\\
- di jendela notebook

Lebih banyak gaya dapat dicapai dengan perintah plot khusus.

Untuk membagi jendela menjadi beberapa plot, gunakan perintah
figure(). Dalam contoh berikut, kami memplotkan x\textasciicircum{}1 hingga x\textasciicircum{}4 ke
dalam 4 bagian jendela. figure(0) mengatur ulang jendela ke default.
\end{eulercomment}
\begin{eulerprompt}
>reset;
>figure(2,2); ...
>for n=1 to 4; figure(n); plot2d("x^"+n); end; ...
>figure(0):
\end{eulerprompt}
\begin{eulercomment}
Dalam plot2d(), ada gaya alternatif yang tersedia dengan grid=x. Untuk
gambaran, kami menunjukkan berbagai gaya grid dalam satu figur.
\end{eulercomment}
\begin{eulerprompt}
>figure(3,3); ...
>for k=1:9; figure(k); plot2d("x^3-x",-2,1,grid=k); end; ...
>figure(0):
\end{eulerprompt}
\begin{eulercomment}
Jika argumen untuk plot2d() adalah ekspresi yang diikuti oleh empat
angka, angka-angka tersebut adalah rentang x dan y untuk plot.

Sebagai alternatif, a, b, c, d dapat ditentukan sebagai parameter yang
ditetapkan, misalnya a=... dll.

Pada contoh berikut, kami mengubah gaya grid, menambahkan label, dan
menggunakan label vertikal untuk sumbu y.
\end{eulercomment}
\begin{eulerprompt}
>aspect(1.5); plot2d("sin(x)",0,2pi,-1.2,1.2,grid=3,xl="x",yl="sin(x)"):
>plot2d("sin(x)+cos(2*x)",0,4pi):
\end{eulerprompt}
\begin{eulercomment}
Gambar yang dihasilkan dari menyisipkan plot ke dalam jendela teks
disimpan di direktori yang sama dengan notebook, secara default di
subdirektori bernama "images". Gambar-gambar ini juga digunakan dalam
ekspor HTML. Anda dapat menandai gambar apa pun dan menyalinnya ke
clipboard dengan Ctrl-C. Tentu saja, Anda juga dapat mengekspor grafik
saat ini dengan fungsi di menu File.

Fungsi atau ekspresi dalam plot2d dievaluasi secara adaptif. Untuk
kecepatan lebih tinggi, nonaktifkan plot adaptif dengan \textless{}adaptive dan
tentukan jumlah subinterval dengan n=.... Ini hanya diperlukan dalam
kasus yang jarang.
\end{eulercomment}
\begin{eulerprompt}
>plot2d("sign(x)*exp(-x^2)",-1,1,<adaptive,n=10000):
>plot2d("x^x",r=1.2,cx=1,cy=1):
\end{eulerprompt}
\begin{eulercomment}
Perlu dicatat bahwa x\textasciicircum{}x tidak didefinisikan untuk x\textless{}=0. Fungsi plot2d
menangkap kesalahan ini, dan mulai memplot begitu fungsi terdefinisi.
Ini berlaku untuk semua fungsi yang mengembalikan NAN di luar rentang
definisinya.
\end{eulercomment}
\begin{eulerprompt}
>plot2d("log(x)",-0.1,2):
\end{eulerprompt}
\begin{eulercomment}
Parameter square=true (atau \textgreater{}square) memilih rentang y secara otomatis
sehingga hasilnya adalah jendela plot persegi. Secara default, Euler
menggunakan ruang persegi di dalam jendela plot.
\end{eulercomment}
\begin{eulerprompt}
>plot2d("x^3-x",>square):
>plot2d(''integrate("sin(x)*exp(-x^2)",0,x)'',0,2): // plot integral
\end{eulerprompt}
\begin{eulercomment}
Jika Anda memerlukan lebih banyak ruang untuk label sumbu y, panggil
shrinkwindow() dengan parameter yang lebih kecil, atau atur nilai
positif untuk "smaller" di plot2d().
\end{eulercomment}
\begin{eulerprompt}
>plot2d("gamma(x)",1,10,yl="y-values",smaller=6,<vertical):
\end{eulerprompt}
\begin{eulercomment}
Ekspresi simbolik juga dapat digunakan, karena mereka disimpan sebagai
ekspresi string sederhana.
\end{eulercomment}
\begin{eulerprompt}
>x=linspace(0,2pi,1000); plot2d(sin(5x),cos(7x)):
>a:=5.6; expr &= exp(-a*x^2)/a; // define expression
>plot2d(expr,-2,2): // plot from -2 to 2
>plot2d(expr,r=1,thickness=2): // plot in a square around (0,0)
>plot2d(&diff(expr,x),>add,style="--",color=red): // add another plot
>plot2d(&diff(expr,x,2),a=-2,b=2,c=-2,d=1): // plot in rectangle
>plot2d(&diff(expr,x),a=-2,b=2,>square): // keep plot square
>plot2d("x^2",0,1,steps=1,color=red,n=10):
>plot2d("x^2",>add,steps=2,color=blue,n=10):
\end{eulerprompt}
\eulerheading{Fungsi dengan Satu Parameter}
\begin{eulercomment}
Fungsi plot paling penting untuk plot planar adalah plot2d(). Fungsi
ini diimplementasikan dalam bahasa Euler di file "plot.e", yang dimuat
saat program dimulai.

Berikut beberapa contoh menggunakan fungsi. Seperti biasa dalam EMT,
fungsi yang bekerja untuk fungsi atau ekspresi lainnya, Anda dapat
memberikan parameter tambahan (selain x) yang bukan variabel global
kepada fungsi dengan parameter titik koma atau dengan koleksi
panggilan.
\end{eulercomment}
\begin{eulerprompt}
>function f(x,a) := x^2/a+a*x^2-x; // define a function
>a=0.3; plot2d("f",0,1;a): // plot with a=0.3
>plot2d("f",0,1;0.4): // plot with a=0.4
>plot2d(\{\{"f",0.2\}\},0,1): // plot with a=0.2
>plot2d(\{\{"f(x,b)",b=0.1\}\},0,1): // plot with 0.1
>function f(x) := x^3-x; ...
>plot2d("f",r=1):
\end{eulerprompt}
\begin{eulercomment}
Berikut ringkasan fungsi yang diterima:

- Ekspresi atau ekspresi simbolik dalam x\\
- Fungsi atau fungsi simbolik berdasarkan nama seperti "f"\\
- Fungsi simbolik hanya dengan nama f

Fungsi plot2d() juga menerima fungsi simbolik. Untuk fungsi simbolik,
nama saja sudah cukup.
\end{eulercomment}
\begin{eulerprompt}
>function f(x) &= diff(x^x,x)
\end{eulerprompt}
\begin{euleroutput}
  
                              x
                             x  (log(x) + 1)
  
\end{euleroutput}
\begin{eulerprompt}
>plot2d(f,0,2):
\end{eulerprompt}
\begin{eulercomment}
Tentu saja, untuk ekspresi atau ekspresi simbolik, nama variabel saja
cukup untuk memplotkannya.
\end{eulercomment}
\begin{eulerprompt}
>expr &= sin(x)*exp(-x)
\end{eulerprompt}
\begin{euleroutput}
  
                                - x
                               E    sin(x)
  
\end{euleroutput}
\begin{eulerprompt}
>plot2d(expr,0,3pi):
>function f(x) &= x^x;
>plot2d(f,r=1,cx=1,cy=1,color=blue,thickness=2);
>plot2d(&diff(f(x),x),>add,color=red,style="-.-"):
\end{eulerprompt}
\begin{eulercomment}
Untuk gaya garis, ada berbagai opsi:\\
- style="...". Pilih dari "-", "–", "-.", ".", ".-.", "-.-".\\
- color: Lihat di bawah untuk warna.\\
- thickness: Default adalah 1.

Warna dapat dipilih sebagai salah satu warna default, atau sebagai
warna RGB:\\
- 0..15: indeks warna default\\
- constanta warna: putih, hitam, merah, hijau, biru, cyan, olive,
lightgray, gray, darkgray, orange, lightgreen, turquoise, lightblue,
lightorange, kuning\\
- rgb(red,green,blue): parameter adalah real dalam [0,1].
\end{eulercomment}
\begin{eulerprompt}
>plot2d("exp(-x^2)",r=2,color=red,thickness=3,style="--"):
\end{eulerprompt}
\begin{eulercomment}
Berikut tampilan warna yang telah ditetapkan dalam EMT:
\end{eulercomment}
\begin{eulerprompt}
>aspect(2); columnsplot(ones(1,16),lab=0:15,grid=0,color=0:15):
\end{eulerprompt}
\begin{eulercomment}
Namun, Anda dapat menggunakan warna apapun.
\end{eulercomment}
\begin{eulerprompt}
>columnsplot(ones(1,16),grid=0,color=rgb(0,0,linspace(0,1,15))):
\end{eulerprompt}
\eulersubheading{Menggambar Beberapa Kurva pada bidang koordinat yang sama}
\begin{eulercomment}
Memplot lebih dari satu fungsi ke dalam satu jendela dapat dilakukan
dengan beberapa cara. Salah satu metodenya adalah menggunakan \textgreater{}add
untuk beberapa panggilan ke plot2d di semua, kecuali panggilan
pertama. Kami telah menggunakan fitur ini dalam contoh sebelumnya.
\end{eulercomment}
\begin{eulerprompt}
>aspect(); plot2d("cos(x)",r=2,grid=6); plot2d("x",style=".",>add):
>aspect(1.5); plot2d("sin(x)",0,2pi); plot2d("cos(x)",color=blue,style="--",>add):
\end{eulerprompt}
\begin{eulercomment}
Salah satu kegunaan \textgreater{}add adalah untuk menambahkan titik pada kurva.
\end{eulercomment}
\begin{eulerprompt}
>plot2d("sin(x)",0,pi); plot2d(2,sin(2),>points,>add):
\end{eulerprompt}
\begin{eulercomment}
Kami menambahkan titik perpotongan dengan label (di posisi "cl" untuk
tengah kiri), dan menyisipkan hasilnya ke dalam notebook. Kami juga
menambahkan judul pada plot.
\end{eulercomment}
\begin{eulerprompt}
>plot2d(["cos(x)","x"],r=1.1,cx=0.5,cy=0.5, ...
>  color=[black,blue],style=["-","."], ...
>  grid=1);
>x0=solve("cos(x)-x",1);  ...
>  plot2d(x0,x0,>points,>add,title="Intersection Demo");  ...
>  label("cos(x) = x",x0,x0,pos="cl",offset=20):
\end{eulerprompt}
\begin{eulercomment}
Dalam demo berikut, kami memplot fungsi sinc(x)=sin(x)/x dan ekspansi
Taylor ke-8 dan ke-16. Kami menghitung ekspansi ini menggunakan Maxima
melalui ekspresi simbolik. Plot ini dilakukan dalam perintah
multi-baris berikut dengan tiga panggilan ke plot2d(). Panggilan kedua
dan ketiga memiliki flag \textgreater{}add, yang membuat plot menggunakan rentang
sebelumnya.

Kami menambahkan kotak label untuk menjelaskan fungsi-fungsi tersebut.
\end{eulercomment}
\begin{eulerprompt}
>$taylor(sin(x)/x,x,0,4)
>plot2d("sinc(x)",0,4pi,color=green,thickness=2); ...
>  plot2d(&taylor(sin(x)/x,x,0,8),>add,color=blue,style="--"); ...
>  plot2d(&taylor(sin(x)/x,x,0,16),>add,color=red,style="-.-"); ...
>  labelbox(["sinc","T8","T16"],styles=["-","--","-.-"], ...
>    colors=[black,blue,red]):
\end{eulerprompt}
\begin{eulercomment}
Dalam contoh berikut, kami menghasilkan polinomial Bernstein.

\end{eulercomment}
\begin{eulerformula}
\[
B_i(x) = \binom{n}{i} x^i (1-x)^{n-i}
\]
\end{eulerformula}
\begin{eulerprompt}
>plot2d("(1-x)^10",0,1); // plot first function
>for i=1 to 10; plot2d("bin(10,i)*x^i*(1-x)^(10-i)",>add); end;
>insimg;
\end{eulerprompt}
\begin{eulercomment}
Metode kedua adalah menggunakan pasangan matriks nilai x dan matriks
nilai y dengan ukuran yang sama.

Kami menghasilkan matriks nilai dengan satu Polinomial Bernstein di
setiap baris. Untuk ini, kami cukup menggunakan vektor kolom dari i.
Lihat pengantar tentang bahasa matriks untuk mempelajari detail lebih
lanjut.
\end{eulercomment}
\begin{eulerprompt}
>x=linspace(0,1,500);
>n=10; k=(0:n)'; // n is row vector, k is column vector
>y=bin(n,k)*x^k*(1-x)^(n-k); // y is a matrix then
>plot2d(x,y):
\end{eulerprompt}
\begin{eulercomment}
Perlu diperhatikan bahwa parameter color dapat berupa vektor. Kemudian
setiap warna digunakan untuk setiap baris matriks.
\end{eulercomment}
\begin{eulerprompt}
>x=linspace(0,1,200); y=x^(1:10)'; plot2d(x,y,color=1:10):
\end{eulerprompt}
\begin{eulercomment}
Metode lain adalah menggunakan vektor ekspresi (string). Anda kemudian
dapat menggunakan array warna, array gaya, dan array ketebalan dengan
panjang yang sama.
\end{eulercomment}
\begin{eulerprompt}
>plot2d(["sin(x)","cos(x)"],0,2pi,color=4:5): 
>plot2d(["sin(x)","cos(x)"],0,2pi): // plot vector of expressions
\end{eulerprompt}
\begin{eulercomment}
Kita bisa mendapatkan vektor seperti itu dari Maxima menggunakan
makelist() dan mxm2str().
\end{eulercomment}
\begin{eulerprompt}
>v &= makelist(binomial(10,i)*x^i*(1-x)^(10-i),i,0,10) // make list
\end{eulerprompt}
\begin{euleroutput}
  
                 10            9              8  2             7  3
         [(1 - x)  , 10 (1 - x)  x, 45 (1 - x)  x , 120 (1 - x)  x , 
             6  4             5  5             4  6             3  7
  210 (1 - x)  x , 252 (1 - x)  x , 210 (1 - x)  x , 120 (1 - x)  x , 
            2  8              9   10
  45 (1 - x)  x , 10 (1 - x) x , x  ]
  
\end{euleroutput}
\begin{eulerprompt}
>mxm2str(v) // get a vector of strings from the symbolic vector
\end{eulerprompt}
\begin{euleroutput}
  (1-x)^10
  10*(1-x)^9*x
  45*(1-x)^8*x^2
  120*(1-x)^7*x^3
  210*(1-x)^6*x^4
  252*(1-x)^5*x^5
  210*(1-x)^4*x^6
  120*(1-x)^3*x^7
  45*(1-x)^2*x^8
  10*(1-x)*x^9
  x^10
\end{euleroutput}
\begin{eulerprompt}
>plot2d(mxm2str(v),0,1): // plot functions
\end{eulerprompt}
\begin{eulercomment}
Alternatif lain adalah menggunakan bahasa matriks dari Euler.

Jika sebuah ekspresi menghasilkan matriks fungsi, dengan satu fungsi
di setiap baris, semua fungsi ini akan dipplot ke dalam satu plot.
Untuk ini, gunakan vektor parameter dalam bentuk vektor kolom.

Jika ditambahkan array warna, ini akan digunakan untuk setiap baris
plot.
\end{eulercomment}
\begin{eulerprompt}
>n=(1:10)'; plot2d("x^n",0,1,color=1:10):
\end{eulerprompt}
\begin{eulercomment}
Ekspresi dan fungsi satu baris dapat melihat variabel global.

Jika Anda tidak bisa menggunakan variabel global, Anda perlu
menggunakan fungsi dengan parameter tambahan, dan meneruskan parameter
ini sebagai parameter titik koma. Pastikan semua parameter yang
ditetapkan berada di akhir perintah plot2d.

Dalam contoh ini, kami meneruskan a=5 ke fungsi f, yang kami plot dari
-10 hingga 10.
\end{eulercomment}
\begin{eulerprompt}
>function f(x,a) := 1/a*exp(-x^2/a); ...
>plot2d("f",-10,10;5,thickness=2,title="a=5"):
\end{eulerprompt}
\begin{eulercomment}
Sebagai alternatif, gunakan koleksi dengan nama fungsi dan semua
parameter tambahan. Daftar khusus ini disebut koleksi panggilan, dan
ini adalah cara yang disukai untuk meneruskan argumen ke fungsi yang
diteruskan sebagai argumen ke fungsi lain.

Dalam contoh berikut, kami menggunakan loop untuk memplot beberapa
fungsi.
\end{eulercomment}
\begin{eulerprompt}
>plot2d(\{\{"f",1\}\},-10,10); ...
>for a=2:10; plot2d(\{\{"f",a\}\},>add); end:
\end{eulerprompt}
\begin{eulercomment}
Hasil yang sama dapat dicapai dengan menggunakan bahasa matriks EMT.
Setiap baris matriks f(x,a) adalah satu fungsi. Selain itu, kita dapat
mengatur warna untuk setiap baris matriks. Klik dua kali pada fungsi
getspectral() untuk penjelasan.
\end{eulercomment}
\begin{eulerprompt}
>x=-10:0.01:10; a=(1:10)'; plot2d(x,f(x,a),color=getspectral(a/10)):
\end{eulerprompt}
\eulersubheading{Label Teks}
\begin{eulercomment}
Dekorasi sederhana dapat berupa:\\
- judul dengan title="..."\\
- label x dan y dengan xl="...", yl="..."\\
- label teks lain dengan label("...",x,y)

Perintah label akan memplot ke plot saat ini di koordinat plot (x,y).
Ia dapat mengambil argumen posisi.
\end{eulercomment}
\begin{eulerprompt}
>plot2d("x^3-x",-1,2,title="y=x^3-x",yl="y",xl="x"):
>expr := "log(x)/x"; ...
>  plot2d(expr,0.5,5,title="y="+expr,xl="x",yl="y"); ...
>  label("(1,0)",1,0); label("Max",E,expr(E),pos="lc"):
\end{eulerprompt}
\begin{eulercomment}
Ada juga fungsi labelbox(), yang dapat menampilkan fungsi dan teks. Ia
menerima vektor string dan warna, satu item untuk setiap fungsi.
\end{eulercomment}
\begin{eulerprompt}
>function f(x) &= x^2*exp(-x^2);  ...
>plot2d(&f(x),a=-3,b=3,c=-1,d=1);  ...
>plot2d(&diff(f(x),x),>add,color=blue,style="--"); ...
>labelbox(["function","derivative"],styles=["-","--"], ...
>   colors=[black,blue],w=0.4):
\end{eulerprompt}
\begin{eulercomment}
Kotak ini diposisikan di sudut kanan atas secara default, tetapi \textgreater{}left
akan memposisikannya di sudut kiri atas. Anda dapat memindahkannya ke
tempat mana saja yang Anda inginkan. Posisi jangkar adalah sudut kanan
atas kotak, dan angka-angka adalah pecahan dari ukuran jendela grafik.
Lebarnya otomatis.

Untuk plot titik, kotak label juga berfungsi. Tambahkan parameter
\textgreater{}points, atau vektor flag, satu untuk setiap label. Dalam contoh
berikut, hanya ada satu fungsi. Jadi kita bisa menggunakan string
alih-alih vektor string. Kami menetapkan warna teks ke hitam untuk
contoh ini.
\end{eulercomment}
\begin{eulerprompt}
>n=10; plot2d(0:n,bin(n,0:n),>addpoints); ...
>labelbox("Binomials",styles="[]",>points,x=0.1,y=0.1, ...
>tcolor=black,>left):
\end{eulerprompt}
\begin{eulercomment}
Gaya plot ini juga tersedia dalam statplot(). Seperti dalam plot2d(),
warna dapat diatur untuk setiap baris plot. Ada lebih banyak plot
khusus untuk tujuan statistik (lihat tutorial tentang statistik).
\end{eulercomment}
\begin{eulerprompt}
>statplot(1:10,random(2,10),color=[red,blue]):
\end{eulerprompt}
\begin{eulercomment}
Fitur serupa adalah fungsi textbox().

Lebarnya secara default adalah lebar maksimal dari garis teks, tetapi
bisa diatur oleh pengguna juga.
\end{eulercomment}
\begin{eulerprompt}
>function f(x) &= exp(-x)*sin(2*pi*x); ...
>plot2d("f(x)",0,2pi); ...
>textbox(latex("\(\backslash\)text\{Example of a damped oscillation\}\(\backslash\) f(x)=e^\{-x\}sin(2\(\backslash\)pi x)"),w=0.85):
\end{eulerprompt}
\begin{eulercomment}
Label teks, judul, kotak label, dan teks lainnya dapat berisi string
Unicode (lihat sintaks EMT untuk lebih lanjut tentang string Unicode).
\end{eulercomment}
\begin{eulerprompt}
>plot2d("x^3-x",title=u"x &rarr; x&sup3; - x"):
\end{eulerprompt}
\begin{eulercomment}
Label pada sumbu x dan y bisa vertikal, begitu juga dengan sumbu itu
sendiri.
\end{eulercomment}
\begin{eulerprompt}
>plot2d("sinc(x)",0,2pi,xl="x",yl=u"x &rarr; sinc(x)",>vertical):
\end{eulerprompt}
\eulersubheading{LaTeX}
\begin{eulercomment}
Anda juga dapat memplotkan formula LaTeX jika Anda sudah menginstal
sistem LaTeX. Saya merekomendasikan MiKTeX. Path ke biner "latex" dan
"dvipng" harus ada dalam path sistem, atau Anda harus menyiapkan LaTeX
di menu opsi.

Perlu dicatat, parsing LaTeX lambat. Jika Anda ingin menggunakan LaTeX
dalam plot animasi, sebaiknya panggil latex() sebelum loop sekali dan
gunakan hasilnya (sebuah gambar dalam matriks RGB).

Dalam plot berikut, kita menggunakan LaTeX untuk label x dan y, sebuah
label, kotak label, dan judul plot.
\end{eulercomment}
\begin{eulerprompt}
>plot2d("exp(-x)*sin(x)/x",a=0,b=2pi,c=0,d=1,grid=6,color=blue, ...
>  title=latex("\(\backslash\)text\{Function $\(\backslash\)Phi$\}"), ...
>  xl=latex("\(\backslash\)phi"),yl=latex("\(\backslash\)Phi(\(\backslash\)phi)")); ...
>textbox( ...
>  latex("\(\backslash\)Phi(\(\backslash\)phi) = e^\{-\(\backslash\)phi\} \(\backslash\)frac\{\(\backslash\)sin(\(\backslash\)phi)\}\{\(\backslash\)phi\}"),x=0.8,y=0.5); ...
>label(latex("\(\backslash\)Phi",color=blue),1,0.4):
\end{eulerprompt}
\begin{eulercomment}
Sering kali, kita menginginkan jarak non-konformal dan label teks pada
sumbu x. Kita dapat menggunakan xaxis() dan yaxis() seperti yang akan
kami tunjukkan nanti.

Cara termudah adalah membuat plot kosong dengan bingkai menggunakan
grid=4, lalu tambahkan grid dengan ygrid() dan xgrid(). Dalam contoh
berikut, kami menggunakan tiga string LaTeX untuk label pada sumbu x
dengan xtick().
\end{eulercomment}
\begin{eulerprompt}
>plot2d("sinc(x)",0,2pi,grid=4,<ticks); ...
>ygrid(-2:0.5:2,grid=6); ...
>xgrid([0:2]*pi,<ticks,grid=6);  ...
>xtick([0,pi,2pi],["0","\(\backslash\)pi","2\(\backslash\)pi"],>latex):
\end{eulerprompt}
\begin{eulercomment}
Tentu saja, fungsi juga bisa digunakan.
\end{eulercomment}
\begin{eulerprompt}
>function map f(x) ...
\end{eulerprompt}
\begin{eulerudf}
  if x>0 then return x^4
  else return x^2
  endif
  endfunction
\end{eulerudf}
\begin{eulercomment}
Parameter "map" membantu menggunakan fungsi untuk vektor. Untuk plot,
ini sebenarnya tidak perlu. Namun, untuk menunjukkan bahwa vektorisasi
berguna, kita tambahkan beberapa titik kunci pada plot di x=-1, x=0,
dan x=1.

Dalam plot berikut, kita juga memasukkan beberapa kode LaTeX. Kita
menggunakannya untuk dua label dan sebuah kotak teks. Tentu saja, Anda
hanya dapat menggunakan LaTeX jika sudah menginstalnya dengan benar.
\end{eulercomment}
\begin{eulerprompt}
>plot2d("f",-1,1,xl="x",yl="f(x)",grid=6);  ...
>plot2d([-1,0,1],f([-1,0,1]),>points,>add); ...
>label(latex("x^3"),0.72,f(0.72)); ...
>label(latex("x^2"),-0.52,f(-0.52),pos="ll"); ...
>textbox( ...
>  latex("f(x)=\(\backslash\)begin\{cases\} x^3 & x>0 \(\backslash\)\(\backslash\) x^2 & x \(\backslash\)le 0\(\backslash\)end\{cases\}"), ...
>  x=0.7,y=0.2):
\end{eulerprompt}
\begin{eulercomment}
\end{eulercomment}
\eulersubheading{Interaksi Pengguna}
\begin{eulercomment}
Saat memplot sebuah fungsi atau ekspresi, parameter \textgreater{}user memungkinkan
pengguna memperbesar dan menggeser plot dengan tombol kursor atau
mouse. Pengguna dapat:\\
- memperbesar dengan + atau -\\
- memindahkan plot dengan tombol kursor\\
- memilih jendela plot dengan mouse\\
- mereset tampilan dengan spasi\\
- keluar dengan return

Tombol spasi akan mengatur ulang plot ke jendela plot asli. Saat
memplot data, flag \textgreater{}user akan menunggu penekanan tombol.
\end{eulercomment}
\begin{eulerprompt}
>plot2d(\{\{"x^3-a*x",a=1\}\},>user,title="Press any key!"):
\end{eulerprompt}
\begin{euleroutput}
  
\end{euleroutput}
\begin{eulerprompt}
>plot2d("exp(x)*sin(x)",user=true, ...
>  title="+/- or cursor keys (return to exit)"):
\end{eulerprompt}
\begin{eulercomment}
Berikut ini adalah contoh cara interaksi pengguna yang lebih maju
(lihat tutorial tentang pemrograman untuk detail).

Fungsi built-in mousedrag() menunggu peristiwa mouse atau keyboard.
Fungsi dragpoints() memanfaatkan ini dan memungkinkan pengguna untuk
menyeret titik mana pun di dalam plot. Pertama, kita membutuhkan
fungsi plot.

Sebagai contoh, kita menginterpolasi 5 titik dengan polinomial. Fungsi
ini harus memplot ke area plot yang tetap.
\end{eulercomment}
\begin{eulerprompt}
>function plotf(xp,yp,select) ...
\end{eulerprompt}
\begin{eulerudf}
    d=interp(xp,yp);
    plot2d("interpval(xp,d,x)";d,xp,r=2);
    plot2d(xp,yp,>points,>add);
    if select>0 then
      plot2d(xp[select],yp[select],color=red,>points,>add);
    endif;
    title("Drag one point, or press space or return!");
  endfunction
\end{eulerudf}
\begin{eulercomment}
Catatan parameter titik koma dalam plot2d (d dan xp), yang diteruskan
ke evaluasi fungsi interp(). Tanpa ini, kita harus menulis fungsi
plotinterp() terlebih dahulu, mengakses nilai-nilai secara global.

Sekarang kita menghasilkan beberapa nilai acak dan membiarkan pengguna
menyeret titik-titik tersebut.
\end{eulercomment}
\begin{eulerprompt}
>t=-1:0.5:1; dragpoints("plotf",t,random(size(t))-0.5):
\end{eulerprompt}
\begin{euleroutput}
  
\end{euleroutput}
\begin{eulercomment}
Ada juga fungsi yang memplot fungsi lain tergantung pada vektor
parameter, dan memungkinkan pengguna menyesuaikan parameter-parameter
ini.

Pertama, kita memerlukan fungsi plot.
\end{eulercomment}
\begin{eulerprompt}
>function plotf([a,b]) := plot2d("exp(a*x)*cos(2pi*b*x)",0,2pi;a,b);
\end{eulerprompt}
\begin{eulercomment}
Kemudian, kita membutuhkan nama-nama untuk parameter, nilai awal, dan
matriks nx2 dari rentang, opsional dengan garis heading. Ada slider
interaktif, yang bisa mengatur nilai-nilai oleh pengguna. Fungsi
dragvalues() menyediakan ini.
\end{eulercomment}
\begin{eulerprompt}
>dragvalues("plotf",["a","b"],[-1,2],[[-2,2];[1,10]], ...
>  heading="Drag these values:",hcolor=black):
\end{eulerprompt}
\begin{eulercomment}
Dimungkinkan untuk membatasi nilai-nilai yang diseret menjadi bilangan
bulat. Sebagai contoh, kita menulis fungsi plot yang memplot
polinomial Taylor dari derajat n untuk fungsi cosinus.
\end{eulercomment}
\begin{eulerprompt}
>function plotf(n) ...
\end{eulerprompt}
\begin{eulerudf}
  plot2d("cos(x)",0,2pi,>square,grid=6);
  plot2d(&"taylor(cos(x),x,0,@n)",color=blue,>add);
  textbox("Taylor polynomial of degree "+n,0.1,0.02,style="t",>left);
  endfunction
\end{eulerudf}
\begin{eulercomment}
Sekarang kita memungkinkan derajat n bervariasi dari 0 hingga 20 dalam
20 langkah. Hasil dari dragvalues() digunakan untuk memplot sketsa
dengan n ini, dan untuk memasukkan plot ke dalam notebook.
\end{eulercomment}
\begin{eulerprompt}
>nd=dragvalues("plotf","degree",2,[0,20],20,y=0.8, ...
>   heading="Drag the value:"); ...
>plotf(nd):
\end{eulerprompt}
\begin{eulercomment}
Berikut ini adalah demonstrasi sederhana dari fungsi ini. Pengguna
dapat menggambar di atas jendela plot, meninggalkan jejak titik-titik.
\end{eulercomment}
\begin{eulerprompt}
>function dragtest ...
\end{eulerprompt}
\begin{eulerudf}
    plot2d(none,r=1,title="Drag with the mouse, or press any key!");
    start=0;
    repeat
      \{flag,m,time\}=mousedrag();
      if flag==0 then return; endif;
      if flag==2 then
        hold on; mark(m[1],m[2]); hold off;
      endif;
    end
  endfunction
\end{eulerudf}
\begin{eulerprompt}
>dragtest // lihat hasilnya dan cobalah lakukan!
\end{eulerprompt}
\eulersubheading{Gaya Plot 2D}
\begin{eulercomment}
Secara default, EMT menghitung ticks sumbu otomatis dan menambahkan
label ke setiap tick. Ini bisa diubah dengan parameter grid. Gaya
default dari sumbu dan label bisa dimodifikasi. Selain itu, label dan
judul bisa ditambahkan secara manual. Untuk mengatur ulang ke gaya
default, gunakan reset().
\end{eulercomment}
\begin{eulerprompt}
>aspect();
>figure(3,4); ...
> figure(1); plot2d("x^3-x",grid=0); ... // tanpa grid, bingkai, atau sumbu
> figure(2); plot2d("x^3-x",grid=1); ... // sumbu x-y
> figure(3); plot2d("x^3-x",grid=2); ... // centang default
> figure(4); plot2d("x^3-x",grid=3); ... // sumbu x-y dengan label di dalam
> figure(5); plot2d("x^3-x",grid=4); ... // tanpa centang, hanya label
> figure(6); plot2d("x^3-x",grid=5); ... // default, tanpa margin
> figure(7); plot2d("x^3-x",grid=6); ... // hanya sumbu
> figure(8); plot2d("x^3-x",grid=7); ... // hanya sumbu, centang di sumbu
> figure(9); plot2d("x^3-x",grid=8); ... // hanya sumbu, centang lebih halus di sumbu
> figure(10); plot2d("x^3-x",grid=9); ... // default, centang kecil di dalam
> figure(11); plot2d("x^3-x",grid=10); ...// tanpa centang, hanya sumbu
> figure(0):
\end{eulerprompt}
\begin{eulercomment}
Parameter \textless{}frame mematikan bingkai, dan framecolor=blue mengatur
bingkai ke warna biru. Jika Anda ingin menambahkan centang sendiri,
Anda bisa menggunakan style=0, dan tambahkan semuanya kemudian.
\end{eulercomment}
\begin{eulerprompt}
>aspect(1.5); 
>plot2d("x^3-x",grid=0); // plot
>frame; xgrid([-1,0,1]); ygrid(0): // add frame and grid
\end{eulerprompt}
\begin{eulercomment}
Untuk judul plot dan label sumbu, lihat contoh berikut.
\end{eulercomment}
\begin{eulerprompt}
>plot2d("exp(x)",-1,1);
>textcolor(black); // set the text color to black
>title(latex("y=e^x")); // title above the plot
>xlabel(latex("x")); // "x" for x-axis
>ylabel(latex("y"),>vertical); // vertical "y" for y-axis
>label(latex("(0,1)"),0,1,color=blue): // label a point
\end{eulerprompt}
\begin{eulercomment}
Sumbu dapat digambar secara terpisah dengan xaxis() dan yaxis().
\end{eulercomment}
\begin{eulerprompt}
>plot2d("x^3-x",<grid,<frame);
>xaxis(0,xx=-2:1,style="->"); yaxis(0,yy=-5:5,style="->"):
\end{eulerprompt}
\begin{eulercomment}
Teks pada plot dapat diatur dengan label(). Dalam contoh berikut, "lc"
berarti lower center. Ini mengatur posisi label relatif terhadap
koordinat plot.
\end{eulercomment}
\begin{eulerprompt}
>function f(x) &= x^3-x
\end{eulerprompt}
\begin{euleroutput}
  
                                   3
                                  x  - x
  
\end{euleroutput}
\begin{eulerprompt}
>plot2d(f,-1,1,>square);
>x0=fmin(f,0,1); // compute point of minimum
>label("Rel. Min.",x0,f(x0),pos="lc"): // add a label there
\end{eulerprompt}
\begin{eulercomment}
Ada juga kotak teks (text boxes).
\end{eulercomment}
\begin{eulerprompt}
>plot2d(&f(x),-1,1,-2,2); // function
>plot2d(&diff(f(x),x),>add,style="--",color=red); // derivative
>labelbox(["f","f'"],["-","--"],[black,red]): // label box
>plot2d(["exp(x)","1+x"],color=[black,blue],style=["-","-.-"]):
>gridstyle("->",color=gray,textcolor=gray,framecolor=gray);  ...
> plot2d("x^3-x",grid=1);   ...
> settitle("y=x^3-x",color=black); ...
> label("x",2,0,pos="bc",color=gray);  ...
> label("y",0,6,pos="cl",color=gray); ...
> reset():
\end{eulerprompt}
\begin{eulercomment}
Untuk kontrol yang lebih mendetail, sumbu x dan sumbu y dapat
dilakukan secara manual. Perintah fullwindow() memperluas jendela plot
karena kita tidak lagi memerlukan tempat untuk label di luar jendela
plot. Gunakan shrinkwindow() atau reset() untuk mengatur ulang ke
default.
\end{eulercomment}
\begin{eulerprompt}
>fullwindow; ...
> gridstyle(color=darkgray,textcolor=darkgray); ...
> plot2d(["2^x","1","2^(-x)"],a=-2,b=2,c=0,d=4,<grid,color=4:6,<frame); ...
> xaxis(0,-2:1,style="->"); xaxis(0,2,"x",<axis); ...
> yaxis(0,4,"y",style="->"); ...
> yaxis(-2,1:4,>left); ...
> yaxis(2,2^(-2:2),style=".",<left); ...
> labelbox(["2^x","1","2^-x"],colors=4:6,x=0.8,y=0.2); ...
> reset:
\end{eulerprompt}
\begin{eulercomment}
Berikut contoh lainnya, di mana digunakan string Unicode dan sumbu di
luar area plot.
\end{eulercomment}
\begin{eulerprompt}
>aspect(1.5); 
>plot2d(["sin(x)","cos(x)"],0,2pi,color=[red,green],<grid,<frame); ...
> xaxis(-1.1,(0:2)*pi,xt=["0",u"&pi;",u"2&pi;"],style="-",>ticks,>zero);  ...
> xgrid((0:0.5:2)*pi,<ticks); ...
> yaxis(-0.1*pi,-1:0.2:1,style="-",>zero,>grid); ...
> labelbox(["sin","cos"],colors=[red,green],x=0.5,y=0.2,>left); ...
> xlabel(u"&phi;"); ylabel(u"f(&phi;)"):
\end{eulerprompt}
\eulerheading{Plotting Data 2D}
\begin{eulercomment}
Jika x dan y adalah vektor data, data ini akan digunakan sebagai
koordinat x dan y dari sebuah kurva. Dalam hal ini, a, b, c, dan d,
atau radius r dapat ditentukan, atau jendela plot akan menyesuaikan
secara otomatis dengan data. Alternatifnya, square dapat diatur untuk
menjaga rasio aspek persegi.

Memplot ekspresi hanyalah singkatan untuk plot data. Untuk plot data,
Anda memerlukan satu atau lebih baris nilai x, dan satu atau lebih
baris nilai y. Dari rentang dan nilai x, fungsi plot2d akan menghitung
data untuk dipetakan, secara default dengan evaluasi adaptif terhadap
fungsi tersebut. Untuk plot titik gunakan "\textgreater{}points", untuk garis dan
titik campuran gunakan "\textgreater{}addpoints".

Namun, Anda dapat memasukkan data secara langsung.\\
- Gunakan vektor baris untuk x dan y untuk satu fungsi.\\
- Matriks untuk x dan y akan dipetakan baris demi baris.

Berikut adalah contoh dengan satu baris untuk x dan y.
\end{eulercomment}
\begin{eulerprompt}
>x=-10:0.1:10; y=exp(-x^2)*x; plot2d(x,y):
\end{eulerprompt}
\begin{eulercomment}
Data juga dapat dipetakan sebagai titik-titik. Gunakan points=true
untuk ini. Plot bekerja seperti poligon, tetapi hanya menggambar
sudut-sudutnya.

style="...": Pilih dari "[]", "\textless{}\textgreater{}", "o", ".", "..", "+", "*", "[]",
"\textless{}\textgreater{}", "o", "..", "", "\textbar{}".\\
Untuk memetakan kumpulan titik gunakan points. Jika warna adalah
vektor warna, setiap titik akan memiliki warna yang berbeda. Untuk
matriks koordinat dan vektor kolom, warna berlaku untuk baris-baris
matriks.

Parameter addpoints menambahkan titik ke segmen garis untuk plot data.
\end{eulercomment}
\begin{eulerprompt}
>xdata=[1,1.5,2.5,3,4]; ydata=[3,3.1,2.8,2.9,2.7]; // data
>plot2d(xdata,ydata,a=0.5,b=4.5,c=2.5,d=3.5,style="."); // lines
>plot2d(xdata,ydata,>points,>add,style="o"): // add points
>p=polyfit(xdata,ydata,1); // get regression line
>plot2d("polyval(p,x)",>add,color=red): // add plot of line
\end{eulerprompt}
\eulerheading{Menggambar Daerah Yang Dibatasi Kurva}
\begin{eulercomment}
Plot data sebenarnya adalah poligon. Kita juga bisa memplot kurva atau
kurva terisi (filled curves).

- filled=true mengisi plot.\\
- style="...": Pilih dari "", "/", "", "/".\\
- fillcolor: Lihat di atas untuk warna yang tersedia.

Warna pengisian ditentukan oleh argumen "fillcolor", dan opsional
\textless{}outline mencegah menggambar batas untuk semua gaya kecuali gaya
default.
\end{eulercomment}
\begin{eulerprompt}
>t=linspace(0,2pi,1000); // parameter for curve
>x=sin(t)*exp(t/pi); y=cos(t)*exp(t/pi); // x(t) and y(t)
>figure(1,2); aspect(16/9)
>figure(1); plot2d(x,y,r=10); // plot curve
>figure(2); plot2d(x,y,r=10,>filled,style="/",fillcolor=red); // fill curve
>figure(0):
\end{eulerprompt}
\begin{eulercomment}
Dalam contoh berikut, kita memplot sebuah elips yang terisi dan dua
heksagon terisi menggunakan kurva tertutup dengan 6 titik dan gaya
pengisian yang berbeda.
\end{eulercomment}
\begin{eulerprompt}
>x=linspace(0,2pi,1000); plot2d(sin(x),cos(x)*0.5,r=1,>filled,style="/"):
>t=linspace(0,2pi,6); ...
>plot2d(cos(t),sin(t),>filled,style="/",fillcolor=red,r=1.2):
>t=linspace(0,2pi,6); plot2d(cos(t),sin(t),>filled,style="#"):
\end{eulerprompt}
\begin{eulercomment}
Contoh berikut adalah sebuah septagon, yang kita buat dengan 7 titik
pada lingkaran satuan.
\end{eulercomment}
\begin{eulerprompt}
>t=linspace(0,2pi,7);  ...
> plot2d(cos(t),sin(t),r=1,>filled,style="/",fillcolor=red):
\end{eulerprompt}
\begin{eulercomment}
Berikut adalah set nilai maksimal dari empat kondisi linear yang
kurang dari atau sama dengan 3. Ini adalah A[k].v\textless{}=3 untuk semua baris
dari A. Untuk mendapatkan sudut yang bagus, kita menggunakan n yang
relatif besar.
\end{eulercomment}
\begin{eulerprompt}
>A=[2,1;1,2;-1,0;0,-1];
>function f(x,y) := max([x,y].A');
>plot2d("f",r=4,level=[0;3],color=green,n=111):
\end{eulerprompt}
\begin{eulercomment}
Poin utama dari bahasa matriks adalah memungkinkan kita untuk
menghasilkan tabel fungsi dengan mudah.
\end{eulercomment}
\begin{eulerprompt}
>t=linspace(0,2pi,1000); x=cos(3*t); y=sin(4*t);
\end{eulerprompt}
\begin{eulercomment}
Sekarang kita memiliki vektor x dan y dari nilai. Fungsi plot2d()
dapat menggambar nilai-nilai ini sebagai kurva yang menghubungkan
titik-titik tersebut. Plot dapat diisi. Dalam hal ini, hasilnya
menjadi indah karena aturan pengisian, yang digunakan untuk mengisi.
\end{eulercomment}
\begin{eulerprompt}
>plot2d(x,y,<grid,<frame,>filled):
\end{eulerprompt}
\begin{eulercomment}
Sebuah vektor interval diplotkan terhadap nilai x sebagai daerah
terisi antara nilai bawah dan atas dari interval.\\
Ini berguna untuk menggambarkan kesalahan perhitungan. Namun, ini juga
dapat digunakan untuk memplot kesalahan statistik.
\end{eulercomment}
\begin{eulerprompt}
>t=0:0.1:1; ...
> plot2d(t,interval(t-random(size(t)),t+random(size(t))),style="|");  ...
> plot2d(t,t,add=true):
\end{eulerprompt}
\begin{eulercomment}
Jika x adalah vektor yang diurutkan, dan y adalah vektor interval,
maka plot2d() akan memplot rentang interval yang terisi di dalam
bidang. Gaya pengisian sama dengan gaya poligon.
\end{eulercomment}
\begin{eulerprompt}
>t=-1:0.01:1; x=~t-0.01,t+0.01~; y=x^3-x;
>plot2d(t,y):
\end{eulerprompt}
\begin{eulercomment}
Kita dapat mengisi daerah nilai untuk fungsi tertentu. Untuk ini,
level harus berupa matriks 2xn. Baris pertama adalah batas bawah dan
baris kedua berisi batas atas.
\end{eulercomment}
\begin{eulerprompt}
>expr := "2*x^2+x*y+3*y^4+y"; // define an expression f(x,y)
>plot2d(expr,level=[0;1],style="-",color=blue): // 0 <= f(x,y) <= 1
\end{eulerprompt}
\begin{eulercomment}
Kita juga dapat mengisi rentang nilai seperti:

\end{eulercomment}
\begin{eulerformula}
\[
-1 \le (x^2+y^2)^2-x^2+y^2 \le 0.
\]
\end{eulerformula}
\begin{eulercomment}
\end{eulercomment}
\begin{eulerprompt}
>plot2d("(x^2+y^2)^2-x^2+y^2",r=1.2,level=[-1;0],style="/"):
>plot2d("cos(x)","sin(x)^3",xmin=0,xmax=2pi,>filled,style="/"):
\end{eulerprompt}
\eulerheading{Grafik Fungsi Parametrik}
\begin{eulercomment}
Nilai x tidak perlu diurutkan. (x, y) hanya menggambarkan sebuah
kurva. Jika x diurutkan, maka kurva tersebut adalah grafik suatu
fungsi.

Pada contoh berikut, kita menggambar spiral.

\end{eulercomment}
\begin{eulerformula}
\[
\gamma(t) = t \cdot (\cos(2\pi t),\sin(2\pi t))
\]
\end{eulerformula}
\begin{eulercomment}
Kita perlu menggunakan banyak titik untuk tampilan yang halus atau
menggunakan fungsi adaptive() untuk mengevaluasi ekspresi (lihat
fungsi adaptive() untuk lebih detail).
\end{eulercomment}
\begin{eulerprompt}
>t=linspace(0,1,1000); ...
>plot2d(t*cos(2*pi*t),t*sin(2*pi*t),r=1):
\end{eulerprompt}
\begin{eulercomment}
Alternatifnya, dimungkinkan menggunakan dua ekspresi untuk kurva.
Berikut ini adalah plot yang sama seperti di atas.
\end{eulercomment}
\begin{eulerprompt}
>plot2d("x*cos(2*pi*x)","x*sin(2*pi*x)",xmin=0,xmax=1,r=1):
>t=linspace(0,1,1000); r=exp(-t); x=r*cos(2pi*t); y=r*sin(2pi*t);
>plot2d(x,y,r=1):
\end{eulerprompt}
\begin{eulercomment}
Pada contoh berikut, kita menggambar kurva:

\end{eulercomment}
\begin{eulerformula}
\[
\gamma(t) = (r(t) \cos(t), r(t) \sin(t))
\]
\end{eulerformula}
\begin{eulercomment}
dengan

\end{eulercomment}
\begin{eulerformula}
\[
r(t) = 1 + \dfrac{\sin(3t)}{2}.
\]
\end{eulerformula}
\begin{eulerprompt}
>t=linspace(0,2pi,1000); r=1+sin(3*t)/2; x=r*cos(t); y=r*sin(t); ...
>plot2d(x,y,>filled,fillcolor=red,style="/",r=1.5):
\end{eulerprompt}
\eulerheading{Menggambar Grafik Bilangan Kompleks}
\begin{eulercomment}
Sebuah array dari bilangan kompleks juga dapat diplotkan. Titik-titik
grid akan terhubung. Jika sejumlah garis grid ditentukan (atau vektor
1x2 dari garis grid) dalam argumen cgrid, hanya garis-garis grid
tersebut yang akan terlihat. Matriks bilangan kompleks secara otomatis
akan diplot sebagai grid di bidang kompleks.

Pada contoh berikut, kita menggambar gambar lingkaran satuan di bawah
fungsi eksponensial. Parameter cgrid menyembunyikan beberapa kurva
grid.
\end{eulercomment}
\begin{eulerprompt}
>aspect(); r=linspace(0,1,50); a=linspace(0,2pi,80)'; z=r*exp(I*a);...
>plot2d(z,a=-1.25,b=1.25,c=-1.25,d=1.25,cgrid=10):
>aspect(1.25); r=linspace(0,1,50); a=linspace(0,2pi,200)'; z=r*exp(I*a);
>plot2d(exp(z),cgrid=[40,10]):
>r=linspace(0,1,10); a=linspace(0,2pi,40)'; z=r*exp(I*a);
>plot2d(exp(z),>points,>add):
\end{eulerprompt}
\begin{eulercomment}
Sebuah vektor bilangan kompleks secara otomatis diplot sebagai kurva
di bidang kompleks dengan bagian nyata dan bagian imajiner.

Pada contoh ini, kita menggambar lingkaran satuan dengan

\end{eulercomment}
\begin{eulerformula}
\[
\gamma(t) = e^{it}
\]
\end{eulerformula}
\begin{eulerprompt}
>t=linspace(0,2pi,1000); ...
>plot2d(exp(I*t)+exp(4*I*t),r=2):
\end{eulerprompt}
\eulerheading{Plot Statistik}
\begin{eulercomment}
Ada banyak fungsi yang dikhususkan untuk plot statistik. Salah satu
plot yang sering digunakan adalah plot kolom. Penjumlahan kumulatif
dari nilai yang terdistribusi normal 0-1 menghasilkan gerak acak.
\end{eulercomment}
\begin{eulerprompt}
>plot2d(cumsum(randnormal(1,1000))):
\end{eulerprompt}
\begin{eulercomment}
Dengan menggunakan dua baris, kita menunjukkan gerak acak dalam dua
dimensi.
\end{eulercomment}
\begin{eulerprompt}
>X=cumsum(randnormal(2,1000)); plot2d(X[1],X[2]):
>columnsplot(cumsum(random(10)),style="/",color=blue):
\end{eulerprompt}
\begin{eulercomment}
Plot ini juga dapat menampilkan string sebagai label.
\end{eulercomment}
\begin{eulerprompt}
>months=["Jan","Feb","Mar","Apr","May","Jun", ...
>  "Jul","Aug","Sep","Oct","Nov","Dec"];
>values=[10,12,12,18,22,28,30,26,22,18,12,8];
>columnsplot(values,lab=months,color=red,style="-");
>title("Temperature"):
>k=0:10;
>plot2d(k,bin(10,k),>bar):
>plot2d(k,bin(10,k)); plot2d(k,bin(10,k),>points,>add):
>plot2d(normal(1000),normal(1000),>points,grid=6,style=".."):
>plot2d(normal(1,1000),>distribution,style="O"):
>plot2d("qnormal",0,5;2.5,0.5,>filled):
\end{eulerprompt}
\begin{eulercomment}
Distribusi ekperimen statistik dapat menggunakan distribution=n dengan
plot2d.
\end{eulercomment}
\begin{eulerprompt}
>w=randexponential(1,1000); // exponential distribution
>plot2d(w,>distribution): // or distribution=n with n intervals
\end{eulerprompt}
\begin{eulercomment}
Atau kita bisa menghitung distribusi dari data dan memplot hasilnya
dengan menggunakan bar dalam plot2d atau column plot.
\end{eulercomment}
\begin{eulerprompt}
>w=normal(1000); // 0-1-normal distribution
>\{x,y\}=histo(w,10,v=[-6,-4,-2,-1,0,1,2,4,6]); // interval bounds v
>plot2d(x,y,>bar):
\end{eulerprompt}
\begin{eulercomment}
Fungsi statplot() menetapkan gaya dengan string sederhana.
\end{eulercomment}
\begin{eulerprompt}
>statplot(1:10,cumsum(random(10)),"b"):
>n=10; i=0:n; ...
>plot2d(i,bin(n,i)/2^n,a=0,b=10,c=0,d=0.3); ...
>plot2d(i,bin(n,i)/2^n,points=true,style="ow",add=true,color=blue):
\end{eulerprompt}
\begin{eulercomment}
Selain itu, data dapat diplot sebagai batang (bars). Dalam hal ini, x
harus diurutkan dan satu elemen lebih panjang dari y. Batang akan
diperpanjang dari x[i] hingga x[i+1] dengan nilai y[i]. Jika x
memiliki ukuran yang sama dengan y, maka akan diperpanjang dengan satu
elemen dengan jarak terakhir.
\end{eulercomment}
\begin{eulerprompt}
>n=10; k=bin(n,0:n); ...
>plot2d(-0.5:n+0.5,k,bar=true,fillcolor=lightgray):
\end{eulerprompt}
\begin{eulercomment}
Data untuk plot batang (bar=1) dan histogram (histogram=1) dapat
diberikan secara eksplisit dalam xv dan yv, atau dihitung dari
distribusi empiris dalam xv dengan distribution=n. Histogram dari
nilai-nilai xv akan dihitung secara otomatis dengan histogram. Jika
even=true ditentukan, nilai-nilai xv akan dihitung dalam interval
integer.
\end{eulercomment}
\begin{eulerprompt}
>plot2d(normal(10000),distribution=50):
>k=0:10; m=bin(10,k); x=(0:11)-0.5; plot2d(x,m,>bar):
>columnsplot(m,k):
>plot2d(random(600)*6,histogram=6):
\end{eulerprompt}
\begin{eulercomment}
Untuk distribusi, terdapat parameter distribution=n, yang secara
otomatis menghitung nilai dan mencetak distribusi relatif dengan n
sub-interval.
\end{eulercomment}
\begin{eulerprompt}
>plot2d(normal(1,1000),distribution=10,style="\(\backslash\)/"):
\end{eulerprompt}
\begin{eulercomment}
Dengan parameter even=true, ini akan menggunakan interval bilangan
bulat.
\end{eulercomment}
\begin{eulerprompt}
>plot2d(intrandom(1,1000,10),distribution=10,even=true):
\end{eulerprompt}
\begin{eulercomment}
Perlu diingat bahwa ada banyak plot statistik yang mungkin berguna.
Lihatlah tutorial tentang statistik.
\end{eulercomment}
\begin{eulerprompt}
>columnsplot(getmultiplicities(1:6,intrandom(1,6000,6))):
>plot2d(normal(1,1000),>distribution); ...
>  plot2d("qnormal(x)",color=red,thickness=2,>add):
\end{eulerprompt}
\begin{eulercomment}
Ada juga banyak plot khusus untuk statistik. Sebuah boxplot
menunjukkan kuartil dari distribusi ini dan banyak outlier. Menurut
definisi, outlier dalam boxplot adalah data yang melebihi 1,5 kali
rentang 50\% tengah dari plot.
\end{eulercomment}
\begin{eulerprompt}
>M=normal(5,1000); boxplot(quartiles(M)):
\end{eulerprompt}
\eulerheading{Fungsi Implisit}
\begin{eulercomment}
Plot implisit menampilkan garis level yang menyelesaikan persamaan
f(x,y)=level, dimana "level" dapat berupa nilai tunggal atau vektor
nilai. Jika level="auto", akan ada nc garis level, yang akan tersebar
merata antara nilai minimum dan maksimum fungsi. Warna yang lebih
gelap atau lebih terang dapat ditambahkan dengan hue untuk menunjukkan
nilai fungsi. Untuk fungsi implisit, xv harus berupa fungsi atau
ekspresi dari parameter x dan y, atau xv bisa berupa matriks nilai.

Euler dapat menandai garis level dari fungsi apapun seperti

\end{eulercomment}
\begin{eulerformula}
\[
f(x,y) = c
\]
\end{eulerformula}
\begin{eulercomment}
Untuk satu atau lebih konstanta c, Anda bisa menggunakan plot2d()
dengan plot implisitnya di bidang. Parameter untuk c adalah level=c,
di mana c bisa menjadi vektor dari garis level. Selain itu, skema
warna dapat digambar di latar belakang untuk menunjukkan nilai fungsi
untuk setiap titik dalam plot. Parameter n menentukan tingkat
ketelitian plot.
\end{eulercomment}
\begin{eulerprompt}
>aspect(1.5); 
>plot2d("x^2+y^2-x*y-x",r=1.5,level=0,contourcolor=red):
>expr := "2*x^2+x*y+3*y^4+y"; // definisikan ekspresi f(x,y)
>plot2d(expr,level=0): // solusi dari f(x,y)=0
>plot2d(expr,level=0:0.5:20,>hue,contourcolor=white,n=200): // tampilan yang baik
>plot2d(expr,level=0:0.5:20,>hue,>spectral,n=200,grid=4): // tampilan yang lebih baik
\end{eulerprompt}
\begin{eulercomment}
Ini juga bekerja untuk plot data. Namun, Anda harus menentukan rentang
untuk label sumbu.
\end{eulercomment}
\begin{eulerprompt}
>x=-2:0.05:1; y=x'; z=expr(x,y);
>plot2d(z,level=0,a=-1,b=2,c=-2,d=1,>hue):
>plot2d("x^3-y^2",>contour,>hue,>spectral):
>plot2d("x^3-y^2",level=0,contourwidth=3,>add,contourcolor=red):
>z=z+normal(size(z))*0.2;
>plot2d(z,level=0.5,a=-1,b=2,c=-2,d=1):
>plot2d(expr,level=[0:0.2:5;0.05:0.2:5.05],color=lightgray):
>plot2d("x^2+y^3+x*y",level=1,r=4,n=100):
>plot2d("x^2+2*y^2-x*y",level=0:0.1:10,n=100,contourcolor=white,>hue):
\end{eulerprompt}
\begin{eulercomment}
Dimungkinkan juga untuk mengisi himpunan

\end{eulercomment}
\begin{eulerformula}
\[
a \le f(x,y) \le b
\]
\end{eulerformula}
\begin{eulercomment}
dengan rentang level

Dimungkinkan untuk mengisi area nilai untuk fungsi tertentu. Untuk
ini, level harus berupa matriks 2xn. Baris pertama adalah batas bawah
dan baris kedua berisi batas atas.
\end{eulercomment}
\begin{eulerprompt}
>plot2d(expr,level=[0;1],style="-",color=blue): // 0 <= f(x,y) <= 1
\end{eulerprompt}
\begin{eulercomment}
Plot implisit juga dapat menunjukkan rentang level. Maka level harus
berupa matriks 2xn dari interval level, di mana baris pertama berisi
awal dan baris kedua berisi akhir dari setiap interval. Sebagai
alternatif, vektor baris sederhana dapat digunakan untuk level, dan
parameter dl memperluas nilai level menjadi interval.
\end{eulercomment}
\begin{eulerprompt}
>plot2d("x^4+y^4",r=1.5,level=[0;1],color=blue,style="/"):
>plot2d("x^2+y^3+x*y",level=[0,2,4;1,3,5],style="/",r=2,n=100):
>plot2d("x^2+y^3+x*y",level=-10:20,r=2,style="-",dl=0.1,n=100):
>plot2d("sin(x)*cos(y)",r=pi,>hue,>levels,n=100):
\end{eulerprompt}
\begin{eulercomment}
Dimungkinkan juga untuk menandai area

\end{eulercomment}
\begin{eulerformula}
\[
a \le f(x,y) \le b.
\]
\end{eulerformula}
\begin{eulercomment}
Ini dilakukan dengan menambahkan level dengan dua baris.
\end{eulercomment}
\begin{eulerprompt}
>plot2d("(x^2+y^2-1)^3-x^2*y^3",r=1.3, ...
>  style="#",color=red,<outline, ...
>  level=[-2;0],n=100):
\end{eulerprompt}
\begin{eulercomment}
Dimungkinkan untuk menentukan level tertentu. Misalnya, kita dapat
memplot solusi dari persamaan seperti:

\end{eulercomment}
\begin{eulerformula}
\[
x^3-xy+x^2y^2=6
\]
\end{eulerformula}
\begin{eulerprompt}
>plot2d("x^3-x*y+x^2*y^2",r=6,level=1,n=100):
>function starplot1 (v, style="/", color=green, lab=none) ...
\end{eulerprompt}
\begin{eulerudf}
    if !holding() then clg; endif;
    w=window(); window(0,0,1024,1024);
    h=holding(1);
    r=max(abs(v))*1.2;
    setplot(-r,r,-r,r);
    n=cols(v); t=linspace(0,2pi,n);
    v=v|v[1]; c=v*cos(t); s=v*sin(t);
    cl=barcolor(color); st=barstyle(style);
    loop 1 to n
      polygon([0,c[#],c[#+1]],[0,s[#],s[#+1]],1);
      if lab!=none then
        rlab=v[#]+r*0.1;
        \{col,row\}=toscreen(cos(t[#])*rlab,sin(t[#])*rlab);
        ctext(""+lab[#],col,row-textheight()/2);
      endif;
    end;
    barcolor(cl); barstyle(st);
    holding(h);
    window(w);
  endfunction
\end{eulerudf}
\begin{eulercomment}
Tidak ada grid atau tanda sumbu di sini. Selain itu, kita menggunakan
seluruh jendela untuk plot.

Kita memanggil reset sebelum menguji plot ini untuk mengembalikan
default grafis. Ini tidak diperlukan jika Anda yakin plot Anda
berfungsi.
\end{eulercomment}
\begin{eulerprompt}
>reset; starplot1(normal(1,10)+5,color=red,lab=1:10):
\end{eulerprompt}
\begin{eulercomment}
Terkadang, Anda mungkin ingin memplot sesuatu yang tidak bisa
dilakukan plot2d, tetapi hampir bisa.

Pada fungsi berikut, kita membuat plot impuls logaritmik. plot2d dapat
membuat plot logaritmik, tetapi tidak untuk batang impuls.
\end{eulercomment}
\begin{eulerprompt}
>function logimpulseplot1 (x,y) ...
\end{eulerprompt}
\begin{eulerudf}
    \{x0,y0\}=makeimpulse(x,log(y)/log(10));
    plot2d(x0,y0,>bar,grid=0);
    h=holding(1);
    frame();
    xgrid(ticks(x));
    p=plot();
    for i=-10 to 10;
      if i<=p[4] and i>=p[3] then
         ygrid(i,yt="10^"+i);
      endif;
    end;
    holding(h);
  endfunction
\end{eulerudf}
\begin{eulercomment}
Mari kita uji dengan nilai yang terdistribusi secara eksponensial.
\end{eulercomment}
\begin{eulerprompt}
>aspect(1.5); x=1:10; y=-log(random(size(x)))*200; ...
>logimpulseplot1(x,y):
\end{eulerprompt}
\begin{eulercomment}
Mari kita animasikan kurva 2D menggunakan plot langsung. Perintah
plot(x,y) hanya memplot kurva ke dalam jendela plot. setplot(a,b,c,d)
mengatur jendela ini.

Fungsi wait(0) memaksa plot muncul di jendela grafis. Jika tidak,
penggambaran ulang terjadi dalam interval waktu yang jarang.
\end{eulercomment}
\begin{eulerprompt}
>function animliss (n,m) ...
\end{eulerprompt}
\begin{eulerudf}
  t=linspace(0,2pi,500);
  f=0;
  c=framecolor(0);
  l=linewidth(2);
  setplot(-1,1,-1,1);
  repeat
    clg;
    plot(sin(n*t),cos(m*t+f));
    wait(0);
    if testkey() then break; endif;
    f=f+0.02;
  end;
  framecolor(c);
  linewidth(l);
  endfunction
\end{eulerudf}
\begin{eulercomment}
Tekan tombol apa saja untuk menghentikan animasi ini.
\end{eulercomment}
\begin{eulerprompt}
>animliss(2,3); // lihat hasilnya, jika sudah puas, tekan ENTER
\end{eulerprompt}
\eulerheading{Grafik Logaritmik}
\begin{eulercomment}
EMT menggunakan parameter logplot untuk skala logaritmik. Plot
logaritmik dapat dibuat dengan menggunakan skala logaritmik pada sumbu
y dengan logplot=1, atau menggunakan skala logaritmik pada sumbu x dan
y dengan logplot=2, atau hanya pada sumbu x dengan logplot=3.

\end{eulercomment}
\begin{eulerttcomment}
 - logplot=1: skala logaritmik pada sumbu y
 - logplot=2: skala logaritmik pada sumbu x dan y
 - logplot=3: skala logaritmik pada sumbu x
\end{eulerttcomment}
\begin{eulerprompt}
>plot2d("exp(x^3-x)*x^2",1,5,logplot=1):
>plot2d("exp(x+sin(x))",0,100,logplot=1):
>plot2d("exp(x+sin(x))",10,100,logplot=2):
>plot2d("gamma(x)",1,10,logplot=1):
>plot2d("log(x*(2+sin(x/100)))",10,1000,logplot=3):
\end{eulerprompt}
\begin{eulercomment}
Ini juga bekerja pada grafik data.
\end{eulercomment}
\begin{eulerprompt}
>x=10^(1:20); y=x^2-x;
>plot2d(x,y,logplot=2):
\end{eulerprompt}
\eulerheading{Latihan Soal}
\begin{eulercomment}
Nama : Muhammad Lutfi Ramadhan\\
Kelas : Matematika B 2023\\
NIM : 23030630021

1. Gambarkan grafik fungsi berikut!\\
\end{eulercomment}
\begin{eulerformula}
\[
f(x)= x^2-4x+3
\]
\end{eulerformula}
\begin{eulercomment}
Penyelesaian:
\end{eulercomment}
\begin{eulerprompt}
>plot2d("x^2-4*x+3",-1,5):
\end{eulerprompt}
\begin{eulercomment}
2. Buatkan grafik fungsi berikut pada interval 0 sampai 2pi!

\end{eulercomment}
\begin{eulerformula}
\[
f(x)=sin(x)
\]
\end{eulerformula}
\begin{eulercomment}
Penyelesaian:
\end{eulercomment}
\begin{eulerprompt}
>plot2d("sin(x)",0,2*pi):
\end{eulerprompt}
\begin{eulercomment}
3. Gambarkan grafik fungsi f(x)=sin(x) dan g(x)=cos(x) secara
bersamaan pada interval 0 sampai 2pi!

Penyelesaian 
\end{eulercomment}
\begin{eulerprompt}
>plot2d(["sin(x)", "cos(x)"], 0, 2*pi):
\end{eulerprompt}
\begin{eulercomment}
4. Gambarkan grafik fungsi f(x)=log(x) pada interval [0.1,5]!

Penyelesaian:
\end{eulercomment}
\begin{eulerprompt}
>plot2d("log(x)", 0.1, 5):
\end{eulerprompt}
\begin{eulercomment}
5. Gambarkan fungsi eksponen berikut pada interval[-2,2]!

\end{eulercomment}
\begin{eulerformula}
\[
f(x)=e^x
\]
\end{eulerformula}
\begin{eulercomment}
Penyelesaian:
\end{eulercomment}
\begin{eulerprompt}
>plot2d("exp(x)", -2, 2):
>      
\end{eulerprompt}
\eulerheading{Rujukan Lengkap Fungsi plot2d()}
\begin{eulercomment}
\end{eulercomment}
\begin{eulerttcomment}
  function plot2d (xv, yv, btest, a, b, c, d, xmin, xmax, r, n,  ..
  logplot, grid, frame, framecolor, square, color, thickness, style, ..
  auto, add, user, delta, points, addpoints, pointstyle, bar, histogram,  ..
  distribution, even, steps, own, adaptive, hue, level, contour,  ..
  nc, filled, fillcolor, outline, title, xl, yl, maps, contourcolor, ..
  contourwidth, ticks, margin, clipping, cx, cy, insimg, spectral,  ..
  cgrid, vertical, smaller, dl, niveau, levels)
\end{eulerttcomment}
\begin{eulercomment}
Multipurpose plot function for plots in the plane (2D plots). This function can do
plots of functions of one variables, data plots, curves in the plane, bar plots, grids
of complex numbers, and implicit plots of functions of two variables.

Parameters
\\
x,y       : equations, functions or data vectors\\
a,b,c,d   : Plot area (default a=-2,b=2)\\
r         : if r is set, then a=cx-r, b=cx+r, c=cy-r, d=cy+r\\
\end{eulercomment}
\begin{eulerttcomment}
            r can be a vector [rx,ry] or a vector [rx1,rx2,ry1,ry2].
\end{eulerttcomment}
\begin{eulercomment}
xmin,xmax : range of the parameter for curves\\
auto      : Determine y-range automatically (default)\\
square    : if true, try to keep square x-y-ranges\\
n         : number of intervals (default is adaptive)\\
grid      : 0 = no grid and labels,\\
\end{eulercomment}
\begin{eulerttcomment}
            1 = axis only,
            2 = normal grid (see below for the number of grid lines)
            3 = inside axis
            4 = no grid
            5 = full grid including margin
            6 = ticks at the frame
            7 = axis only
            8 = axis only, sub-ticks
\end{eulerttcomment}
\begin{eulercomment}
frame     : 0 = no frame\\
framecolor: color of the frame and the grid\\
margin    : number between 0 and 0.4 for the margin around the plot\\
color     : Color of curves. If this is a vector of colors,\\
\end{eulercomment}
\begin{eulerttcomment}
            it will be used for each row of a matrix of plots. In the case of
            point plots, it should be a column vector. If a row vector or a
            full matrix of colors is used for point plots, it will be used for
            each data point.
\end{eulerttcomment}
\begin{eulercomment}
thickness : line thickness for curves\\
\end{eulercomment}
\begin{eulerttcomment}
            This value can be smaller than 1 for very thin lines.
\end{eulerttcomment}
\begin{eulercomment}
style     : Plot style for lines, markers, and fills.\\
\end{eulercomment}
\begin{eulerttcomment}
            For points use
            "[]", "<>", ".", "..", "...",
            "*", "+", "|", "-", "o"
            "[]#", "<>#", "o#" (filled shapes)
            "[]w", "<>w", "ow" (non-transparent)
            For lines use
            "-", "--", "-.", ".", ".-.", "-.-", "->"
            For filled polygons or bar plots use
            "#", "#O", "O", "/", "\(\backslash\)", "\(\backslash\)/",
            "+", "|", "-", "t"
\end{eulerttcomment}
\begin{eulercomment}
points    : plot single points instead of line segments\\
addpoints : if true, plots line segments and points\\
add       : add the plot to the existing plot\\
user      : enable user interaction for functions\\
delta     : step size for user interaction\\
bar       : bar plot (x are the interval bounds, y the interval values)\\
histogram : plots the frequencies of x in n subintervals\\
distribution=n : plots the distribution of x with n subintervals\\
even      : use inter values for automatic histograms.\\
steps     : plots the function as a step function (steps=1,2)\\
adaptive  : use adaptive plots (n is the minimal number of steps)\\
level     : plot level lines of an implicit function of two variables\\
outline   : draws boundary of level ranges.
\\
If the level value is a 2xn matrix, ranges of levels will be drawn\\
in the color using the given fill style. If outline is true, it\\
will be drawn in the contour color. Using this feature, regions of\\
f(x,y) between limits can be marked.
\\
hue       : add hue color to the level plot to indicate the function\\
\end{eulercomment}
\begin{eulerttcomment}
            value
\end{eulerttcomment}
\begin{eulercomment}
contour   : Use level plot with automatic levels\\
nc        : number of automatic level lines\\
title     : plot title (default "")\\
xl, yl    : labels for the x- and y-axis\\
smaller   : if \textgreater{}0, there will be more space to the left for labels.\\
vertical  :\\
\end{eulercomment}
\begin{eulerttcomment}
  Turns vertical labels on or off. This changes the global variable
  verticallabels locally for one plot. The value 1 sets only vertical
  text, the value 2 uses vertical numerical labels on the y axis.
\end{eulerttcomment}
\begin{eulercomment}
filled    : fill the plot of a curve\\
fillcolor : fill color for bar and filled curves\\
outline   : boundary for filled polygons\\
logplot   : set logarithmic plots\\
\end{eulercomment}
\begin{eulerttcomment}
            1 = logplot in y,
            2 = logplot in xy,
            3 = logplot in x
\end{eulerttcomment}
\begin{eulercomment}
own       :\\
\end{eulercomment}
\begin{eulerttcomment}
  A string, which points to an own plot routine. With >user, you get
  the same user interaction as in plot2d. The range will be set
  before each call to your function.
\end{eulerttcomment}
\begin{eulercomment}
maps      : map expressions (0 is faster), functions are always mapped.\\
contourcolor : color of contour lines\\
contourwidth : width of contour lines\\
clipping  : toggles the clipping (default is true)\\
title     :\\
\end{eulercomment}
\begin{eulerttcomment}
  This can be used to describe the plot. The title will appear above
  the plot. Moreover, a label for the x and y axis can be added with
  xl="string" or yl="string". Other labels can be added with the
  functions label() or labelbox(). The title can be a unicode
  string or an image of a Latex formula.
\end{eulerttcomment}
\begin{eulercomment}
cgrid     :\\
\end{eulercomment}
\begin{eulerttcomment}
  Determines the number of grid lines for plots of complex grids.
  Should be a divisor of the the matrix size minus 1 (number of
  subintervals). cgrid can be a vector [cx,cy].
\end{eulerttcomment}
\begin{eulercomment}

Overview

The function can plot

- expressions, call collections or functions of one variable,\\
- parametric curves,\\
- x data against y data,\\
- implicit functions,\\
- bar plots,\\
- complex grids,\\
- polygons.

If a function or expression for xv is given, plot2d() will compute\\
values in the given range using the function or expression. The\\
expression must be an expression in the variable x. The range must\\
be defined in the parameters a and b unless the default range\\
[-2,2] should be used. The y-range will be computed automatically,\\
unless c and d are specified, or a radius r, which yields the range\\
[-r,r] for x and y. For plots of functions, plot2d will use an\\
adaptive evaluation of the function by default. To speed up the\\
plot for complicated functions, switch this off with \textless{}adaptive, and\\
optionally decrease the number of intervals n. Moreover, plot2d()\\
will by default use mapping. I.e., it will compute the plot element\\
for element. If your expression or your functions can handle a\\
vector x, you can switch that off with \textless{}maps for faster evaluation.

Note that adaptive plots are always computed element for element. \\
If functions or expressions for both xv and for yv are specified,\\
plot2d() will compute a curve with the xv values as x-coordinates\\
and the yv values as y-coordinates. In this case, a range should be\\
defined for the parameter using xmin, xmax. Expressions contained\\
in strings must always be expressions in the parameter variable x.
\end{eulercomment}
\end{eulernotebook}
\end{document}
